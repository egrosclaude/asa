
\documentclass[11pt,a4paper]{article}
\usepackage[utf8x]{inputenc}
\usepackage[T1]{fontenc}
\usepackage[spanish]{babel}
\usepackage{amsmath}
\usepackage{amssymb,amsfonts,textcomp}
\usepackage{color}
\usepackage{array}
\usepackage{multirow}
\usepackage{hhline}
\usepackage{hyperref}
\usepackage{float}
\usepackage{xkeyval}
\usepackage[pdftex]{graphicx}
\usepackage[yyyymmdd,hhmmss]{datetime}
\usepackage[usenames,dvipsnames]{xcolor}
\usepackage{appendix}
\usepackage{listings}
\definecolor{dkgreen}{rgb}{0,0.6,0}
\definecolor{gray}{rgb}{0.5,0.5,0.5}
\definecolor{mauve}{rgb}{0.58,0,0.82}
\definecolor{lstbackground}{rgb}{0.90,0.90,0.90}
% \lstset{frame=tb,
% 	backgroundcolor=\color{lstbackground},
%   language=Bash,
%   aboveskip=3mm,
%   belowskip=3mm,
%   showstringspaces=false,
%   columns=flexible,
%   basicstyle={\small\ttfamily},
%   numbers=none,
%   numberstyle=\tiny\color{gray},
%   keywordstyle=\color{blue},
%   %commentstyle=\color{dkgreen},
%   stringstyle=\color{mauve},
%   breaklines=true,
%   breakatwhitespace=true,
%   extendedchars=true,
%   tabsize=3
% }
\lstset{frame=tb,
	backgroundcolor=\color{lstbackground},
%  language=Bash,
  aboveskip=3mm,
  belowskip=3mm,
  showstringspaces=false,
  columns=flexible,
  basicstyle={\small\ttfamily},
  numbers=none,
%  numberstyle=\tiny\color{gray},
%  keywordstyle=\color{blue},
%  commentstyle=\color{dkgreen},
%  stringstyle=\color{mauve},
  breaklines=true,
  breakatwhitespace=true
  tabsize=4
}
\usepackage{caption}

%\DeclareCaptionFont{black}{ \color{black} }
%\DeclareCaptionFormat{listing}{
%  \colorbox[cmyk]{0.93, 0.95, 0.95,0.01 }{
%    \parbox{\textwidth}{\hspace{15pt}#1#2#3}
%  }
%}
%\captionsetup[lstlisting]{ format=listing, labelfont=black, textfont=black, singlelinecheck=false, margin=0pt, font={bf,footnotesize} }
\captionsetup[lstlisting]{format=plain, font={footnotesize}}
% ...


%\renewcommand{\lstlistingname}{Code}
\usepackage{verbatim}
\begin{comment}
	\hypersetup{
		pdftex, 
		colorlinks=true, 
		linkcolor=blue, 
		citecolor=blue, 
		filecolor=blue, 
		urlcolor=blue, 
	pdftitle={Software Libre}, 
	pdfauthor={Eduardo Grosclaude}, 
	pdfsubject={Documento de la materia Software Libre}, 
	pdfkeywords={Software Libre, Tecnicatura en Administración de Sistemas y 		Software Libre, Universidad Nacional del Comahue}
	}
\end{comment}	



%\addto\captionsspanish {%
%	\def\appendixname{Apéndices}
%}
% Outline numbering
\setcounter{secnumdepth}{1}
% Reset section numbering between parts
\makeatletter
\@addtoreset{section}{part}
\makeatother  
% List styles
\newcommand\liststyleLi{%
\renewcommand\labelitemi{\tiny${\blacksquare}$}
\renewcommand\labelitemii{\tiny${\square}$}
\renewcommand\labelitemiii{\tiny${\circ}$}
\renewcommand\labelitemiv{\tiny${\circ}$}
}
\newcommand\liststyleLii{%
\renewcommand\labelitemi{{\textbullet}}
\renewcommand\labelitemii{${\circ}$}
\renewcommand\labelitemiii{${\blacksquare}$}
\renewcommand\labelitemiv{{\textbullet}}
}
\newcommand\liststyleLiii{%
\renewcommand\labelitemi{{\textbullet}}
\renewcommand\labelitemii{${\circ}$}
\renewcommand\labelitemiii{${\blacksquare}$}
\renewcommand\labelitemiv{{\textbullet}}
}

\liststyleLi

% Page layout (geometry)
\setlength\voffset{-1in}
\setlength\hoffset{-1in}
\setlength\topmargin{2cm}
\setlength\oddsidemargin{2cm}
\setlength\textheight{23.246668cm}
\setlength\textwidth{17.006cm}
\setlength\footskip{26.144882pt}
\setlength\headheight{1.016cm}
\setlength\headsep{0.508cm}
% Footnote rule
\setlength{\skip\footins}{0.119cm}
\renewcommand\footnoterule{\vspace*{-0.018cm}\setlength\leftskip{0pt}\setlength\rightskip{0pt plus 1fil}\noindent\textcolor{black}{\rule{0.25\columnwidth}{0.018cm}}\vspace*{0.101cm}}
% Pages styles
\makeatletter
\newcommand\ps@Standard{
  \renewcommand\@oddhead{{\raggedleft Cabecera \ } {\raggedright \thepage{}}}
  \renewcommand\@evenhead{\@oddhead}
  \renewcommand\@oddfoot{}
  \renewcommand\@evenfoot{\@oddfoot}
  \renewcommand\thepage{\arabic{page}}
}

% \pagestyle{Standard}
\usepackage{fancyhdr}
\pagestyle{fancy}


%%--------------------------------------
% F O N T S 
% \usepackage{dejavu}
%\usepackage{librebaskerville}
\usepackage{sans}
%\usepackage{libertine}
%\usepackage{lmodern}
%\usepackage{opensans}
%\usepackage{helvet}
%\usepackage{times}

%% LaTeX Preamble - Font choices
%% Each block selects new math, roman (serif), sans serif, and typewriter fonts.
%% Delete or comment out all but one to make your choice.

% Fourier for math | Utopia (scaled) for rm | Helvetica for ss | Latin Modern for tt
%\usepackage{fourier} % math & rm
%\usepackage[scaled=0.875]{helvet} % ss
%\renewcommand{\ttdefault}{lmtt} %tt

% Latin Modern (similar to CM with more characters)
%\usepackage{lmodern} % math, rm, ss, tt
%\usepackage[T1]{fontenc}

% Palatino for rm and math | Helvetica for ss | Courier for tt
%\usepackage{mathpazo} % math & rm
%\linespread{1.05}        % Palatino needs more leading (space between lines)
%\usepackage[scaled]{helvet} % ss
%\usepackage{courier} % tt
%\normalfont
%\usepackage[T1]{fontenc}

% Euler for math | Palatino for rm | Helvetica for ss | Courier for tt
%\renewcommand{\rmdefault}{ppl} % rm
%\linespread{1.05}        % Palatino needs more leading
%\usepackage[scaled]{helvet} % ss
%\usepackage{courier} % tt
%\usepackage{euler} % math
%\usepackage{eulervm} % a better implementation of the euler package (not in gwTeX)

%\normalfont
%\usepackage[T1]{fontenc}

% Times for rm and math | Helvetica for ss | Courier for tt
%\usepackage{mathptmx} % rm & math
%\usepackage[scaled=0.90]{helvet} % ss
%\usepackage{courier} % tt
%\normalfont
%\usepackage[T1]{fontenc}

% !! COMMERICAL FONT !! Lucida Bright (w/expert package)
%\usepackage[T1]{fontenc}
%\usepackage[expert,vargreek,altbullet]{lucidabr}

%% END Font choices
%%---------------------------------------------
% \renewcommand*\familydefault{\sfdefault}
% \pagestyle{Standard}
\usepackage{mdframed}


% footnotes configuration
\makeatletter
\renewcommand\thefootnote{\arabic{footnote}}
\makeatother
\title{Administración de Sistemas Avanzada}
\author{Eduardo Grosclaude}
\date{2014-08-11}
\usepackage{graphicx}

\usepackage{xkeyval}
\usepackage{pifont}
\usepackage{xcolor}
\newcommand{\revisar}[1]{{\color{red}[#1]}}
%\newcommand{\nota}[1]{{\color{red}[#1]}}
%\newcommand{\revisar}[1]{}

\newcommand{\borrador}{
\revisar{\today, \currenttime  -  Material en preparación, se ruega no imprimir mientras aparezca esta nota}
}




\newcommand{\nota}[1]{}

\newcommand{\nonota}[1]{#1}

\newcommand{\quotes}[1]{``#1''}

   
\newcommand{\shade}[1]{\textcolor{black!50}{#1}}

% ancho opcional, por defecto 15cm
% \figura{copyleft}{Símbolo de Copyleft}{copyleft.png}
% \figura[6]{copyleft}{Símbolo de Copyleft}{copyleft.png}
\newcommand{\figura}[4][15]{
 \begin{figure}[htbp] 
 \centering 
 \includegraphics[width=#1cm]{./img/#4} 
 \caption{#3} 
 \label{fig:#2} 
 \end{figure} 
}

% tabla{label}{caption}{columns}{contents}
\newcommand{\tabla}[4]{
 \begin{table} 
 \centering 
 \small
 \begin{tabular}{#3}
 #4
 \end{tabular}
 \caption{#2}
 \label{tab:#1} 
 \end{table} 
}

\newcommand{\recuadro}[1]{
\begin{minipage}[c]{0.84\textwidth}
\begin{mdframed}
#1
\end{mdframed}
\end{minipage}
}

\newcommand{\code}[1]{\lstinline$#1$}
