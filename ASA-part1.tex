
\section{Objetivos}
\subsection{De la carrera}
Según el documento fundamental de la Tecnicatura, el Técnico Superior en Administración de Sistemas y Software Libre estará capacitado para:
\begin{itemize}
	\item Desarrollar actividades de administración de infraestructura. Comprendiendo la administración de sistemas, redes y los distintos componentes que forman la
infraestructura de tecnología de una institución, ya sea pública o privada.
	\item Aportar criterios básicos para la toma de decisiones relativas a la adopción de nuevas tecnologías libres.
	\item Desempeñarse como soporte técnico, solucionando problemas afines por medio de la comunicación con comunidades de Software Libre, empresas y desarrolladores de
software.
	\item Realizar tareas de trabajo en modo colaborativo, intrínseco al uso de tecnologías libres.
	\item Comprender y adoptar el estado del arte local, nacional y regional en lo referente a implementación de tecnologías libres. Tanto en los aspectos técnicos como legales.
\end{itemize}
\subsection{De la asignatura}

\begin{itemize}
	\item Saber implementar configuraciones especiales de almacenamiento
	\item Saber aplicar programación avanzada a la automatización de tareas
	\item Saber diseñar e implementar estrategias de respaldo 
	\item Conocer formas de implementar estrategias de tolerancia a fallos para servicios críticos
\end{itemize}


\section{Cursado}
\begin{itemize}
	\item Cuatrimestral de 16 semanas, 128 horas totales
	\item Clases teórico-prácticas presenciales
	\item Promocionable con trabajos prácticos
\end{itemize}


\section {Contenidos}
\subsection{Contenidos mínimos}
\begin{itemize}
	\item  Instalación sobre configuraciones de almacenamiento especiales. 
	\item  Scripting avanzado. 
	\item  Planificación de tareas. 
	\item  Virtualización. 
	\item  Alta Disponibilidad.
\end{itemize}


\subsection {Programa}
\begin{enumerate}
\item Scripting avanzado
\begin{itemize}
	\item Estructuras de programación
	\item Scripting para tratamiento de archivos
	\item Planificación de tareas
\end{itemize}

\item Configuraciones de almacenamiento
\begin{itemize}
	\item Arquitectura de E/S, Dispositivos de E/S, Filesystems
	\item	Diseños típicos de almacenamiento
	\item	Software RAID, instalación y mantenimiento niveles 0, 1, 10
	\item	LVM, instalación y mantenimiento	 
\end{itemize}
	
\item Estrategias de respaldo
\begin{itemize}
	\item Copiado y sincronización de archivos
	\item Estrategias y herramientas de backup, LVM snapshots
	\item Control de versiones
\end{itemize}
\item Virtualización
\begin{itemize}
	\item Formas de virtualización, herramientas. KVM, Proxmox, otras
	\item Creación, instalación, migración de MV
	\item Cloud. IaaS, PaaS, SaaS, etc.
\end{itemize}
\item Alta Disponibilidad
\begin{itemize}
	\item Clustering de LB, de HA, de HPC. Conceptos de HA.
	\item Balance de Carga
	\item Heartbeat, DRBD, Clustering de aplicaciones
	\item Alta Disponibilidad en Redes. Bonding, STP
\end{itemize}
\end{enumerate}

\section {Bibliografía inicial}
\begin{itemize}
\item Kemp, Juliet. Linux System Administration Recipes: A Problem-Solution Approach. Apress, 2009. 
\item Lakshman, Sarath. Linux Shell Scripting Cookbook Solve Real-World Shell Scripting Problems with over 110 Simple but Incredibly Effective Recipes. Birmingham, U.K.: Packt Pub., 2011. 
\item Parker, Steve. Shell Scripting Expert Recipes for Linux, Bash, and More. Hoboken, N.J.; Chichester: Wiley; John Wiley, 2011.
\item Quigley, Ellie. UNIX Shells by Example. 3rd ed. Upper Saddle River, NJ: Prentice Hall, 2002.
\end{itemize}



