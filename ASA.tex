
\documentclass[11pt,a4paper]{article}
\usepackage[utf8x]{inputenc}
\usepackage[T1]{fontenc}
\usepackage[spanish]{babel}
\usepackage{amsmath}
\usepackage{amssymb,amsfonts,textcomp}
\usepackage{color}
\usepackage{array}
\usepackage{multirow}
\usepackage{hhline}
\usepackage{hyperref}
\usepackage{float}
\usepackage{xkeyval}
\usepackage[pdftex]{graphicx}
\usepackage[yyyymmdd,hhmmss]{datetime}
\usepackage[usenames,dvipsnames]{xcolor}
\usepackage{appendix}
\usepackage{listings}
\definecolor{dkgreen}{rgb}{0,0.6,0}
\definecolor{gray}{rgb}{0.5,0.5,0.5}
\definecolor{mauve}{rgb}{0.58,0,0.82}
\definecolor{lstbackground}{rgb}{0.90,0.90,0.90}
% \lstset{frame=tb,
% 	backgroundcolor=\color{lstbackground},
%   language=Bash,
%   aboveskip=3mm,
%   belowskip=3mm,
%   showstringspaces=false,
%   columns=flexible,
%   basicstyle={\small\ttfamily},
%   numbers=none,
%   numberstyle=\tiny\color{gray},
%   keywordstyle=\color{blue},
%   %commentstyle=\color{dkgreen},
%   stringstyle=\color{mauve},
%   breaklines=true,
%   breakatwhitespace=true,
%   extendedchars=true,
%   tabsize=3
% }
\lstset{frame=tb,
	backgroundcolor=\color{lstbackground},
%  language=Bash,
  aboveskip=3mm,
  belowskip=3mm,
  showstringspaces=false,
  columns=flexible,
  basicstyle={\small\ttfamily},
  numbers=none,
%  numberstyle=\tiny\color{gray},
%  keywordstyle=\color{blue},
%  commentstyle=\color{dkgreen},
%  stringstyle=\color{mauve},
  breaklines=true,
  breakatwhitespace=true
  tabsize=4
}
\usepackage{caption}

%\DeclareCaptionFont{black}{ \color{black} }
%\DeclareCaptionFormat{listing}{
%  \colorbox[cmyk]{0.93, 0.95, 0.95,0.01 }{
%    \parbox{\textwidth}{\hspace{15pt}#1#2#3}
%  }
%}
%\captionsetup[lstlisting]{ format=listing, labelfont=black, textfont=black, singlelinecheck=false, margin=0pt, font={bf,footnotesize} }
\captionsetup[lstlisting]{format=plain, font={footnotesize}}
% ...


%\renewcommand{\lstlistingname}{Code}
\usepackage{verbatim}
\begin{comment}
	\hypersetup{
		pdftex, 
		colorlinks=true, 
		linkcolor=blue, 
		citecolor=blue, 
		filecolor=blue, 
		urlcolor=blue, 
	pdftitle={Software Libre}, 
	pdfauthor={Eduardo Grosclaude}, 
	pdfsubject={Documento de la materia Software Libre}, 
	pdfkeywords={Software Libre, Tecnicatura en Administración de Sistemas y 		Software Libre, Universidad Nacional del Comahue}
	}
\end{comment}	



%\addto\captionsspanish {%
%	\def\appendixname{Apéndices}
%}
% Outline numbering
\setcounter{secnumdepth}{1}
% Reset section numbering between parts
\makeatletter
\@addtoreset{section}{part}
\makeatother  
% List styles
\newcommand\liststyleLi{%
\renewcommand\labelitemi{\tiny${\blacksquare}$}
\renewcommand\labelitemii{\tiny${\square}$}
\renewcommand\labelitemiii{\tiny${\circ}$}
\renewcommand\labelitemiv{\tiny${\circ}$}
}
\newcommand\liststyleLii{%
\renewcommand\labelitemi{{\textbullet}}
\renewcommand\labelitemii{${\circ}$}
\renewcommand\labelitemiii{${\blacksquare}$}
\renewcommand\labelitemiv{{\textbullet}}
}
\newcommand\liststyleLiii{%
\renewcommand\labelitemi{{\textbullet}}
\renewcommand\labelitemii{${\circ}$}
\renewcommand\labelitemiii{${\blacksquare}$}
\renewcommand\labelitemiv{{\textbullet}}
}

\liststyleLi

% Page layout (geometry)
\setlength\voffset{-1in}
\setlength\hoffset{-1in}
\setlength\topmargin{2cm}
\setlength\oddsidemargin{2cm}
\setlength\textheight{23.246668cm}
\setlength\textwidth{17.006cm}
\setlength\footskip{26.144882pt}
\setlength\headheight{1.016cm}
\setlength\headsep{0.508cm}
% Footnote rule
\setlength{\skip\footins}{0.119cm}
\renewcommand\footnoterule{\vspace*{-0.018cm}\setlength\leftskip{0pt}\setlength\rightskip{0pt plus 1fil}\noindent\textcolor{black}{\rule{0.25\columnwidth}{0.018cm}}\vspace*{0.101cm}}
% Pages styles
\makeatletter
\newcommand\ps@Standard{
  \renewcommand\@oddhead{{\raggedleft Cabecera \ } {\raggedright \thepage{}}}
  \renewcommand\@evenhead{\@oddhead}
  \renewcommand\@oddfoot{}
  \renewcommand\@evenfoot{\@oddfoot}
  \renewcommand\thepage{\arabic{page}}
}

% \pagestyle{Standard}
\usepackage{fancyhdr}
\pagestyle{fancy}


%%--------------------------------------
% F O N T S 
%\usepackage{fancyhdr}
\usepackage{sans}
%\usepackage{libertine}
%\usepackage{lmodern}
%\usepackage{opensans}
%\usepackage{helvet}

%\usepackage{times}

%% LaTeX Preamble - Font choices
%% Each block selects new math, roman (serif), sans serif, and typewriter fonts.
%% Delete or comment out all but one to make your choice.

% Fourier for math | Utopia (scaled) for rm | Helvetica for ss | Latin Modern for tt
%\usepackage{fourier} % math & rm
%\usepackage[scaled=0.875]{helvet} % ss
%\renewcommand{\ttdefault}{lmtt} %tt

% Latin Modern (similar to CM with more characters)
%\usepackage{lmodern} % math, rm, ss, tt
%\usepackage[T1]{fontenc}

% Palatino for rm and math | Helvetica for ss | Courier for tt
%\usepackage{mathpazo} % math & rm
%\linespread{1.05}        % Palatino needs more leading (space between lines)
%\usepackage[scaled]{helvet} % ss
%\usepackage{courier} % tt
%\normalfont
%\usepackage[T1]{fontenc}

% Euler for math | Palatino for rm | Helvetica for ss | Courier for tt
%\renewcommand{\rmdefault}{ppl} % rm
%\linespread{1.05}        % Palatino needs more leading
%\usepackage[scaled]{helvet} % ss
%\usepackage{courier} % tt
%\usepackage{euler} % math
%\usepackage{eulervm} % a better implementation of the euler package (not in gwTeX)

%\normalfont
%\usepackage[T1]{fontenc}

% Times for rm and math | Helvetica for ss | Courier for tt
%\usepackage{mathptmx} % rm & math
%\usepackage[scaled=0.90]{helvet} % ss
%\usepackage{courier} % tt
%\normalfont
%\usepackage[T1]{fontenc}

% !! COMMERICAL FONT !! Lucida Bright (w/expert package)
%\usepackage[T1]{fontenc}
%\usepackage[expert,vargreek,altbullet]{lucidabr}

%% END Font choices
%%---------------------------------------------
% \renewcommand*\familydefault{\sfdefault}
% \pagestyle{Standard}
\usepackage{mdframed}


% footnotes configuration
\makeatletter
\renewcommand\thefootnote{\arabic{footnote}}
\makeatother
\title{Administración de Sistemas Avanzada}
\author{Eduardo Grosclaude}
\date{2014-08-11}
\usepackage{graphicx}

\usepackage{xkeyval}
\usepackage{pifont}
\usepackage{xcolor}
\newcommand{\revisar}[1]{{\color{red}[#1]}}
%\newcommand{\nota}[1]{{\color{red}[#1]}}
%\newcommand{\revisar}[1]{}

\newcommand{\borrador}{
\revisar{\today, \currenttime  -  Material en preparación, se ruega no imprimir mientras aparezca esta nota}
}




\newcommand{\nota}[1]{}

\newcommand{\nonota}[1]{#1}

\newcommand{\quotes}[1]{``#1''}

   
\newcommand{\shade}[1]{\textcolor{black!50}{#1}}

% ancho opcional, por defecto 15cm
% \figura{copyleft}{Símbolo de Copyleft}{copyleft.png}
% \figura[6]{copyleft}{Símbolo de Copyleft}{copyleft.png}
\newcommand{\figura}[4][15]{
 \begin{figure}[htbp] 
 \centering 
 \includegraphics[width=#1cm]{./img/#4} 
 \caption{#3} 
 \label{fig:#2} 
 \end{figure} 
}

% tabla{label}{caption}{columns}{contents}
\newcommand{\tabla}[4]{
 \begin{table} 
 \centering 
 \small
 \begin{tabular}{#3}
 #4
 \end{tabular}
 \caption{#2}
 \label{tab:#1} 
 \end{table} 
}

\newcommand{\recuadro}[1]{
\begin{minipage}[c]{0.84\textwidth}
\begin{mdframed}
#1
\end{mdframed}
\end{minipage}
}


\hypersetup{colorlinks=true, linkcolor=blue, citecolor=blue, filecolor=blue, urlcolor=blue, 
	pdftitle={Administración de Sistemas Avanzada}, 
	pdfauthor={Eduardo Grosclaude}, 
	pdfsubject={Documento de la materia Administración de Sistemas Avanzada}, 
	pdfkeywords={Administración de Sistemas, Alta Disponibilidad, Virtualización, Almacenamiento, Tecnicatura en Administración de Sistemas y Software Libre, Universidad Nacional del Comahue}}

 
% --------------------------------------------------------------------
\begin{document}

\maketitle

\borrador

\abstract {En este escrito se presenta la descripción y material inicial de la asignatura \textbf{Administración de Sistemas Avanzada}, para la carrera de Tecnicatura Universitaria en Administración de Sistemas y Software Libre, de la Universidad Nacional del Comahue. 

La materia es cuatrimestral en modalidad presencial y las clases son de carácter teórico-práctico, desarrolladas en forma colaborativa. Está preparada con los objetivos generales de capacitar al estudiante para  \textbf{implementar configuraciones especiales de almacenamiento, aplicar programación avanzada a la automatización de tareas, y diseñar e implementar estrategias de respaldo y de tolerancia a fallos para servicios críticos}. 
 

\newpage
\emph{}
\newpage

\tableofcontents

\newpage 
\emph{}

%---------- P R E S E N T A C I O N  ---------

\newpage
\part {La asignatura}


\section{Objetivos}
\subsection{De la carrera}
Según el documento fundamental de la Tecnicatura, el Técnico Superior en Administración de Sistemas y Software Libre estará capacitado para:
\begin{itemize}
	\item Desarrollar actividades de administración de infraestructura. Comprendiendo la administración de sistemas, redes y los distintos componentes que forman la
infraestructura de tecnología de una institución, ya sea pública o privada.
	\item Aportar criterios básicos para la toma de decisiones relativas a la adopción de nuevas tecnologías libres.
	\item Desempeñarse como soporte técnico, solucionando problemas afines por medio de la comunicación con comunidades de Software Libre, empresas y desarrolladores de
software.
	\item Realizar tareas de trabajo en modo colaborativo, intrínseco al uso de tecnologías libres.
	\item Comprender y adoptar el estado del arte local, nacional y regional en lo referente a implementación de tecnologías libres. Tanto en los aspectos técnicos como legales.
\end{itemize}
\subsection{De la asignatura}

\begin{itemize}
	\item Saber implementar configuraciones especiales de almacenamiento
	\item Saber aplicar programación avanzada a la automatización de tareas
	\item Saber diseñar e implementar estrategias de respaldo 
	\item Conocer formas de implementar estrategias de tolerancia a fallos para servicios críticos
\end{itemize}


\section{Cursado}
\begin{itemize}
	\item Cuatrimestral de 16 semanas, 128 horas totales
	\item Clases teórico-prácticas presenciales
	\item Promocionable con trabajos prácticos
\end{itemize}


\section {Contenidos}
\subsection{Contenidos mínimos}
\begin{itemize}
	\item  Instalación sobre configuraciones de almacenamiento especiales. 
	\item  Scripting avanzado. 
	\item  Planificación de tareas. 
	\item  Virtualización. 
	\item  Alta Disponibilidad.
\end{itemize}


\subsection {Programa}
\begin{enumerate}
\item Scripting avanzado
\begin{itemize}
	\item Estructuras de programación
	\item Scripting para tratamiento de archivos
	\item Planificación de tareas
\end{itemize}

\item Configuraciones de almacenamiento
\begin{itemize}
	\item Arquitectura de E/S, Dispositivos de E/S, Filesystems
	\item	Diseños típicos de almacenamiento
	\item	Software RAID, instalación y mantenimiento niveles 0, 1, 10
	\item	LVM, instalación y mantenimiento	 
\end{itemize}
	
\item Estrategias de respaldo
\begin{itemize}
	\item Copiado y sincronización de archivos
	\item Estrategias y herramientas de backup, LVM snapshots
	\item Control de versiones
\end{itemize}
\item Virtualización
\begin{itemize}
	\item Formas de virtualización, herramientas. KVM, Proxmox, otras
	\item Creación, instalación, migración de MV
	\item Cloud. IaaS, PaaS, SaaS, etc.
\end{itemize}
\item Alta Disponibilidad
\begin{itemize}
	\item Clustering de LB, de HA, de HPC. Conceptos de HA.
	\item Balance de Carga
	\item Heartbeat, DRBD, Clustering de aplicaciones
	\item Alta Disponibilidad en Redes. Bonding, STP
\end{itemize}
\end{enumerate}

\section {Bibliografía inicial}
\begin{itemize}
\item Kemp, Juliet. Linux System Administration Recipes: A Problem-Solution Approach. Apress, 2009. 
\item Lakshman, Sarath. Linux Shell Scripting Cookbook Solve Real-World Shell Scripting Problems with over 110 Simple but Incredibly Effective Recipes. Birmingham, U.K.: Packt Pub., 2011. 
\item Parker, Steve. Shell Scripting Expert Recipes for Linux, Bash, and More. Hoboken, N.J.; Chichester: Wiley; John Wiley, 2011.
\item Quigley, Ellie. UNIX Shells by Example. 3rd ed. Upper Saddle River, NJ: Prentice Hall, 2002.
\item W. Soyinka, Linux administration a beginners guide. New York, NY: McGraw-Hill Osborne Media, 2012.
\item C. Wolf and E. M. Halter, Virtualization from the desktop to the enterprise. Berkeley, CA; New York, NY: Apress; Distributed in U.S. by Springer-Verlag New York, 2005.



\end{itemize}



%
\section{Evaluación}
La evaluación de la materia se realizará mediante trabajos grupales de investigación y desarrollo sobre proyectos de Software Libre, de la siguiente manera.
\begin{itemize}
	\item Los estudiantes se dividirán en grupos de 2 a 5 personas. 
	\item Los grupos desarrollarán trabajos prácticos en etapas que se distribuirán a lo largo de la materia. 
	\item Cada grupo abrirá un diario, blog o wiki de acceso público en cualquier sitio disponible y publicará, mediante el Foro de la materia, la forma de acceder al diario para lectura. Los docentes y los demás estudiantes de la materia podrán acceder al diario del grupo para lectura. Todo cambio en la dirección o forma de acceso deberá ser informado mediante el Foro.
	\item El grupo irá aportando los resultados de cada etapa de los trabajos a su diario, y periódicamente comentará además en clase las experiencias surgidas durante la realización de los trabajos.
	\item El material publicado en el diario será reunido en un documento final que será entregado \textbf{en formato electrónico} al finalizar la materia. El documento indicará tema del trabajo, resumen, integrantes del grupo, desarrollo y conclusiones. 
	\item El documento será acompañado por una presentación de no más de treinta minutos que será expuesta según el cronograma adjunto. 
	\item La acreditación final tendrá en cuenta la calidad del material aportado al diario por el grupo, la calidad de los documentos finales de los trabajos, la presentación oral y la participación en clase ofreciendo la experiencia adquirida durante la realización de los trabajos.
\end{itemize}

\subsection {Trabajo I - Colaboración con proyectos libres}
\subsubsection{Etapa 1}  
Descargar e instalar software ofrecido por un proyecto de Software Libre que esté en actividad (puede tratarse de un entorno de escritorio, un programa de sistema, programas de usuario final, una distribución completa, etc.). Familiarizarse con el software utilizándolo. 
\subsubsection{Etapa 2} 
Basándose en el conocimiento adquirido con el uso del software, colaborar de alguna forma con el proyecto que lo origina: 
\begin{itemize}
	\item traduciendo o localizando parte del software,
	\item generando documentación faltante, 
	\item traduciendo parte de la documentación, 
	\item detectando y denunciando errores en el software o en la documentación,
	\item aportando, modificando o corrigiendo código,
	\item aportando conocimiento a los usuarios del proyecto en blogs, salas de chat, bases de conocimiento, etc.
\end{itemize}
Puede abordarse cualquier cantidad manejable de proyectos. La colaboración debe consistir en alguna interacción positiva y completa con cada proyecto. El grupo incorporará al diario los reportes que acrediten esa interacción. Cuando no sea posible realizar o completar la interacción se indicarán las causas, y las acciones realizadas.

El aporte al proyecto debe efectuarse por los canales establecidos por el proyecto. Si se trata de documentación, respetar el formato utilizado; si es el reporte de un error, hacerlo por la vía preferida por el proyecto, etc.

\subsubsection{Etapa 3} 
El grupo entregará un documento conteniendo la historia de las interacciones con cada proyecto, adjuntando las pruebas en anexos y ofrecerá una presentación.

\subsection {Trabajo II - Evaluación de proyectos libres}

\subsubsection{Etapa 1} 
El grupo enunciará un determinado requerimiento concreto de software que puede ser presentado por un empleador. Algunos ejemplos posibles son:
\begin{itemize}
	\item \quotes{un servidor de correo electrónico que maneje listas},
	\item  \quotes{una aplicación de control de asistencia para empleados},
	\item  \quotes{un sistema de edición de textos para traductores},
	\item  \quotes{un sistema de gestión de contenidos web que incluya workflow}, 
	\item \quotes{un motor de juegos 2D para crear juegos que asistan en la enseñanza de matemática},
	\item  \quotes{un programa de simulación de ataques para evaluar postura de seguridad}, 
	\item \quotes{un sistema de control de stock para zapaterías},
	\item \quotes{una distribución de GNU/Linux para escuelas de arte},
	\item \quotes{una distribución para sistemas empotrados}, etc.
\end{itemize}
El grupo debe comprender el propósito del software requerido y debe contar con al menos un integrante con conocimiento razonable de la temática involucrada. El grupo escribirá una entrada en el diario consignando toda la información posible sobre los requerimientos. 

\subsubsection{Etapa 2} 

\begin{itemize}
	\item El grupo $n$ (en adelante \quotes{el proveedor}) tomará a su cargo el requerimiento del grupo $n+1$ (en adelante \quotes{el cliente}), y se atendrá a dicha descripción para el resto del trabajo. 
	\item El grupo proveedor buscará proyectos de SL que apunten a cubrir esos requerimientos y seleccionará al menos dos proyectos, idealmente tres, de entre ellos.
\end{itemize}

\subsubsection{Etapa 3}
Los proyectos serán comparados en función de varios parámetros o dimensiones.
\begin{itemize}
	\item  ajuste a los requerimientos (actual, previsto o potencial),
	\item  licenciamiento, 
	\item  motivación del desarrollo, 
	\item  modelos de negocio del proyecto, 
	\item  tamaño y permanencia de la comunidad,
	\item  dinámica de soporte, 
	\item  dinámica de actualizaciones y mejoras del software.
\end{itemize}

Se podrán agregar a la comparación uno o más desarrollos no libres. 

Las dudas sobre detalles de los requerimientos serán dirigidas al grupo cliente, y contestadas por aquél, mediante el Foro de la página de la materia.  
\subsubsection{Etapa 4} 
El grupo entregará un documento conteniendo la comparación y haciendo una recomendación final, explicando sus fundamentos. Deberán volcar en el trabajo lo que se vaya aprendiendo durante el curso de la materia, en cada uno de los parámetros o dimensiones nombrados. Finalmente ofrecerán una presentación sobre el trabajo.

\label{sub:acreditacion}

\subsection {Cronograma de ejecución}
\begin{tabular}{c|l|l|l}
Semana & Unidad & Trabajo I & Trabajo II\\
\hline
\hline
1	& 	1. Introducción, Software Libre & Etapa 1 &  \\
2 	& 								 	& \\
\hline
\hline
3	& 	2. Aspectos técnicos			& Etapa 2 &  \\
4 	& 									&\\
5	& 									&\\
6	& 									&\\
\hline
\hline
7 	& 	3. Aspectos legales				& Etapa 3 \\
8	& 									& Entrega y presentaciones\\ 
9	& 									& & Etapas 1 y 2\\
\hline
\hline
10	& 	4. Uso de SL					&& Etapa 3\\ 
11	& 									& \\
12	& 									&\\
13	& 									&\\
\hline
\hline
14	& 	5. Producción de SL				&& Etapa 4\\
15	& 									&\\
16	& 									&& Entrega y presentaciones\\ 
\hline
\end{tabular}



% \begin{tabular}{|r|c|c|c|c|c|c|c|c|}
% \hline
%\textsf{7} & fbox {algo} & & & & & & &\\ 
%\hline
%\textsf{7} & & & & & & & &\\ 
%\hline
%\end{tabular}

% subsection  (end)


%----------- M A T E R I A L ---------
\newpage
\part {Scripting Avanzado}
\section {Contenidos}

\begin{enumerate}
\item Comandos básicos de archivos ls, cd, mkdir, cp, mv, rm, ln, patrones de nombres
\item Redirección y piping, comandos head, tail, more, less, grep
\item Variables, ambiente, aritmética
\item Sentencias de control if, for, while, case
\item Funciones
\item Arreglos
\item Expresiones regulares, uso de grep
\item Uso de sort, diff, comm, uniq, cut
\item Uso de cron
\item Otros intérpretes: sed, awk, Perl
\end{enumerate}

\section{Ejercitación básica}

\subsection{Redirección y piping}
\begin{enumerate}
	\item Crear un archivo conteniendo la salida del comando ls
	\item Crear un archivo conteniendo la salida del comando ls -lR /tmp
	\item Obtener las cinco primeras líneas del archivo anterior
	\item Crear un archivo conteniendo las cinco primeras líneas y las cinco  últimas del archivo generado en 2
	\item Crear un archivo conteniendo las primeras cinco líneas de la salida del comando ls -lR /tmp
	\item Usando el anterior, crear un archivo conteniendo esas líneas, numeradas
	\item Crear un archivo conteniendo las últimas cinco líneas de la salida del comando ls -lR /tmp
\end{enumerate}


\subsection {Variables, ambiente}
\begin{enumerate}
	\item Asignar e imprimir el contenido de dos variables
	\item Asignar dos variables, imprimir sus valores, intercambiar sus valores, imprimirlos
	\item Crear un script que imprima un valor que será pasado como argumento
	\item Crear un script que imprima dos valores que serán pasados como argumento
	\item Crear un script que imprima todos los valores que le sean pasados como argumento
\end{enumerate}


\subsection{Sentencias de control}
\begin{enumerate}
	\item 
Imprimir cinco veces "Linux"
	\item 
Imprimir cinco veces el contenido de una variable
	\item 
Imprimir los números de 0 a 5
	\item 
Imprimir los dígitos de -1 a 6
	\item 
Imprimir los números de 0 a 99
	\item 
Imprimir junto al nombre de cada archivo en el directorio actual, su tamaño y su fecha de modificación
	\item 
Copiar los archivos terminados en .txt en archivos con igual nombre pero extensión .bak
	\item 
Renombrar los archivos con extensión .tex que comienzan en ASA reemplazando la partícula ASA con RII
	\item 
Para cada archivo modificado hace más de cinco días en un directorio, mostrar su cantidad de líneas
	\item 
Obtener mediante un cliente de HTTP una lista de archivos cuyos nombres están dados por  una expresión variable y controlada por un lazo
	\item 
De un conjunto de archivos tar, encontrar aquellas versiones de un archivo dado, contenido en ellos, que hayan sido modificadas entre dos fechas dadas.
\end{enumerate}


\subsection{Aritmética}
\begin{lstlisting}
$ declare -i num
$ num="hola"
$ echo $num
	0
$ num=5 + 5
	bash: +: command not found
$ num=5+5
$ echo $num
	10
$ num=4*6
$ echo $num
	24
$ num="4 * 6"
$ echo $num
	24
$ num=6.5
	bash: num: 6.5: syntax error in expression (remainder of expression is ".5")
$ i=5; j=$i+1; echo $j
$ i=5; let j=$i+1; echo $j
$ let i=5
$ let i=i+1
$ echo $i
	6
$ let "i = i + 2"
$ echo $i
	8
$ let "i+=1"
$ echo $i
	9
$ i=3
$ (( i+=4 ))
$ echo $i
	7
$ (( i=i-2 ))
$ echo $i
	5
$ let b=2#101; echo $b
$ let h=16#ABCD; echo $h
\end{lstlisting}

\subsection{Arreglos}
\begin{lstlisting}
$ A=(1 2 3 cuatro cinco)
$ echo ${!A[*]}
0 1 2 3 4
$ echo ${A[4]}
cinco
$ echo ${A[*]}
1 2 3 cuatro cinco
$ A[2]='banana'
$ echo ${A[*]}
1 2 banana cuatro cinco
\end{lstlisting}

\subsection{Arreglos asociativos}
\begin{lstlisting}
$ declare -A B
$ B=([francia]='paris' [espana]='madrid' [argentina]='buenos aires')
$ echo ${!B[*]}
espana argentina francia
$ echo ${B[*]}
madrid buenos aires paris
$ echo ${B[francia]}
paris
\end{lstlisting}


\subsection{Here-Documents}
\begin{lstlisting}
$ cat > texto.txt << END
> Hola
> Probando...
> END
$ cat texto.txt
\end{lstlisting}

\subsection{Traps}
\begin{lstlisting}
# man 7 signal
# 1 = SIGHUP (Hangup of controlling terminal or death of parent)
# 2 = SIGINT (Interrupted by the keyboard)
# 3 = SIGQUIT (Quit signal from keyboard)
# 6 = SIGABRT (Aborted by abort(3))
# 9 = SIGKILL (Sent a kill command)

trap limpieza 1 2 3 6 9
function limpieza
{
	echo "Recibimos senal - desmantelando..."
	rm -f ${tempfiles}
	echo Finalizando
}
\end{lstlisting}



\section{Casos de uso}


\subsection{Investigar el sistema}
\begin{enumerate}
	\item 
Modificar la salida del comando blkid para conocer el UUID, el nombre y tipo, y punto de montado, de cada dispositivo de bloques del sistema.
	\item 
Analizar archivos de log buscando conocimiento: duración de sesiones ssh por usuario, mensajes de mail entre usuarios, con histograma por tamaños, etc. (ver iptables.log, \ref{sec:iptables.log})
	\item 
Detectar momentos en que la salida de vmstat muestra picos de I/O, procesos corriendo, procesos en espera, uso de swap, etc.
\end{enumerate}


\subsection{Recuperar espacio de almacenamiento}
\begin{enumerate}
	\item Encontrar los diez archivos más grandes en un directorio y sus hijos, imprimirlos junto con su tamaño de mayor a menor.
	\item Encontrar los diez archivos más grandes en un directorio y sus hijos, moverlos a otro directorio (en otro filesystem).
	\item Encontrar los diez archivos más grandes del sistema, imprimir el nombre de usuario dueño.
	\item Agregar al script anterior el envío de notificación por mail al usuario responsable.
		\item 
Encontrar archivos en directorios de usuario con la cadena \quotes{cache} en su nombre e imprimir el uso de disco de cada uno.
	\item 
Idem, enviando nombres a un archivo y usándolo como lista para borrarlos, comprimirlos o moverlos.

\end{enumerate}



\subsection{Networking}
\begin{enumerate}
	\item 
Disparar un aviso cuando se pierde la conectividad a un conjunto dado de nodos de la red.
	\item 
Analizar la salida del comando netstat para descubrir en qué momento aparece un nuevo port abierto y a qué aplicación corresponde.
	\item 
Obtener un log de tráfico y obtener orígenes máximos y mínimos de tráfico, cantidades totales de bytes traficados por interfaz, etc.
	\item 
Recoger estadísticas de espacio en disco, cantidad de procesos, carga de CPU, en diferentes nodos de la red, y centralizarlos en un nodo monitor que presente los resultados.
\end{enumerate}

\subsection{Seguridad}
\begin{enumerate}
	\item 
Detener el script si la identidad del proceso corresponde a root.
	\item 
Solicitar información confidencial (como claves) con video inhibido.
	\item 
Capturar señales para impedir la interrupción del script por BREAK o fallos de ejecución.
	\item 
Utilizar MD5/SHAx para confirmar integridad de archivos.
\end{enumerate}




\subsection{Tratamiento de datos}
\begin{enumerate}
	\item 
Revisar el uso de los comandos cut, join, sort, uniq, comm.
	\item 
Crear script que administra una base de datos en formato CSV.
	\item 
Dado un archivo con una lista de direcciones IP, adjuntarles la resolución inversa de nombres correspondiente.
	\item 
Crear un histograma de accesos por nombre de dominio, a partir de los paquetes registrados en un archivo de log generado por iptables. 
	\item 
Dada una base de datos CSV implementar búsqueda por expresiones regulares.
	\item 
Dada una base de datos CSV implementar proyección sobre un conjunto de campos dados.
	\item 
Convertir un listado de individuos PDF en archivo CSV.
	\item 
Preparar un conjunto de scripts con un único punto de entrada para el administrador. Estos scripts mantendrán un conjunto de bases de datos en formato CSV:
\begin{lstlisting}
alumnos: UID, Username, Apellido, Nombres, NoLegajo, Activo
materias: MID, Nombre, Carrera, Docente
cursadas: UID, MID, Ano, Cuatrimestre
\end{lstlisting}
El dato Activo es booleano. Con estas bases de datos:
\begin{itemize}
	\item 
Listar todas las materias asignadas a un mismo docente.
	\item 
Listar todas las materias cursadas por un alumno.
	\item 
Listar todos los alumnos activos inscriptos en una materia.
	\item 
Listar todos los alumnos que cursan una misma carrera dada durante un año dado.
	\item 
Listar todos los alumnos, agrupados por materia cursada, dentro de cada año. 
	\item 
Listar todos los alumnos de un mismo docente.
	\item 
Dado un alumno por su legajo, consultar su estado Activo/Inactivo.
	\item 
Para aquellos alumnos que hace más de tres años que no se inscriben en ninguna cursada, pasar su dato Activo a falso (Inactivo).
	\item 
Generar un par de archivos en el formato de /etc/passwd y /etc/shadow para todos los alumnos activos.
	\item 
Generar un directorio /home/usuario para cada alumno activo, con UID correspondiente.
\end{itemize}
\end{enumerate}


\subsection{Accesibilidad para usuarios finales}
\begin{enumerate}
	\item 
Preparar un script con interfaz gráfica para copiar archivos seleccionados a una carpeta preestablecida con el fin de obtener un backup periódico de todos sus contenidos.
	\item 
Preparar un script con interfaz gráfica que presente los cinco directorios con mayor ocupación de almacenamiento dentro del home del usuario.
	\item 
Agregar interfaz gráfica a los scripts de administración de bases de datos de alumnos y materias.
\end{enumerate}


\newpage
\part {Configuraciones de Almacenamiento}

\section {Contenidos}

\begin{enumerate}
	\item Arquitectura de E/S, Dispositivos de E/S, Filesystems
	\item	Software RAID, instalación y mantenimiento, niveles 0, 1, 10
	\item	LVM, instalación y mantenimiento
	\item	Diseños típicos de almacenamiento
\end{enumerate}


\section{Dispositivos y filesystems}
Los dispositivos lógicos de bloques están asociados a algún medio de almacenamiento, real o virtual.  Ejemplos de dispositivos de bloques que encontramos con frecuencia son \lstinline$/dev/sda, /dev/sda1, /dev/dvd$, etc.

\figura[14]{IO}{I/O y dispositivos}{IO.jpg} 


\nonota{Presentan una interfaz que provee direccionamiento random o directo, es decir, sus bloques están numerados y se puede acceder a cualquier bloque con independencia de cuál haya sido accedido anteriormente (operación de \emph{seek}). Pueden directamente contener un filesystem u ofrecer soporte a otros dispositivos virtuales, que los agrupan (como los dispositivos RAID) o en general los utilizan (como los dispositivos snapshot de LVM). Los típicos dispositivos de bloques con los que nos encontramos son los discos y las particiones, pero es interesante conocer otros dispositivos que están soportados por volúmenes lógicos, archivos, u otros, remotos, que se acceden por medio de la red.}

\subsection{Temas de práctica}
\begin{enumerate}
	\item Crear y destruir particiones con fdisk, parted, gparted. 
	\item Reconocer tipos de particiones. Comprender la estructura de la tabla de particiones, particiones primarias, extendidas y lógicas.
	\item Comando dd, modificadores bs y count. Copia de dispositivos y archivos.
	\item Dispositivos /dev/null y /dev/zero. Creación de archivos prealojados. Modificador seek. 
	\item Comando mkfs. Tipo de filesystem. Filesystems sobre una partición, sobre un archivo.
	\item Loop devices. Comando losetup. Comando mount. Opciones ro, loop, offset. Montado de filesystems sobre una partición física, sobre un archivo, sobre una partición en una imagen de disco.
	\item Redimensionamiento de filesystems. Comando dd y modificador conv=notrunc. Comando resize2fs. Opciones relacionadas con filesystems en parted.
\end{enumerate}

\subsection{Loop devices}
\begin{lstlisting}
$ dd if=/dev/zero of=imagen.img bs=1024 count=1024
1024+0 records in
1024+0 records out
1048576 bytes (1.0 MB) copied, 0.00223564 s, 469 MB/s
$ ls -l imagen.img
-rw-r--r-- 1 root root 1048576 Sep  1 11:54 imagen.img
$ losetup /dev/loop0 imagen.img
$ losetup -a
/dev/loop0: [0808]:2260385 (/tmp/imagen.img)
$ mkfs -t ext3 /dev/loop0
mke2fs 1.42.8 (20-Jun-2013)

Filesystem too small for a journal
Discarding device blocks:          done                            
Filesystem label=
OS type: Linux
Block size=1024 (log=0)
Fragment size=1024 (log=0)
Stride=0 blocks, Stripe width=0 blocks
128 inodes, 1024 blocks
51 blocks (4.98%) reserved for the super user
First data block=1
Maximum filesystem blocks=1048576
1 block group
8192 blocks per group, 8192 fragments per group
128 inodes per group

Allocating group tables: 0/1	done                            
Writing inode tables: 0/1	done                            
Writing superblocks and filesystem accounting information: 0/1	done

$ mkdir mnt
$ mount -o loop /dev/loop0 mnt
$ df -h mnt
Filesystem      Size  Used Avail Use% Mounted on
/dev/loop0     1003K   17K  915K   2% /tmp/mnt
$ ls -l mnt
total 12
drwx------ 2 root root 12288 Sep  1 11:54 lost+found
$ ls / > mnt/lista.txt
$ ls -l mnt
total 13
-rw-r--r-- 1 root root   167 Sep  1 11:54 lista.txt
drwx------ 2 root root 12288 Sep  1 11:54 lost+found
$ df -h mnt
Filesystem      Size  Used Avail Use% Mounted on
/dev/loop0     1003K   18K  914K   2% /tmp/mnt
$ dd if=/dev/zero of=imagen.img bs=1024 count=1024 oflag=append conv=notrunc
1024+0 records in
1024+0 records out
1048576 bytes (1.0 MB) copied, 0.00206669 s, 507 MB/s
$ ls -l imagen.img
-rw-r--r-- 1 root root 2097152 Sep  1 11:54 imagen.img
$ losetup -c /dev/loop0
$ losetup -a
/dev/loop0: [0808]:2260385 (/tmp/imagen.img)
/dev/loop1: [0005]:5178 (/dev/loop0)
$ umount mnt
$ e2fsck -fp /dev/loop0
/dev/loop0: 12/128 files (0.0% non-contiguous), 39/1024 blocks
$ resize2fs /dev/loop0
resize2fs 1.42.8 (20-Jun-2013)
Resizing the filesystem on /dev/loop0 to 2048 (1k) blocks.
The filesystem on /dev/loop0 is now 2048 blocks long.

$ mount -o loop /dev/loop0 mnt
$ df -h mnt/
Filesystem      Size  Used Avail Use% Mounted on
/dev/loop0      2.0M   18K  1.9M   1% /tmp/mnt

\end{lstlisting}


\section{RAID}

Los \emph{arrays} RAID (Redundant Array of Independent Disks) son dispositivos virtuales creados como combinación de dos o más dispositivos físicos. El dispositivo virtual resultante puede contener un filesystem. 

Los diferentes modos de combinación de dispositivos, llamados niveles RAID, ofrecen diferentes características de redundancia y performance. Un array RAID con redundancia ofrece protección contra fallos de dispositivos. 

Los dispositivos Software RAID de Linux son creados y manejados por el driver \lstinline{md} (Multiple Device) y por eso suelen recibir nombres como \lstinline{md0}, \lstinline{md1}, etc.

 
\begin{itemize}
	\item Redundancia para tolerancia a fallos
	\item Mejoramiento de velocidad de acceso
	\item Hardware RAID, Fake RAID, Software RAID
	\item Niveles RAID
	\item RAID Devices
	\item Spare disks, faulty disks
\end{itemize}



\subsection {Niveles RAID}

\begin{description}
	\item [Linear mode] Dos o más dispositivos concatenados. La escritura de datos ocupa los dispositivos en el orden en que son declarados. 
Sin redundancia.
Mejora la performance cuando diferentes usuarios acceden a diferentes secciones del file system, soportadas en diferentes dispositivos.
	\item [RAID-0] Las operaciones son distribuidas (\emph{striped}) entre los dispositivos, alternando circularmente entre ellos. Cada dispositivo se accede en paralelo, mejorando el rendimiento. Sin redundancia. 
	\item [RAID-1]
Dos o más dispositivos replicados (\emph{mirrored}), con cero o más \emph{spares}. 
Con redundancia. Los dispositivos deben ser del mismo tamaño. Si existen \emph{spares}, en caso de falla o salida de servicio de un dispositivo, el sistema reconstruirá automáticamente una réplica de los datos sobre uno de ellos. 
En un RAID-1 de $N$ dispositivos, pueden fallar o quitarse hasta $N-1$ de ellos sin afectar la disponibilidad de los datos. 
Si $N$ es grande, el bus de I/O puede ser un cuello de botella (al contrario que en Hardware RAID-1). El scheduler de Software RAID en Linux asigna las lecturas a aquel dispositivo cuya cabeza lectora está más cerca de la posición buscada. 
	\item [RAID-4] No se usa frecuentemente. Usado sobre tres o más dispositivos. Mantiene información de paridad sobre un dispositivo, y escribe sobre los restantes en la misma forma que RAID-0. El tamaño del array será $(N-1)*S$, donde $S$ es el tamaño del dispositivo de menor capacidad en el array. 
Al fallar un dispositivo, los datos se reconstruirán automáticamente usando la información de paridad. El dispositivo que soporta la paridad se constituye en el cuello de botella del sistema. 


\item [RAID-5]
Utilizado sobre tres o más dispositivos con cero o más \emph{spares}. El tamaño del dispositivo RAID será $(N-1)*S$. La diferencia con RAID-4 es que la información de paridad se distribuye entre los dispositivos, eliminando el cuello de botella de RAID-4 y obteniendo mejor performance en lectura. Al fallar uno de los dispositivos, los datos siguen disponibles. Si existen \emph{spares}, el sistema reconstruirá automáticamente el dispositivo faltante. Si se pierden dos o más dispositivos simultáneamente, o durante una reconstrucción, los datos se pierden definitivamente. RAID-5 sobrevive a la falla de un dispositivo, pero no de dos o más. 
La performance en lectura y escritura mejora con respecto a un solo dispositivo. 

\item [RAID-6]
Usado sobre cuatro o más dispositivos con cero o más \emph{spares}. La diferencia con RAID-5 es que existen dos diferentes bloques de información de paridad, distribuidos entre los dispositivos participantes, mejorando la robustez. El tamaño del dispositivo RAID-6 es $(N-2)*S$. Si fallan uno o dos de los dispositivos, los datos siguen disponibles. Si existen \emph{spares}, el sistema reconstruirá automáticamente los dispositivos faltante. La performance en lectura es similar a RAID-5, pero la de escritura no es tan buena.

\item [RAID-10]
Combinación de RAID-1 y RAID-0 completamente ejecutada por el kernel, más eficiente que aplicar dos niveles de RAID independientemente. Es capaz de aumentar la eficiencia en lectura de acuerdo a la cantidad de dispositivos, en lugar de la cantidad de pares RAID-1, ofreciendo un 95\% del rendimiento de RAID-0 con la misma cantidad de dispositivos. Los \emph{spares} pueden ser compartidos entre todos los pares RAID-1.

\begin{comment}

\item [FAULTY]

Nivel especial de RAID que sirve para debugging del array por inyección de fallos de lectura y escritura. Sólo permite un dispositivo. Simula fallos a bajo nivel, permitiendo analizar comportamiento en caso de fallos de sector en lugar de fallos de discos.
\end {comment}

\end{description}

\subsection {Modos de operación}

\begin{description}
	\item [Create] 
	Creación de un array nuevo con superblocks por cada dispositivo.
	 \item [Assemble]
	Ensamblar dispositivos componentes previamente creados para conformar un dispositivo RAID activo. Los componentes pueden especificarse en el comando o ser identificados por scanning. 
	\item [Follow o Monitor]
Monitorizar un dispositivo RAID y actuar en caso de eventos interesantes. No se aplica a RAID niveles 0 ni linear, ya que sus componentes no poseen estados (\emph{failed}, \emph{spare}, \emph{missing}). 

	\item [Build]
	Construir un array sin superblocks por dispositivo (para expertos).
	\item [Grow]
	Extender o reducir un array, cambiando el tamaño de los componentes activos (RAID 1, 4, 5 y 6) o cambiando el número de dispositivos activos en RAID1.
	\item [Manage]
	Manipular componentes específicos de un array, como al agregar nuevos dispositivos \emph{spare} o al eliminar dispositivos \emph{faulty}.
	\item [Misc]
	Toda otra clase de operaciones sobre arrays activos o sus componentes.

\end{description}

\subsection{Construcción y uso de un array RAID-1}



Crear particiones en ambos discos, tipo fd (Linux RAID autodetect)
\begin{lstlisting}
fdisk /dev/sdb; fdisk /dev/sdc
\end{lstlisting}

Crear el array
\begin{lstlisting}
mdadm --create /dev/md0 --level=1 --raid-devices=2 /dev/sdb1 /dev/sdc1
watch cat /proc/mdstat
\end{lstlisting}

Usar el array
\begin{lstlisting}
mkfs -t ext3 /dev/md0
mkdir /datos
mount -t ext3 /dev/md0 /datos
cp /etc/hosts /datos
ll /datos
\end{lstlisting}

Examinar el array
\begin{lstlisting}
cat /proc/mdstat
cat /proc/partitions
mdadm --examine --brief --scan --config=partitions
mdadm --examine /dev/sdc
mdadm --query --detail /dev/md0
\end{lstlisting}

Crear script de alerta
\begin{lstlisting}
cat > /root/raidalert
#!/bin/bash
echo $(date) $* >> /root/alert
^D
chmod a+x /root/raidalert
\end{lstlisting}

Monitorear el arreglo con script de alerta
\begin{lstlisting}
mdadm --monitor -1 --scan --config=partitions --program=/root/raidalert
\end{lstlisting}

Crear configuración
\begin{lstlisting}
cat > /etc/mdadm.conf
DEVICE=/dev/sdb1 /dev/sdc1
ARRAY=/dev/md0 devices=/dev/sdb1,/dev/sdc1
PROGRAM=/root/raidalert
\end{lstlisting}

Establecer tarea periódica de monitoreo
\begin{lstlisting}
crontab -e
MAILTO=""
*/2 * * * * /sbin/mdadm --monitor -1 --scan 
\end{lstlisting}

Declarar un fallo
\begin{lstlisting}
mdadm /dev/md0 -f /dev/sdb1
cat /root/alert
\end{lstlisting}

Quitar un disco del array
\begin{lstlisting}
mdadm /dev/md0 -r /dev/sdb1 
cat /root/alert
\end{lstlisting}

Reincorporar el disco al array
\begin{lstlisting}
mdadm /dev/md0 -a /dev/sdb1 
cat /proc/mdstat
cat /root/alert
\end{lstlisting}

Destruir el array
\begin{lstlisting}
mdadm --stop /dev/md0
\end{lstlisting}

\subsection {Temas de práctica}
\begin{enumerate}
	\item ¿Qué marcas, modelos y precios de tarjetas controladoras RAID puede encontrar? ¿Con qué capacidades?
	\item ¿Qué diferencias hay entre Software RAID y Hardware RAID?
	\item ¿Qué niveles RAID ofrecen redundancia? ¿Contra qué eventos protege esta redundancia? ¿Contra qué eventos \emph{no} protege esta redundancia? 
	\item El uso de un dispositivo RAID, ¿es un reemplazo efectivo para las políticas y actividades de backup?
	\item ¿Cuáles niveles RAID ofrecen mejor velocidad de trabajo? ¿De qué factores depende la ventaja en performance de los diferentes niveles RAID entre sí y con respecto al uso de una única unidad de disco?
	\item ¿Cuál es la diferencia entre los niveles Linear RAID y RAID 0? ¿Qué clase de redundancia ofrece cada uno? ¿Contra qué eventos protege?
	\item Preparar un arreglo linear RAID sobre dos dispositivos loop. Observe qué relación tiene el espacio disponible en el dispositivo con los archivos que soportan los dispositivos loop.
	\item Preparar un arreglo linear RAID sobre dos discos extra agregados al equipo. 
	\item Preparar un arreglo RAID nivel 0 sobre dos discos extra agregados al equipo. 	
	\item ¿Puede medir la diferencia en performance entre los dispositivos de   los ejercicios anteriores, de linear RAID y de nivel 0? ¿Tiene sentido esta medición cuando el equipo es una máquina virtual? 
	\item Preparar un RAID nivel 1 sobre dos discos extra. Inyectar un fallo en uno de los discos. Agregar un nuevo disco e incorporarlo al RAID. Observar la sincronización del dispositivo.
	\item Como antes, preparar un RAID nivel 1 sobre dos discos extra, pero con una unidad \emph{spare}. Inyectar un fallo en uno de los discos y observar el comportamiento del dispositivo. 
	\item Retire el disco fallado y compruebe en qué estado queda el dispositivo.
	\item Vuelva a agregar el disco y compruebe en qué estado queda el dispositivo. 
	\item ¿Con qué comandos se investiga el estado de un dispositivo RAID? ¿Cómo se sabe cuándo un dispositivo RAID está activo o en modo degradado? ¿Cómo se sabe cuándo un dispositivo está fallado, activo, sincronizando?
	\item ¿Cuál es la forma de convertir en dispositivo RAID 1 un filesystem ya existente?
	\item ¿Cómo se puede adaptar el comportamiento de un RAID 1 a una situación con discos asimétricos (uno más rápido que el otro)?
	
\end{enumerate}



\section{Administración de LVM}
\subsection{Introducción a  LVM}
\label{sub:introLVM}
El soporte habitual para los file systems de servidores son los discos magnéticos, particionados según un cierto diseño definido al momento de la instalación del sistema. Las particiones se definen a nivel del hardware. El conjunto de aplicaciones y servicios del sistema utiliza los filesystems que se instalan sobre estas particiones. 

Las particiones de disco son un concepto de hardware, y dado que las unidades de almacenamiento se definen estáticamente al momento del particionamiento, presentan un problema de administración a la hora de modificar sus tamaños. 


El diseño del particionamiento se prepara para distribuir adecuadamente el espacio de almacenamiento entre los diferentes destinos a los que se dedicará el sistema. Sin embargo, es frecuente que el patrón de uso del sistema vaya cambiando, y el almacenamiento se vuelva insuficiente o quede distribuido en forma inadecuada. La solución a este problema implica normalmente el reparticionamiento de los discos, operación que obliga a desmontar los filesystems y a interrumpir el servicio. Para redimensionar una partición, normalmente es necesario el reboot del equipo, con la consiguiente interrupción del servicio en producción. 


La alternativa consiste en interponer una capa intermedia de software entre el hardware crudo, con sus particiones, y los filesystems sobre los que descansan los servicios. Esta capa intermedia está implementada por Logical Volume Manager (LVM). LVM es un subsistema orientado a flexibilizar la administración de almacenamiento, al interponer una capa de software que implementa dispositivos de bloques lógicos por encima de las particiones físicas. 

Usando LVM, el almacenamiento queda estructurado en capas, y las unidades lógicas pueden crearse, redimensionarse, o destruirse, sin necesidad de reboot, desmontar ni detener el funcionamiento del sistema. Con LVM pueden definirse por software contenedores de filesystems, de límites flexibles, que admiten el redimensionamiento \quotes{en caliente}, es decir sin salir de actividad, mejorando la disponibilidad general de los servicios.

Con LVM pueden agregarse unidades físicas mientras el hardware lo permita, extendiéndose dinámicamente las unidades lógicas y redistribuyendo el espacio disponible a conveniencia. Presenta también otras ventajas como la posibilidad de extraer \emph{snapshots} o instantáneas de un filesystem en funcionamiento (para obtener backups consistentes a nivel de filesystem), y manipular con precisión el mapeo a unidades físicas para aprovechar características del sistema (como \emph{striping} sobre diferentes discos).


% subsubsection  (end)




\subsection{Componentes de LVM}
\label{sub:compLVM}
En la terminología LVM, los dispositivos de bloques entregados al sistema LVM se llaman PV (physical volumes). Cualquier dispositivo de bloques escribible puede convertirse en un PV de LVM. Esto incluye particiones de discos y dispositivos múltiples como conjuntos RAID. Los PVs se agrupan en VGs (volume groups) que funcionan como \emph{pools} de almacenamiento físico. De cada pool pueden extraerse a discreción LVs (logical volumes), que se comportan nuevamente como dispositivos de bloques, y que pueden, por ejemplo, alojar filesystems. Estos serán los usuarios finales de la jerarquía (Fig. \ref{fig:jerLVM}).


\figura{jerLVM}{Jerarquía de componentes LVM}{LVM.png}

\begin{minipage}[c]{0.84\textwidth}
\begin{mdframed}
Conviene tener en mente la jerarquía de los siguientes elementos:
\begin{description}
	\item [Volumen físico o PV (physical volume)] Es un contenedor físico que ha sido agregado al sistema LVM. Puede ser una partición u otro dispositivo de bloques adecuado.
	\item [Grupo de volúmenes o VG (volume group)] Es un pool o repositorio de espacio conformado por uno o varios PVs. Un VG ofrece un espacio de almacenamiento virtualmente continuo, cuyo tamaño corresponde aproximadamente a la suma de los PVs que lo constituyen. Los límites entre los PVs que conforman un VG son transparentes.
	\item [Volumen lógico o LV (logical volume)] Es una zona de un VG que ha sido delimitada para ser usada por otro software, como por ejemplo un filesystem. Los tamaños de los LVs dentro de un VG no necesariamente coinciden con los de los PVs que los soportan.
\end{description}
\end{mdframed}
\end{minipage}

\subsection {Uso de LVM}
Los pasos lógicos para utilizar almacenamiento bajo LVM son: 
\begin{itemize}
	\item Crear uno o más PVs a partir de particiones u otros dispositivos.
	\item Reunir los PVs en un VG con lo cual sus límites virtualmente desaparecen.
	\item Particionar lógicamente el VG en uno o más LVs y utilizarlos como normalmente se usan las particiones.
\end{itemize}

El sistema LVM incluye comandos para realizar estas tareas y en general administrar todas estas unidades. Con ellos se puede, dinámicamente:
\begin{itemize}
	\item Redimensionar LVs de modo de ocupar más o menos espacio dentro del VG.
	\item Aumentar la capacidad de los VGs con nuevos PVs sin detener el sistema.
	\item Mover LVs a nuevos PVs, más rápidos, sin detener el sistema.
	\item Usar \emph{striping} entre PVs de un mismo VG para mejorar las prestaciones.
	\item Tomar una instantánea o \emph{snapshot} de un LV para hacer un backup del filesystem contenido en el LV.
	\item Tomar una instantánea como medida preventiva antes de una actualización o modificación.
\end{itemize}



Los comandos tienen nombres con los prefijos \lstinline$pv$, \lstinline$vg$, \lstinline$lv$, etc. Además, el comando \lstinline$lvm$ ofrece una consola donde se pueden dar esos comandos y pedir ayuda.

	


% subsubsection  (end)

\subsection{Redimensionamiento de volúmenes}
\label{sub:redimVol}
Una vez creado un LV, su capacidad puede ser reducida o aumentada (siempre que exista espacio extra en el VG que lo contiene). 
Si el LV redimensionado contuviera un filesystem, éste también debe ser redimensionado en forma acorde. 
\begin{itemize}
	\item Si un filesystem va a ser extendido, primero debe extenderse el LV que lo contiene. 
	\item Si un filesystem va a ser reducido, luego debe reducirse el LV que lo contiene. 
	\item Si un LV que va a ser reducido está ocupado en un porcentaje, la reducción del LV sólo puede llevarse a cabo en forma segura en dicho porcentaje. 

\end{itemize}
Los filesystems ext3 y ext4 cuentan con una herramienta, \lstinline$resize2fs$, que es capaz de redimensionarlos sin detener la operación.

\subsection{Snapshots y backups}
\label{sub:snapshots}

Un snapshot es un LV virtual, especialmente preparado, asociado a un LV original cuyo estado se necesita \quotes{congelar} para cualquier propósito de mantenimiento. Una vez creado el snapshot, mediante un mecanismo de \emph{copy-on-write (o COW)}, LVM provee una instantánea o vista inmutable del filesystem original, aunque éste se actualice. Una vez creado el LV virtual de snapshot, sus contenidos son estáticos y permanentemente iguales al LV original. Puede ser montado y usado como un filesystem corriente. El snapshot es temporario y una vez utilizado se descarta. 

La motivación principal del mecanismo de snapshots es la extracción de copias de respaldo. Durante la operación del sistema, las aplicaciones y el kernel leen y escriben sobre archivos, y por lo tanto el filesystem pasa por una sucesión de estados. Una operación de backup que se desarrolle concurrentemente con la actividad del filesystem no garantiza la consistencia de la imagen obtenida, ya que archivos diferentes pueden ser copiados en diferentes momentos, bajo diferentes estados del filesystem. Como consecuencia, la imagen grabada no necesariamente representa un estado concreto de la aplicación; y esto puede dar lugar a problemas al momento de la recuperación del backup. 

Hay muchos escenarios posibles de inconsistencia. Algunos ejemplos son:
\begin{itemize}
	\item Una aplicación que mantiene un archivo temporario en disco con modificaciones automáticas y periódicas (por ejemplo, \lstinline$vi$). 
	\item Aplicaciones que mantienen conjuntos de archivos fuertemente acoplados (bases de datos compuestas por tablas e índices en archivos separados).
	\item Una instalación de un paquete de software, que suele afectar muchos directorios del sistema. 

\end{itemize}

Una solución consiste en \quotes{congelar} de alguna forma el estado del filesystem durante la operación de copia (por ejemplo, desmontándolo). Con LVM, gracias al mecanismo de \emph{COW}, esta instantánea puede ser obtenida sin detener la operación del LV original, o sea sin afectar la disponibilidad del servicio. La operación con el filesystem del LV origen (LVO) no se interrumpe, ni modifica en nada la conducta de las aplicaciones que lo estén usando. 

%\begin{mdframed}
Para la creación de un snapshot de un LVO se necesita contar con espacio extra disponible dentro del mismo VG al cual pertenece el LVO. Este espacio extra no necesita ser del mismo tamaño que el LVO\footnote{No es fácil determinar con precisión este tamaño, ya que debe ser suficiente para contener todos los bloques modificados en el LVO durante el tiempo en que se use el snapshot; y este conjunto puede ser variable, dependiendo del patrón de uso del LVO y del medio hacia el cual se pretende hacer el backup (otro disco local, un servidor en la red local, un servidor en una red remota, una unidad de cinta, tienen diferente ancho de banda y diferente demora de grabación).}. Normalmente es suficiente un 15\% a 20\% del tamaño del LV original. Si el VG no tiene suficiente espacio, puede extenderse.
%\end{mdframed}

\subsubsection{Creación de snapshots}
\label{ssub:snapcreate}

% subsubsection  (end)
Crear un snapshot es preparar un nuevo LV, virtual, con un filesystem virtualmente propio, que se monta en un punto de montado diferente del original (Fig. \ref{fig:snap1}). 

\figura[8]{snap1}{Un LVO y su snapshot}{LVM-snapshot-1.jpg}
 
A partir de este momento se pueden hacer operaciones de lectura y escritura en ambos filesystems separadamente, con efectos distintos en cada caso. 

\subsubsection{Lectura y escritura del LVO}
\label{ssub:lvorw}
Los snapshots son creados tomando un espacio de bloques de datos (la tabla de excepciones) dentro del mismo VG del LVO. Mientras un bloque no sea modificado, las operaciones de lectura lo recuperarán del LVO. Pero, cada vez que se modifique un bloque del LVO, la versión original, sin modificar, de dicho bloque, será copiada en el snapshot (Fig. \ref{fig:snap2}). 

\figura[8]{snap2}{Escritura de un bloque del LVO}{LVM-snapshot-2.jpg}
 
\subsubsection{Lectura y escritura del snapshot}
\label{ssub:snaprw}
De esta manera, el snapshot muestra siempre los contenidos originales del LVO (Fig. \ref{fig:snap3}), salvo que se modifiquen por alguna operación de escritura en el snapshot.

\figura[8]{snap3}{Lectura de un bloque desde el snapshot}{LVM-snapshot-3.jpg}

Los LVs pueden declararse R/O o R/W. En LVM2, los snapshots son R/W por defecto.  Al escribir sobre un filesystem de un LV snapshot R/W, se grabará el bloque modificado en el espacio privado del snapshot sin afectar el LVO (Fig. \ref{fig:snap4}). Al eliminar el snapshot, todas las modificaciones hechas sobre el mismo desaparecen. 

\figura[8]{snap4}{Escritura sobre el snapshot}{LVM-snapshot-4.jpg}



\subsection {Creación de backups con snapshots}
\begin{enumerate}
	\item Se crea un snapshot del LV de interés.
	\item Se monta el snapshot en modo RO sobre un punto de montado conocido. 
	\item Se copian los archivos del snapshot con la técnica de backup que se desee. 
	\item Se verifica la integridad del backup.
	\item Si la verificación no fue satisfactoria, se repite el backup.
	\item En caso satisfactorio, el snapshot se destruye con \lstinline$lvremove <ID del snapshot>$.  

\end{enumerate}

\begin {comment}
--------------------------------------------
, en un momento en que el filesystem esté en estado consistente (por ejemplo, al arranque del sistema, mientras aún no ha sido montado, o mientras los servicios están administrativamente detenidos).
--------------------------------------------
\end{comment}

\subsection{Uso preventivo de snapshots}
El mecanismo de snapshots ofrece la posibilidad de recuperar el estado original del LVO al efectuar operaciones que pueden afectar críticamente la integridad u operatividad del sistema, como pruebas de instalación, etc. 


Como se ha visto, el conjunto de bloques almacenados en un snapshot está formado por bloques que, o bien provienen del LVO original, en su estado anterior a ser modificados (Fig. \ref{fig:snap2}), o bien, fueron modificados en el snapshot creado y montado en modo R/W (Fig. \ref{fig:snap4}). 

En ambos casos, este conjunto de bloques puede volver a ser aplicado sobre el LVO (o revertido) con la opción \lstinline$--merge$ del comando \lstinline$lvconvert$. El resultado, sin embargo, será sumamente diferente en uno y otro caso. 

Si se aplica el snapshot \emph{sin modificaciones}, el LVO vuelve al estado original al momento de ser tomado el snapshot. Si los bloques \emph{han sido modificados}, al revertir el snapshot el LVO cambia, asumiendo las modificaciones que se hayan hecho en el snapshot. Ambas propiedades pueden ser útiles para recuperar el estado después de una modificación que no ha sido satisfactoria, pero la forma de aplicar las operaciones es distinta.


Una técnica consiste en:

\begin{enumerate}
	\item Crear el snapshot del LVO, con espacio suficiente para registrar las modificaciones que se piensa hacer.
	\item Ejecutar sobre el LVO las operaciones críticas (instalar, actualizar, modificar).
	\item Verificar el resultado.
	\item Si el resultado fue satisfactorio, destruir el snapshot con \lstinline$lvremove <ID del snapshot>$.
	\item Si el resultado no fue satisfactorio, recuperar el estado original del LVO con el comando \lstinline$lvconvert --merge <ID del snapshot>$  
\end{enumerate}. Este comando restituye (\emph{roll-back}) al LVO todos sus bloques originales, que fueron grabados en el espacio de excepciones del snapshot, y luego destruye el snapshot.

Con la técnica anterior, las modificaciones se realizan sobre el filesystem del LVO y en caso necesario se revierten. Una segunda posibilidad es dejar el LVO fuera de línea y trabajar sobre el snapshot R/W. 

Para esto se crea el snapshot, se desmonta el LVO, se monta en su lugar el snapshot, se realizan las modificaciones, y se verifica el resultado. Si fue satisfactorio, se vuelcan las modificaciones al LVO con  \lstinline$lvconvert --merge <ID del snapshot>$. Si no, se elimina el snapshot. Finalmente, en uno u otro caso, se vuelve a montar el LVO en su lugar.

\subsection{Eliminación del snapshot}
\label{ssub:snapdel}

El snapshot debe ser destruido al finalizar el backup o terminar de usarlo, ya que, al obligar a  copiar  cada bloque del LVO que se modifica, representa un  costo en performance del sistema de I/O. Por lo demás, el snapshot se define con una cierta capacidad, que al ser excedida hace inutilizable el snapshot completo.

% subsubsection  (end)

\subsection{Ejemplos LVM}
\label{sub:ejemplosLVM}

Creación de Physical Volumes (PV)
\begin{lstlisting}
fdisk /dev/sdb
pvcreate /dev/sdb1 /dev/sdb2
pvdisplay
\end{lstlisting}

Creación de Volume Groups (VG)
\begin{lstlisting}
vgcreate vg0 /dev/sdb1 /dev/sdb2
vgdisplay
\end{lstlisting}

Creación de Logical Volumes (LV)
\begin{lstlisting}
lvcreate --size 512M vg0 -n lvol0
lvcreate -l 50%VG vg0 -n lvol1
lvcreate -l 50%FREE vg0 -n lvol2
lvdisplay vg0
\end{lstlisting}

Examinar LVM
\begin{lstlisting}
pvs
vgs
lvs
pvscan
vgscan
lvscan
\end{lstlisting}

Uso de volúmenes
\begin{lstlisting}
mkfs -t ext3 /dev/vg0/lvol0
mkdir volumen
mount /dev/vg0/lvol0 volumen
cp *.gz volumen
ls -l volumen
\end{lstlisting}

Extensión de un volumen
\begin{lstlisting}
umount /dev/vg0/lvol0
lvextend --size +1G vg0/lvol0
mount /dev/vg0/lvol0 volumen
resize2fs /dev/vg0/lvol0 
\end{lstlisting}

Agregar un disco al sistema
\begin{lstlisting}
fdisk /dev/hdd
pvcreate /dev/hdd1
vgextend vg0 /dev/hdd1
lvextend --size +1G vg0/lvol0 
ext2online /dev/vg0/lvol0 
\end{lstlisting}

Snapshot de un volumen
\begin{lstlisting}
lvcreate -s -n snap --size 100M vg0/lvol0
ls -l /dev/vg0
mkdir volumen-snap
mount /dev/vg0/snap volumen-snap
ls -l volumen-snap/
rm volumen/archivo1.tar.gz volumen/archivo2.tar.gz
ls -l volumen-snap/
\end{lstlisting}

Destruir un snapshot
\begin{lstlisting}
lvremove vg0/snap
\end{lstlisting}
% subsubsection Ejemplos LVM (end)

\subsection{Temas de práctica}
\begin{enumerate}
	\item Crear una partición, convertirla en PV, crear un VG y definir un LV \lstinline$lv0$ dentro del mismo dejando un 25\% del espacio libre. Crear un filesystem sobre el LV, montarlo y utilizarlo para administrar archivos.
	\item Definir un nuevo LV \lstinline$lv1$ en el mismo VG creado anteriormente, ocupando la totalidad del espacio del VG.
	\item Crear otra partición en el mismo u otro medio de almacenamiento, convertirla en PV y adjuntarla al VG del ejercicio anterior. Examinar el resultado de las operaciones con los comandos de revisión correspondientes. 
	\item Extender el LV \lstinline$lv1$ para ocupar nuevamente la totalidad del espacio del VG extendido. Crear un filesystem sobre el LV, montarlo y utilizarlo para administrar archivos.
	\item Modificar los tamaños de ambos LVs, extendiendo uno y reduciendo el otro. Recordar que al reducir un LV se debe primero reducir el filesystem alojado, y que para extender un filesystem se debe primero extender el LV que lo aloja. Comprobar que los filesystems alojados siguen siendo funcionales.
	\item Supongamos que, al querer crear un snapshot de un LV, el administrador recibe un mensaje de error diciendo que el VG no cuenta con espacio disponible. Sugiera un método para enfrentar este problema usando LVM.
	\item Dado un LV, ponga en práctica las técnicas de creación de snapshot para a) obtener un backup, y b) realizar modificaciones sobre el LV volviendo después al estado original.  
\end{enumerate}



\newpage
\part {Estrategias de Respaldo}

\section{Decisiones sobre los respaldos}
\begin{itemize}
	\item Propósitos y requerimientos legales, confidencialidad, etc.
	\item Archivos a respaldar 
	\begin{itemize}
		\item Archivos de sistema, de los usuarios, de las aplicaciones
		\item Vuelcos de bases de datos, formatos portables
	\end{itemize}	
	\item Dinámica de los archivos
	\begin{itemize}
		\item Estáticos o volátiles 
		\item Tasa de crecimiento 
	\end{itemize}
	\item Período de cobertura (ventana de seguridad)
	\item Esquema y política de respaldo 
	\begin{itemize}
		\item Quién es responsable
		\item Frecuencia y alcance de cada operación de backup
		\item Diarios, semanales
		\item Completos, diferenciales, incrementales
	\end{itemize}
	\item Almacenamiento del respaldo 
	\begin{itemize}
		\item Dónde se guarda
		\item Quién y cómo accede
		\item Recuperación de desastres, sitios remotos
	\end{itemize}
	\item Medios de respaldo (cinta, disco, DVD...)
	\begin{itemize}
		\item Costo en tiempo de respaldo
		\item Costo en tiempo de restauración
		\item Costo monetario por byte de almacenamiento 
	\end{itemize}
	\item Presupuesto en medios
	\item Herramienta y formato (ftp, sftp, scp, tar, cpio, rsync, dd...)  
	\item Sistema de apoyo (AMANDA, Bacula, BackupPC, Mondo...)
	\item Procedimiento de restauración
	\item Procedimiento de verificación
	\item Eliminación del respaldo
	\begin{itemize}
		\item Quién lo hace
		\item El medio se destruye o se reutiliza 
	\end{itemize}
	\item ¿Cómo escala la solución? ¿Se adapta al crecimiento de los datos?
\end{itemize}



\section{Sincronización de archivos}

\subsection{Herramienta rsync}

Rsync es un comando de sincronización de directorios y archivos. Puede usarse básicamente como un reemplazo de los comandos rcp o scp, pero presenta gran cantidad de opciones interesantes. Entre otras cosas, utiliza un algoritmo propio para computar las diferencias entre los archivos origen y destino antes de iniciar la copia, y transfiere únicamente las modificaciones. Es decir, para aquellos archivos que son diferentes entre origen y destino, únicamente copia aquellos bloques de datos que son efectivamente diferentes.  

Las opciones de rsync son numerosísimas (ver Anexo \ref{sec:rsync}). Algunas de las más utilizadas:


\begin{lstlisting}
rsync -av \                             # modo archive y modo verbose
  --compress \                          # comprimir al transferir
  --force \                             # borrar directorios aun si no vacios
  --delete \                            # borrar archivos no existentes
  --delete-excluded \                   # borrar tambien los excluidos
  --ignore-errors \                     # borrar aun bajo error
  --exclude-from=exclude_file \         # excluir los archivos listados
  --backup \                            # modo backup
  --backup-dir=`date +%Y-%m-%d` \       # directorio del modo backup
  origen destino
\end{lstlisting}

La opción -a funciona como sinónimo de un conjunto de otras opciones convenientes para las copias de respaldo en general. Esta opción equivale a -rlptgoD, que se traduce como r=recursivo; l=respetar los links simbólicos; p=preservar los permisos, t=los tiempos de acceso de los archivos, g=el grupo y o=el dueño; y D=recrear los pseudoarchivos de dispositivos en el lado destino. La compresión de archivos será conveniente cuando origen y destino se sitúen en equipos diferentes sobre la red.


Rsync puede usarse de muchas formas:
\begin{itemize}
	\item Copia de archivos locales, cuando ninguno de los elementos origen y destino contiene un separador \emph{dos puntos} (\quotes{:}).
	\item Copia entre host local y host remoto usando un shell remoto (por defecto, ssh) como transporte, cuando el destino contiene \quotes{:}.
	\item Copia desde un servidor rsync remoto al host local, cuando el origen contiene un separador \quotes{::} o es un URL que comienza en \quotes{rsync://}
	\item Copia entre el host local y un servidor rsync en el host remoto usando un shell como transporte, caso en que el origen contiene un separador “::” y se da la opción –rsh=COMMAND
\end{itemize}


\subsubsection{Especificación del origen}

La sintaxis de rsync tiene una particularidad: al especificar el directorio origen de la copia debe tenerse en cuenta que una barra al final (“/”) indica copiar solamente los contenidos del directorio origen, mientras que si no está presente la barra se entiende que el directorio también debe copiarse en el destino. 

\begin{lstlisting}
$ ll xmms
total 2052
-rw-rw-r--  1 oso oso 1973069 Aug  5 11:18 xmms-1.2.10-16.i386.rpm
-rw-rw-r--  1 oso oso   36388 Aug  5 11:18 xmms-cdread-0.14-6.a.i386.rpm
-rw-rw-r--  1 oso oso   79784 Aug  5 11:18 xmms-mp3-1.2.10-0.lvn.3.4.i386.rpm
\end{lstlisting}

El comando \lstinline$rsync -avz xmms/ xmms2$ copiará los tres archivos en un nuevo directorio llamado xmms2. En cambio, el comando \lstinline$rsync -avz xmms xmms2$ producirá lo siguiente.

\begin{lstlisting}
$ rsync -avz xmms xmms2
building file list ... done
created directory xmms2
xmms/
xmms/xmms-1.2.10-16.i386.rpm
xmms/xmms-cdread-0.14-6.a.i386.rpm
xmms/xmms-mp3-1.2.10-0.lvn.3.4.i386.rpm

sent 2068345 bytes  received 92 bytes  827374.80 bytes/sec
total size is 2089241  speedup is 1.01
$ ll xmms2
total 4
drwxrwxr-x  2 oso oso 4096 Aug  5 11:28 xmms
\end{lstlisting}

\subsubsection{Modo backup}
El modo backup de rsync hace que los archivos preexistentes en el destino sean renombrados cada vez que se reemplazan o se borran. La opción --backup-dir controla dónde serán creadas estas copias de backup con nuevos nombres. La opción --backup-suffix indica qué extensión tendrán estos archivos (por defecto se agrega un signo \quotes{\textasciitilde}).
En el caso anterior, supongamos que luego de hacer la copia, se modifican los archivos presentes en xmms.

\begin{lstlisting}
$ ll xmms2/xmms/
total 2052
-rw-rw-r--  1 oso oso 1973069 Aug  5 11:18 xmms-1.2.10-16.i386.rpm
-rw-rw-r--  1 oso oso   36388 Aug  5 11:18 xmms-cdread-0.14-6.a.i386.rpm
-rw-rw-r--  1 oso oso   79784 Aug  5 11:18 xmms-mp3-1.2.10-0.lvn.3.4.i386.rpm
$ touch xmms/*
$ rsync -avz --backup xmms xmms2
building file list ... done
xmms/xmms-1.2.10-16.i386.rpm
xmms/xmms-cdread-0.14-6.a.i386.rpm
xmms/xmms-mp3-1.2.10-0.lvn.3.4.i386.rpm

sent 2068339 bytes  received 86 bytes  827370.00 bytes/sec
total size is 2089241  speedup is 1.01
$ ll xmms2/xmms/
total 4104
-rw-rw-r--  1 oso oso 1973069 Aug  5 11:30 xmms-1.2.10-16.i386.rpm
-rw-rw-r--  1 oso oso 1973069 Aug  5 11:18 xmms-1.2.10-16.i386.rpm~
-rw-rw-r--  1 oso oso   36388 Aug  5 11:30 xmms-cdread-0.14-6.a.i386.rpm
-rw-rw-r--  1 oso oso   36388 Aug  5 11:18 xmms-cdread-0.14-6.a.i386.rpm~
-rw-rw-r--  1 oso oso   79784 Aug  5 11:30 xmms-mp3-1.2.10-0.lvn.3.4.i386.rpm
-rw-rw-r--  1 oso oso   79784 Aug  5 11:18 xmms-mp3-1.2.10-0.lvn.3.4.i386.rpm~
\end{lstlisting}

\section{Replicación de datos} 
Más allá de las diferentes alternativas para realizar la copia periódica de archivos, con una u otra herramienta, está el concepto de replicación de información. Su propósito es diferente del de la copia de resguardo. En lugar de proteger los datos, manteniendo una historia de copias a través del tiempo, la replicación busca aumentar la disponibilidad de los datos, manteniendo dos imágenes iguales en dos lugares de almacenamiento. Esta forma de tratamiento de la información no protege de pérdidas o errores, pero facilita el rápido regreso a la operación de un sistema ante fallas de hardware.
 
 
La replicación puede ser asimétrica (es decir, una copia es master y la segunda recibe las modificaciones), o simétrica (ambas copias reciben las modificaciones de la otra); y puede ser continua o manual. 

Herramientas que permiten ejecutar replicación, con diferentes alcances y objetivos, son el comando rsync, el utilitario unison (que utiliza el algoritmo de rsync), filesystems como GlusterFS, los sistemas de almacenamiento en nube con clientes automáticos, y el dispositivo de bloques replicado DRBD. Este último es frecuentemente un componente de los sistemas de Alta Disponibilidad. 


\subsection{Replicación con rsync}

El comando rsync siguiente tomará las acciones necesarias para obtener una réplica del directorio /datos en el servidor backupserver, directorio /backups. La opción -a copiará todos los archivos en todos los subdirectorios del directorio origen que no existan en el directorio destino, o que hayan sido modificados, preservando los permisos; e ignorará todos aquellos archivos que sean iguales en ambos directorios. Además, la opción --delete borrará los archivos en el directorio destino que no existan en el origen. 

\lstinline$rsync -aE --delete /datos backupserver.ejemplo.com.ar:/backups$

\subsubsection{Modo \emph{dry-run}}

Por supuesto, la opción –delete es peligrosa. Para verificar cuál será el efecto de un comando rsync antes de ejecutarlo, se puede utilizar la opción -n o su sinónimo –dry-run.

\subsection{Replicación de particiones}

Una extensión de la idea de backup es la de crear imágenes de particiones completas de un sistema, de modo de poder recuperarlo ante fallas generalizadas en menos tiempo y con menos trabajo que una reinstalación. Esta técnica es conveniente para aquellas particiones que alojan filesystems \quotes{de sistema}, es decir, donde los datos son mayormente binarios o archivos de configuración de sistema, que no son afectados por el trabajo cotidiano del usuario.

\subsection{Comando dd}
Una herramienta útil para el resguardo de particiones completas es el utilitario dd. Con él se puede obtener una imagen de una partición o dispositivo completo. 

\begin{lstlisting}
$ dd if=/dev/sda1 of=part1.dd; scp part1.dd serverbackup.ejemplo.com.ar:
$ dd if=/dev/fd0 of=diskette.img
\end{lstlisting}

Un uso alternativo es el resguardo de las tablas de particiones:

\lstinline$dd if=/dev/sda of=tpart.dd bs=1K count=1$

Si se necesitara recuperar un conjunto de archivos de una imagen de una partición o dispositivo, puede hacerse montando la imagen sobre un directorio vacío, como si fuera un dispositivo físico. La opción de mount que permite esto es loop.

\begin{lstlisting}
# mount -t vfat -o ro,loop /backups/partwin.img /mnt/rescate
# cp /mnt/rescate/* /datos
\end{lstlisting}


\subsection{Comando partimage}

La réplica de particiones con dd, sin embargo, carece de flexibilidad. Los backups son del mismo tamaño que la partición, y son difíciles de manipular. Un utilitario tal como partimage tiene conocimiento del filesystem, copia solamente los bloques en uso del dispositivo, y opcionalmente aplica compresión. Además, puede dividir el resguardo en múltiples archivos para facilitar el almacenamiento. Puede ser usado en volúmenes de los diferentes sistemas de archivos de Linux, tanto como en FAT o NTFS. Tiene un modo interactivo y uno de línea de comandos.

Tanto partimage como dd tienen limitaciones. No se puede recuperar un backup sobre una partición de tamaño menor que la original. En caso de ser sobre una de tamaño mayor, el espacio sobrante se desaprovecha, salvo que se redimensione el filesystem con una herramienta tal como resize2fs. No es sencillo recuperar selectivamente un conjunto de archivos de una imagen.

\subsection{Comando netcat}

El comando nc (o netcat) permite crear una conexión entre dos equipos a través de la red y utilizar entrada y salida standard para transferir información. Esto puede aprovecharse para la replicación de particiones de forma muy simple.

El siguiente ejemplo utiliza el comando dd para crear un flujo de bytes directamente de una partición del equipo origen, que se desea replicar. El flujo de bytes es emitido por la salida standard del comando dd y entubado a la entrada del comando nc que se conecta con un servidor nc en el host destino. En este host se recibe el flujo a través de la conexión y se entuba a la entrada standard de dd, que lo vuelca en la partición destino. El resultado es una transferencia \emph{raw} de la partición sin almacenamiento intermedio. En el ejemplo, el host origen tiene la dirección IP 10.0.0.1, y el destino, 10.0.0.2. El servidor nc en el destino atiende por el port 3000. La partición 5 del primer host se copia en la partición 3 del segundo, la que debe tener al menos el tamaño de la primera.

\begin{lstlisting}
# en el host origen
dd if=/dev/sda5 | nc 10.0.0.2 3000

# en el host destino
nc -l -p 3000 | dd of=/dev/sda3
\end{lstlisting}

\section{Temas de práctica}
\begin{enumerate}
	\item ¿Cómo determinar cuánto cambio hay en un filesystem? Diseñe un script que sea capaz de determinar qué archivos han sido modificados y cuánto ha crecido un filesystem entre dos fechas cualesquiera. Diseñe un script que sea capaz de presentar esta información en forma de tabla semanal. 
	\item Utilizando rsync, ejecute una transferencia a través de la red, de un archivo de tamaño considerable. Anote el tiempo que registra el programa. Verifique las opciones de compresión que está utilizando su comando.
	\item Repita la experiencia utilizando scp. Verifique las opciones de compresión de modo que la comparación sea justa. Puede usar el comando time para obtener el tiempo real de ejecución. Compare los tiempos. 
	\item Repita una vez más la transferencia utilizando rsync, sin eliminar el archivo ya copiado en su lugar de destino. Repita con scp y compare los tiempos.
	\item Prepare un directorio con uno o dos niveles de subdirectorios. Guarde archivos de tamaño considerable en esa estructura. Repita el experimento anterior con rsync y con scp, registrando tiempos. Luego modifique un único carácter de un único archivo de la jerarquía, repitiendo la transferencia con ambas herramientas y registrando tiempos.
	\item Cuando rsync utiliza ssh como transporte, se adapta a la política de autenticación de usuarios que utiliza ssh. ¿Cómo se puede evitar que rsync pida password de usuario para ejecutar una transferencia entre diferentes hosts?
	\item Prepare un script para replicar un directorio junto con sus subdirectorios, usando rsync, en otro host. Instale el script en crontab. Verifique su funcionamiento modificando los contenidos del directorio.
	\item Modifique el script anterior para utilizar el modo backup de rsync en lugar de replicar el directorio.
	\item ¿De qué forma, y en qué casos, puede ser de ayuda la herramienta \lstinline$anacron$ en lugar de \lstinline$cron$? 
\end{enumerate}

\newpage
\part {Virtualización}

\section{Formas de virtualización}
\begin{itemize}
	\item \emph{Hosts} o Anfitriones
	\item \emph{Guests} o Huéspedes
	\item Virtualización completa o Emulación
 	\begin{itemize}
		\item Reproducción, mediante software, de la conducta de todo el hardware. Es necesario escribir emuladores para cada dispositivo, lo cual es laborioso, costoso y generalmente difícil; pero presenta la ventaja de que la máquina completamente emulada permite correr cualquier sistema operativo sin modificaciones y sin que ese sistema operativo, ni las aplicaciones, perciban que están corriendo sobre hardware emulado. Esta forma de virtualización presenta la  mayor fidelidad y transparencia, pero performance limitada.
 	\end{itemize}
 	\item Virtualización asistida por hardware
	\begin{itemize}
		\item Apoyo provisto por el hardware para funciones de multiplexado de entrada/salida y manejo especial de memoria. Las generaciones recientes de procesadores (familia Core 2 de Intel en adelante) contienen la funcionalidad necesaria para asistir en la virtualización. Las ventajas consisten en buena performance y el hecho de poder correr en forma virtualizada cualquier sistema operativo sin necesidad de modificarlo. La desventaja es, por supuesto, que se necesita contar con la capacidad necesaria en el hardware.
 		\item VMWare, QEMU, KVM, Xen
	\end{itemize}
  	\item Paravirtualización
		\begin{itemize}
			\item Ejecución de un SO modificado
 			\item Reescritura de los drivers del sistema operativo host y de los guests. Los drivers se construyen según el modelo de \textit{split drivers}, o drivers divididos. El sistema operativo host es quien sigue actuando, con la parte inferior de los drivers, sobre los dispositivos físicos. Las máquinas virtuales, al efectuarse un requerimiento de entrada/salida, activan la mitad superior del driver y provocan un trap al monitor de virtualización, que administra los pedidos y los deriva a la mitad inferior del driver. 
Los nuevos drivers permiten multiplexar entrada/salida entre los dispositivos físicos y una cantidad de dispositivos virtuales asociados. La ventaja principal es la gran performance lograda con respecto a la emulación o virtualización completa. Para determinadas cargas de trabajo, especialmente para programas acotados por CPU, la eficiencia de un conjunto de máquinas paravirtualizadas es muy cercana al óptimo. La principal desventaja es que claramente se necesita modificar o instrumentar el código, tanto del sistema operativo host como de los guests. Sin embargo las aplicaciones siguen funcionando sin modificaciones.
			\item  Xen, UML, VirtualBox en modo software
		\end{itemize}
		\item Virtualización a nivel del SO
	\begin{itemize}
		\item Creación de múltiples espacios de usuario independientes en lugar de uno solo (llamados, según la tecnología específica, contenedores, \textit{virtual engines} o VEs, \textit{virtual private servers} o VPS, o \textit{jails}). Permiten al usuario y a las aplicaciones obtener la misma vista que si se tratara de un server real. Implementación avanzada del mecanismo de \textit{chroot}. El kernel ofrece características de administración de recursos para garantizar el uso equitativo de los mismos entre los contenedores. 
		\item VServer, OpenVZ
	\end{itemize}
\end{itemize}

\subsection {Dominios o anillos de protección}

\figura[10]{rings1}{Anillos de protección, normalmente 1 y 2 sin uso}{rings-1.pdf}
\figura[10]{rings2}{Hipervisor de tipo 1 y sistema guest}{rings-2.pdf}

\begin{itemize}
	\item Orden de privilegios impuesto por el hardware
		\begin{itemize}
		\item Kernel, anillo 0
		\item Drivers, anillos 1 y 2		
		\item Aplicaciones, procesos de usuario, anillo 3
	\end{itemize}
	\item Posibilita el mecanismo de los system calls
	\begin{itemize}
	\item El código que se ejecuta en un anillo más exterior no puede ejecutar instrucciones de los anillos más interiores ni acceder a memoria no asignada 
	\item El salto a los anillos más interiores se hace por puntos de acceso predefinidos (las llamadas al sistema)	
	\item Normalmente los kernels monolíticos corren en modo supervisor o kernel anillo 0, y las aplicaciones en modo usuario en el anillo 3 (Fig. \ref{fig:rings1})
	\end{itemize}
	\item Hardware de hipervisor (VTx, AMD-V)
	\begin{itemize}
		\item Característica de algunos procesadores desde 2005
		\item Crea un anillo modo 1 donde ejecutar el código privilegiado de las máquinas virtuales (Fig. \ref{fig:rings2})
	\end{itemize}
	\item Hipervisor o VMM
	\begin{itemize}
		\item Tipo 1
		\begin{itemize}
			\item Corre directamente sobre el hardware (Bare Metal)
			\item XenServer, VMware ESX/ESXi, Hyper-V, Oracle VM Server
		\end{itemize}		
		\item Tipo 2
		\begin{itemize}
			\item Corre sobre un sistema operativo dado
			\item VMware Workstation, VirtualBox, KVM
		\end{itemize}
	\end{itemize}
\end{itemize}

\section{Aplicaciones} 
\begin{itemize}
	\item Para el usuario final
	\begin{itemize}
		\item Revolución del multicore
			\begin{itemize}
				\item Muchos equipos en uno
			\end{itemize}
		\item Probar nuevas distribuciones u otro software
		\item Probar actualizaciones
	\end{itemize}
	\item Para el administrador de sistemas
	\begin{itemize}
		\item Independencia del hardware
		\item Sistemas \textit{legacy}
		\item \textit{Provisioning}
		\item \textit{Live Migration} y balance de carga
		\item Validación de backups, \textit{staging}
		\item Hosting de servicios
		\item Consolidación de servidores
		\begin{itemize}
			\item Menos espacio
			\item Menos consumo de energía
			\item Menos calor disipado
		\end{itemize}
		\item Aumentar la disponibilidad
	\end{itemize}
\end{itemize}


\section{Proxmox}

\begin{itemize}
	\item PVE - Proxmox Virtual Environment
	\item Infraestructura para administración de recursos de virtualización
	\item Virtualización 
	\begin{itemize}
		\item KVM
		\begin{itemize}
			\item Incorporado al kernel Linux
			\item Soporta múltiples SOs
			\item Almacenamiento en archivos o volúmenes
			\item Raw o QCOW2
			\item Compatible con otros SO pero costoso en CPU  
		\end{itemize}		
		\item Consolas de acceso a VMs
		\begin{itemize}
			\item SSH
			\item VNC
			\item QXL/Spice
			\item HTML5
		\end{itemize}
	\end{itemize}	
	\item Containers
	\begin{itemize}
		\item 	OpenVZ
		\begin{itemize}
		\item Mismo kernel que el host
		\item Cada contenedor tiene sus propios archivos, bibliotecas, aplicaciones, /proc, /sys, locks, usuarios, grupos, árbol de procesos, red, dispositivos, instancias de IPC
		\item Puede dárseles acceso a dispositivos reales
		\item Soporta únicamente Linux
		\item Almacenamiento en directorio del host	
		\item Mejor performance
		\item Turnkey Linux aporta \textit{appliances}
		\end{itemize}
	\end{itemize}
\end{itemize}



\figura{proxmox}{Cliente web de Proxmox}{pve01.png}

\subsection{Línea de comandos}


\tabla {pveadmtools}{Herramientas de administración Proxmox}{l|c}
{
Utilizar API & pvesh \\
Estudiar performance & pveperf \\
Administrar el cluster de PVEs & pvecm  \\
Administrar contenedores OpenVZ & vzctl \\
Administrar VMs & qm \\
Administrar templates & pveam \\
}

Comando qm
	\begin{lstlisting}
	qm start <vmID>			start vm
	qm stop <vmID>			kill vm (immediate stop)
	qm shutdown <vmID>		gracefully stop vm (send poweroff)
	qm reboot <vmID>		reboot vm (shutdown, start)
	qm reset <vmID>			reset vm (stop, start)
	qm suspend <vmID>		suspend vm
	qm resume <vmID>		resume vm
	qm destroy <vmID>		destroy vm (delete all files)
	qm startall			start all virtual machines
	qm stopall [timeout]		stop all virtual machines
	\end{lstlisting}

Script para agregar discos
	\begin{lstlisting}
	#!/bin/bash

	for ((m=501; m<=510; m+=1))
	do
		for d in 1 2; do qm set $m -sata$d volume=OSO-LABS:32; done
		qm config $m
	done
	\end{lstlisting}



		
\begin{lstlisting}
	root@pvetest:~# qm config 501
	balloon: 512
	bootdisk: ide0
	cores: 1
	description: TUASSL-asa010
	ide0: OSO-LABS:601/base-601-disk-1.qcow2/501/vm-501 disk-2.qcow2,format=qcow2,size=64G
	ide2: OSO-LABS:iso/CentOS-6.5-x86_64-minimal.iso,media=cdrom
	memory: 1024
	name: asa010
	net0: e1000=6E:AF:21:76:AA:0A,bridge=vmbr2
	ostype: l26
	sata1: OSO-LABS:501/vm-501-disk-3.raw,format=raw,size=32G
	sata2: OSO-LABS:501/vm-501-disk-4.raw,format=raw,size=32G
	sockets: 1
	vga: qxl	
\end{lstlisting}

\subsection{Otras características}
	\begin{itemize}
		\item Almacenamiento local o de red iSCSI, NFS, etc
		\item Importación/exportación a VMWare
		\item Cliente Web, login Linux/PVE
		\item Clustering
		\begin{itemize}
 			\item Simétrico, sin nodo central de administración 
			\item Cluster File System, importante para preservar la disponibilidad
	 	\end{itemize}		
 		\item Backup + Restore
		\begin{itemize}
		 	\item Scheduling, compresión, Live Snapshots
			\item Migración en vivo
		 \end{itemize}
		\item Administración de recursos
	\begin{itemize}
 		\item Pools de almacenamiento asignables a usuarios
 	\end{itemize}
\end{itemize}

\newpage 
\part {Alta Disponibilidad}
%

\section{Heartbeat}


\subsection{Conceptos}
El componente heartbeat tiene el rol de cluster manager. Es el software cuya misión es 1) determinar qué nodos pertenecen al cluster en cada momento y 2) actuar ante eventos de cambio de pertenencia al cluster. 

Para determinar la pertenencia al cluster, heartbeat establece un tráfico periódico de latidos (heartbeats) en ambos sentidos. Mientras cada nodo escuche los latidos del otro, el cluster permanecerá en estado completo. 

Heartbeat controla la adquisición y entrega de los recursos servidos por el cluster en función de la pertenencia de los nodos. Los eventos posibles que afectan el estado de estos recursos son la caída de un nodo, o bien el regreso de un nodo al cluster. Las acciones ante cada evento dependen de la configuración de los servicios que serán alojados en cada nodo, y consistirán en asumir o entregar recursos en un orden preestablecido. 

Al caer un nodo, el sobreviviente dejará de escuchar latidos por un intervalo de tiempo mayor que un umbral dado, y así declarará muerto al nodo peer, luego de lo cual asumirá la prestación de determinados servicios. El cluster pasará a modo degradado, lo que significa que seguirá prestando servicios, pero sin redundancia, hasta la recuperación del nodo afectado.

Si un nodo quedara sin comunicación con el resto de la red, podría llegar erróneamente a la conclusión de que el nodo peer ha caído. Para separar este falso positivo de falla, ambos nodos comprueban permanentemente que son capaces de alcanzar al menos uno de un conjunto de nodos, confiables, que se asume estarán siempre en línea.

Si ambos nodos del cluster perdieran contacto entre sí pero no con el resto de la red, ambos se considerarían a sí mismos únicos miembros vivos del cluster. Esta condición se llama \emph{split brain}, o cerebro dividido. 

El cluster no es capaz de recuperarse automáticamente de esta condición. Una vez que se ha perdido la definición de quién es el nodo primario, solamente la intervención del administrador puede decidir el conflicto. Para evitar la situación de cerebro dividido, se aplica redundancia en las vías de comunicación de heartbeat (normalmente, más de una placa de red y algún canal secundario separado del subsistema de red, tal como una línea serial).

\subsection{Configuración}
Heartbeat se configura con tres archivos de control, situados dentro de /etc/ha.d en la instalación default, y que son ha.cf (de configuración general), haresources (de recursos) y authkeys (de autenticación). 
La configuración de heartbeat es simétrica, es decir, los tres archivos deben ser idénticos en los dos nodos. El hostname de cada nodo se utiliza para asumir la pertenencia al cluster en la directiva node del archivo ha.cf y para ligar el nodo a su rol en haresources. Lo que identifica el rol del nodo es únicamente su hostname. 

\subsubsection{Archivo de configuración general}
El archivo /etc/ha.d/ha.cf establece algunos parámetros de configuración general de heartbeat. A continuación se muestra un ejemplo de configuración suficiente establecida en ha.cf.


\tabla{hacf}{Configuración suficiente para ha.cf}{l|l}{
\lstinline$logfacility daemon$ & Se usará para logging el facility daemon \\
\lstinline$node nodo1 nodo2$ & Nodos que participan del cluster \\
\lstinline$keepalive 2$ & Intervalo en segundos entre latidos \\
\lstinline$deadtime 10$ & Declarar nodo muerto luego de n segundos \\
\lstinline$bcast eth0 eth1$ & Por dónde emitir latidos en broadcast \\
\lstinline$ping_group 172.16.20.1$ & Pseudomiembros del cluster \\
& para comprobar que la red sigue viva \\
\lstinline$auto_failback yes$ &  Intentar mantener los recursos \\
& en su responsable nominal \\
\lstinline$respawn hacluster /usr/lib/heartbeat/ipfail$ & Failover en caso de falla de red\\
}


\section{Parámetros importantes}

\begin{description}
	\item [node] Define los nodos que integrarán el cluster. En particular es importante que esté correctamente consignado el nombre de cada nodo tal como es entregado por el comando uname -n.
	\item [auto\_failback] La característica de auto\_failback yes es opcional, e implica que luego de un incidente ocurrido al primario de un servicio, y al recuperarse éste, el nodo secundario devolverá el recurso al responsable nominal.
	\item [keepalive] El tiempo de keepalive es el intervalo entre latidos. Debe tenerse en cuenta que, si se propaga el latido por la red de servicio, un tiempo de keepalive muy bajo significará tráfico espurio en la red. 
	\item [deadtime]Este parámetro define el tiempo luego del cual se declara muerto a un nodo que no ha respondido un latido. Un deadtime muy bajo, si bien permite una recuperación del servicio más rápida, implica mayor peligro de ingresar en la condición de cerebro dividido, lo que puede ocasionar graves problemas (como el acceso simultáneo al almacenamiento) y requiere intervención humana para ser corregida. El deadtime debe afinarse en función de la capacidad de respuesta del nodo que debe contestar el latido. Si los nodos están sujetos a variaciones importantes en la carga de trabajo, es preferible adoptar una posición conservadora y evitar los problemas de cerebro dividido a costa de un mayor tiempo de failover.	 
	\item [bcast] Define las interfaces que emitirán heartbeats. Puede utilizarse la red privada, que es confiable, para propagar el latido. Es preferible no utilizar la red de servicio. 
	\item [serial, baud] Para obtener redundancia se recomienda utilizar además un enlace serial null-modem. Se agrega una vía de heartbeat alternativa con la directiva serial ttyS0.	
	\item[ping, ping\_group] Con estas directivas se identifican un host de control o un grupo de control de hosts que se supone están siempre activos en la red y pueden contestar pings. Este grupo de control sirve para validar el estado de la red de cada nodo y determinar que una pérdida de latidos realmente se trata de un miembro que no contesta.
	\item [respawn] Esta directiva establece procesos que serán monitoreados por heartbeat y reiniciados en caso de fallo. En el caso normal, el programa que se desea monitorear es ipfail, que vigila si la conectividad del nodo sufre algún inconveniente y en caso necesario entrega los recursos al otro miembro. En la línea de la directiva respawn, el primer argumento es la identidad del usuario con la cual correrá el proceso monitoreado.
\end{description}



\subsubsection{Archivo de recursos}
El archivo /etc/ha.d/haresources define los recursos que serán propios de cada nodo. Junto al nombre de cada nodo se especifican los recursos que adquirirá ese nodo mientras el cluster opere en modo normal. En modo degradado, el nodo superviviente adquirirá los recursos que sean propios del otro.
Los recursos se adquieren de izquierda a derecha como están enumerados en haresources y se liberan de derecha a izquierda cuando el nodo sale del rol primario. 
A continuación se muestra un ejemplo de este archivo para un cluster activo-activo.

\begin{lstlisting}
nodo1	drbddisk::mail \
		Filesystem::/dev/drbd0::/ha/mail::ext3 \
		postfix \
		172.16.20.101 
nodo2	drbddisk::web \
		Filesystem::/dev/drbd1::/ha/web::ext3 \
		httpd \
		172.16.20.102
\end{lstlisting}

Aquí se especifican:
Los recursos DRBD definidos en el archivo de configuración /etc/drbd.conf, con sus nombres tal como figuran en dicho archivo.
Los filesystems ubicados sobre los recursos DRBD, con su dispositivo, punto de montado y tipo.
Los servicios, dependientes de /etc/rc.d/init.d o /etc/init.d, que utilizarán esos filesystems.
Las direcciones de los dos servicios principales del cluster, que serán asumidas por interfaces secundarias, creadas dinámicamente por heartbeat.
Archivo de autenticación
El archivo /etc/ha.d/authkeys permite la autenticación de un nodo frente al otro. Puede ser generado con un script tal como el siguiente.

\begin{lstlisting}
DATE=`date`
cat <<-!AUTH >/etc/ha.d/authkeys
  # Automatically generated authkeys file $DATE
  auth 1
  1 sha1 `dd if=/dev/urandom count=4 2>/dev/null | md5sum | cut -c1-32`
!AUTH
chown root:root /etc/ha.d/authkeys
chmod 0600 /etc/ha.d/authkeys
\end{lstlisting}


\subsubsection{Configuración de los servicios}
Los servicios del cluster deben ser configurados para utilizar el almacenamiento replicado provisto por DRBD. Por ejemplo, si el servicio fuera el server HTTP Apache, se deberá indicar que la raíz de los documentos servidos por Apache se ubica en algún lugar del filesystem replicado. Opcionalmente, cuando lo permitan las aplicaciones, otros archivos (como los de configuración u otros auxiliares de runtime) pueden quedar soportados dentro de este filesystem, y sus ubicaciones originales reemplazadas por links simbólicos a éstos.
Estos servicios deben depender de un script de control como los normalmente proporcionados por el sistema\footnote{Habitualmente localizados en /etc/rc.d/init.d (RedHat) o bien en /etc/init.d (Debian)}. El proceso heartbeat buscará los scripts de inicialización al arranque del sistema, y manejará el arranque y detención de esos servicios.


\subsubsection{Entrega de los servicios a heartbeat}
Heartbeat requiere el control del arranque y parada de los servicios que componen los grupos de recursos. Se debe controlar que los servicios del cluster no sean lanzados automáticamente a la inicialización del sistema. 
Deshabilitar el lanzamiento de los servicios3 que deben quedar bajo control de heartbeat, a saber en el ejemplo anterior, httpd y postfix.
La deshabilitación debe hacerse para todos los runlevels bajo los cuales se piense hacer correr al sistema (típicamente 3 y 5).
Habilitar el arranque de heartbeat al inicio.
Monitoreo de heartbeat
La salida de logging de heartbeat depende de la configuración. Cuando la salida se dirige al log del sistema host, el comando tail -f /var/log/messages | grep heartbeat muestra la actividad en forma continua. Otros archivos de logging pueden ser /var/log/ha-log, /var/log/ha-debug.

\subsection{Testing del cluster}
Es imprescindible llevar a cabo algunos tests para verificar la corrección de la configuración [TLEC].


\begin{enumerate}
	\item Desconectar la alimentación del servidor primario.
En el servidor secundario, heartbeat debe detectar la pérdida de latidos e iniciar un failover. Deberán ser iniciados los scripts de recursos correspondientes. El secundario debe enviar broadcasts ARP gratuitos para notificar al resto de la red que la dirección MAC correspondiente al IP de servicio ha cambiado.
	\item Comprobar el efecto del comando hb\_standby.
Verificar que el comando hb\_standby en el primario fuerza la migración de los recursos al secundario. Nuevamente verificar que el mismo comando, dado en el secundario, devuelve los servicios al primario.
	\item Desconectar la conexión a la red de producción del primario.
El servicio ipfail debe detectar esta condición y los recursos deben recaer sobre el secundario.
	\item Desconectar uno de los caminos de heartbeat entre los dos servidores.
Debe existir dos o más caminos de heartbeat entre los servidores para evitar los falsos positivos. Al eliminar uno de estos caminos, la operación no debe sufrir ningún cambio.
	\item Desconectar todos los caminos de heartbeat entre los dos servidores.
Si se está usando algún dispositivo de Stonith, el secundario debe suponer que el primario ha salido de servicio, iniciar un evento de Stonith, y asumir los recursos. Lo que siga dependerá de cómo se ha configurado Stonith y de si se está usando o no la opción auto\_failback. Con Stonith y auto\_failback, ambos servidores comenzarán cíclicamente a sacarse de servicio mutuamente. Para evitarlo, deshabilitar auto\_failback.
	\item Matar el daemon heartbeat en el primario.
El secundario debe ejecutar Stonith sobre el primario antes de asumir los servicios para evitar un escenario de split-brain.
	\item Matar los daemons de servicios en el server primario.
Si se están usando cl\_status y/o cl\_respawn, o la aplicación de monitoreo Mon, el cluster debe tomar la acción correspondiente (reiniciar los servicios, alertar al administrador).
	\item Reiniciar ambos servidores.
Los servidores deben arrancar correctamente y dejar los servicios activos en su primario. Esta acción de tuning puede sugerir un cambio en la configuración del tiempo initdead si el secundario trata de capturar los servicios antes de que el primario termine de bootear. 
\end{enumerate}




%----------- A N E X O S ---------
\newpage
\part {Anexos}
\appendix
\section{iptables.log}
\label{sec:iptables.log}
\begin{lstlisting}
 Logged 539 packets on interface eth1
   From 0000:0000:1011:1213:0100:0000:0000:0000 - 3 packets to icmpv6(130)
   From 0000:0000:0000:859e:0100:0000:0000:0000 - 1 packet to icmpv6(130)
   From 0000:0023:ff53:4d42:0100:0000:0000:0000 - 1 packet to icmpv6(130)
   From 0000:0000:0000:3433:0100:0000:0000:0000 - 1 packet to icmpv6(130)
   From 0000:0000:0000:3132:0100:0000:0000:0000 - 1 packet to icmpv6(130)
   From 0000:0000:0000:7d3a:0100:0000:0000:0000 - 1 packet to icmpv6(130)
   From 0000:0000:0000:6e63:0100:0000:0000:0000 - 1 packet to icmpv6(130)
   From 0000:0000:0000:937f:0100:0000:0000:0000 - 1 packet to icmpv6(130)
   From 0000:0000:0000:0000:0000:0000:0000:0000 - 2 packets to icmpv6(130)
   From 0000:0000:0000:0000:0100:0000:0000:0000 - 40 packets to icmpv6(130)
   From 0000:0000:0000:6569:0100:0000:0000:0000 - 1 packet to icmpv6(130)
   From 000e:175f:531c:580e:0100:0000:0000:0000 - 1 packet to icmpv6(130)
   From 0011:11db:a2d4:0a00:0100:0000:0000:0000 - 2 packets to icmpv6(130)
   From 002c:799a:0694:8c26:0100:0000:0000:0000 - 1 packet to icmpv6(130)
   From 0100:0000:0600:0000:0100:0000:0000:0000 - 2 packets to icmpv6(130)
   From 0101:080a:00c6:0621:0100:0000:0000:0000 - 1 packet to icmpv6(130)
   From 0101:080a:00c6:f565:0007:994d:0000:0000 - 1 packet to icmpv6(130)
   From 0101:080a:00c7:39ad:0100:0000:0000:0000 - 1 packet to icmpv6(130)
   From 0101:080a:00c7:3e49:0100:0000:0000:0000 - 1 packet to icmpv6(130)
   From 0101:080a:00c7:5bd1:0100:0000:0000:0000 - 1 packet to icmpv6(130)
   From 0101:080a:1551:7183:0100:0000:0000:0000 - 1 packet to icmpv6(130)
   From 0101:080a:1a9a:1543:0100:0000:0000:0000 - 1 packet to icmpv6(130)
   From 0101:080a:4553:94db:0100:0000:0000:0000 - 1 packet to icmpv6(130)
   From 0101:080a:4557:126c:0100:0000:0000:0000 - 1 packet to icmpv6(130)
   From 0101:080a:4559:6ffd:0100:0000:0000:0000 - 1 packet to icmpv6(130)
   From 0101:080a:4559:e9ac:0100:0000:0000:0000 - 1 packet to icmpv6(130)
   From 0101:080a:455a:3fd8:0100:0000:0000:0000 - 1 packet to icmpv6(130)
   From 0101:080a:455a:cc78:0100:0000:0000:0000 - 1 packet to icmpv6(130)
   From 0101:080a:455c:c658:0100:0000:0000:0000 - 1 packet to icmpv6(130)
   From 0101:080a:455d:4147:0100:0000:0000:0000 - 1 packet to icmpv6(130)
   From 0101:080a:455d:bbfc:0100:0000:0000:0000 - 1 packet to icmpv6(130)
   From 0101:080a:455e:3351:0100:0000:0000:0000 - 1 packet to icmpv6(130)
   From 0101:080a:4567:104c:0100:0000:0000:0000 - 1 packet to icmpv6(130)
   From 0101:080a:4575:8fa3:0100:0000:0000:0000 - 1 packet to icmpv6(130)
   From 0101:080a:4575:fcd3:0100:0000:0000:0000 - 1 packet to icmpv6(130)
   From 0101:080a:4584:d388:0100:0000:0000:0000 - 1 packet to icmpv6(130)
   From 0101:080a:458c:f368:0100:0000:0000:0000 - 1 packet to icmpv6(130)
   From 0101:080a:4594:8809:0100:0000:0000:0000 - 1 packet to icmpv6(130)
   From 0102:0417:0000:0000:0100:0000:0000:0000 - 1 packet to icmpv6(130)
   From 0102:a16d:0a00:00c8:0100:0000:0000:0000 - 1 packet to icmpv6(130)
   From 0204:05b4:0402:080a:0100:0000:0000:0000 - 2 packets to icmpv6(130)
   From 0300:0000:0400:0000:0100:0000:0000:0000 - 3 packets to icmpv6(130)
   From 036b:696d:0675:6e63:0100:0000:0000:0000 - 1 packet to icmpv6(130)
   From 10d4:d5cb:f80c:14d5:0100:0000:0000:0000 - 1 packet to icmpv6(130)
   From 1400:0300:9709:0000:0100:0000:0000:0000 - 1 packet to icmpv6(130)
   From 2269:6422:3a20:2232:0100:0000:0000:0000 - 1 packet to icmpv6(130)
   From 3037:3337:3431:3832:0100:0000:0000:0000 - 1 packet to icmpv6(130)
   From 3135:3332:3a38:3439:0100:0000:0000:0000 - 1 packet to icmpv6(130)
   From 3234:3238:323a:3834:0100:0000:0000:0000 - 1 packet to icmpv6(130)
   From 3333:3238:323a:3834:0100:0000:0000:0000 - 1 packet to icmpv6(130)
   From 3335:3839:323a:3834:0100:0000:0000:0000 - 1 packet to icmpv6(130)
   From 3336:3532:323a:3834:0100:0000:0000:0000 - 1 packet to icmpv6(130)
   From 3336:3837:3039:3132:0100:0000:0000:0000 - 8 packets to icmpv6(130)
   From 3430:3734:3330:3532:0100:0000:0000:0000 - 1 packet to icmpv6(130)
   From 3432:3230:3239:3a33:0100:0000:0000:0000 - 1 packet to icmpv6(130)
   From 3432:3635:3239:3a33:0100:0000:0000:0000 - 1 packet to icmpv6(130)
   From 3433:3230:3332:3a38:0100:0000:0000:0000 - 1 packet to icmpv6(130)
   From 3433:3534:3039:3a33:0100:0000:0000:0000 - 1 packet to icmpv6(130)
   From 3734:3237:3339:313a:0100:0000:0000:0000 - 1 packet to icmpv6(130)
   From 3734:3330:3238:313a:0100:0000:0000:0000 - 1 packet to icmpv6(130)
   From 3734:3330:3636:313a:0100:0000:0000:0000 - 1 packet to icmpv6(130)
   From 3734:3330:3930:323a:0100:0000:0000:0000 - 1 packet to icmpv6(130)
   From 3734:3334:3533:313a:0100:0000:0000:0000 - 1 packet to icmpv6(130)
   From 3734:3334:3736:393a:0100:0000:0000:0000 - 1 packet to icmpv6(130)
   From 616a:6f72:223a:2031:0100:0000:0000:0000 - 1 packet to icmpv6(130)
   From 6d65:6e74:7322:3a7b:0100:0000:0000:0000 - 9 packets to icmpv6(130)
   From 6f20:3134:3037:3433:0100:0000:0000:0000 - 2 packets to icmpv6(130)
   From 7473:223a:7b7d:7d00:0100:0000:0000:0000 - 1 packet to icmpv6(130)
   From 7473:223a:7b7d:7d32:0100:0000:0000:0000 - 1 packet to icmpv6(130)
   From 7473:223a:7b7d:7d3a:0100:0000:0000:0000 - 4 packets to icmpv6(130)
   From 7473:223a:7b7d:7d7b:0100:0000:0000:0000 - 3 packets to icmpv6(130)
   From 756e:6e69:6e67:223a:0100:0000:0000:0000 - 1 packet to icmpv6(130)
   From bf1d:f661:984e:fcb0:0100:0000:0000:0000 - 1 packet to icmpv6(130)
   From c011:fdda:43fe:4d59:65fe:e1aa:c7d5:683a - 1 packet to icmpv6(130)
   From c012:aa5c:173c:261c:0100:0000:0000:0000 - 1 packet to icmpv6(130)
   From c012:cfa3:e641:3b5f:0100:0000:0000:0000 - 1 packet to icmpv6(130)
   From c013:4b15:1c15:3311:0100:0000:0000:0000 - 1 packet to icmpv6(130)
   From c013:9389:bcf8:f7d4:0100:0000:0000:0000 - 1 packet to icmpv6(130)
   From c014:ff7d:0000:0000:0100:0000:0000:0000 - 1 packet to icmpv6(130)
   From e063:837f:0000:0000:0100:0000:0000:0000 - 3 packets to icmpv6(130)
   From e063:837f:2077:937f:0100:0000:0000:0000 - 1 packet to icmpv6(130)
   From e063:837f:301f:937f:0100:0000:0000:0000 - 3 packets to icmpv6(130)
   From 0.0.0.0 - 154 packets to igmp(0)
   From 10.0.3.4 - 154 packets to igmp(0)
   From 10.0.3.21 - 75 packets to igmp(0)
   From 10.0.3.209 - 2 packets to igmp(0)

 Listed by source hosts:
 Logged 18 packets on interface virbr0
   From fe80:0000:0000:0000:5054:00ff:feed:8246 - 18 packets to udp(5353)

 Listed by source hosts:
 Logged 50 packets on interface wlan0
   From 10.0.4.1 - 2 packets to udp(68)
   From 64.233.186.188 - 15 packets to tcp(44421,53418)
   From 64.235.151.8 - 5 packets to tcp(56214)
   From 173.194.42.0 - 1 packet to tcp(56677)
   From 173.194.42.21 - 3 packets to tcp(58597)
   From 173.194.42.22 - 7 packets to tcp(35536)
   From 173.194.42.75 - 3 packets to tcp(48585)
   From 173.194.42.85 - 4 packets to tcp(44550)
   From 173.194.42.86 - 1 packet to tcp(51940)
   From 192.168.1.1 - 2 packets to udp(68)
   From 192.168.2.1 - 2 packets to udp(68)
   From 195.135.221.134 - 1 packet to tcp(51756)
   From 195.154.174.66 - 1 packet to tcp(58047)
   From 200.42.136.212 - 3 packets to tcp(59351,59361)
\end{lstlisting}

\section{Opciones de rsync}
\label{sec:rsync}
\begin{lstlisting}
$ rsync --help
rsync  version 3.1.0  protocol version 31
Copyright (C) 1996-2013 by Andrew Tridgell, Wayne Davison, and others.
Web site: http://rsync.samba.org/
Capabilities:
    64-bit files, 64-bit inums, 32-bit timestamps, 64-bit long ints,
    socketpairs, hardlinks, symlinks, IPv6, batchfiles, inplace,
    append, ACLs, xattrs, iconv, symtimes, prealloc

rsync comes with ABSOLUTELY NO WARRANTY.  This is free software, and you
are welcome to redistribute it under certain conditions.  See the GNU
General Public Licence for details.

rsync is a file transfer program capable of efficient remote update
via a fast differencing algorithm.

Usage: rsync [OPTION]... SRC [SRC]... DEST
  or   rsync [OPTION]... SRC [SRC]... [USER@]HOST:DEST
  or   rsync [OPTION]... SRC [SRC]... [USER@]HOST::DEST
  or   rsync [OPTION]... SRC [SRC]... rsync://[USER@]HOST[:PORT]/DEST
  or   rsync [OPTION]... [USER@]HOST:SRC [DEST]
  or   rsync [OPTION]... [USER@]HOST::SRC [DEST]
  or   rsync [OPTION]... rsync://[USER@]HOST[:PORT]/SRC [DEST]
The ':' usages connect via remote shell, while '::' & 'rsync://' usages connect
to an rsync daemon, and require SRC or DEST to start with a module name.

Options
 -v, --verbose               increase verbosity
     --info=FLAGS            fine-grained informational verbosity
     --debug=FLAGS           fine-grained debug verbosity
     --msgs2stderr           special output handling for debugging
 -q, --quiet                 suppress non-error messages
     --no-motd               suppress daemon-mode MOTD (see manpage caveat)
 -c, --checksum              skip based on checksum, not mod-time & size
 -a, --archive               archive mode; equals -rlptgoD (no -H,-A,-X)
     --no-OPTION             turn off an implied OPTION (e.g. --no-D)
 -r, --recursive             recurse into directories
 -R, --relative              use relative path names
     --no-implied-dirs       don't send implied dirs with --relative
 -b, --backup                make backups (see --suffix & --backup-dir)
     --backup-dir=DIR        make backups into hierarchy based in DIR
     --suffix=SUFFIX         set backup suffix (default ~ w/o --backup-dir)
 -u, --update                skip files that are newer on the receiver
     --inplace               update destination files in-place (SEE MAN PAGE)
     --append                append data onto shorter files
     --append-verify         like --append, but with old data in file checksum
 -d, --dirs                  transfer directories without recursing
 -l, --links                 copy symlinks as symlinks
 -L, --copy-links            transform symlink into referent file/dir
     --copy-unsafe-links     only "unsafe" symlinks are transformed
     --safe-links            ignore symlinks that point outside the source tree
     --munge-links           munge symlinks to make them safer (but unusable)
 -k, --copy-dirlinks         transform symlink to a dir into referent dir
 -K, --keep-dirlinks         treat symlinked dir on receiver as dir
 -H, --hard-links            preserve hard links
 -p, --perms                 preserve permissions
 -E, --executability         preserve the file's executability
     --chmod=CHMOD           affect file and/or directory permissions
 -A, --acls                  preserve ACLs (implies --perms)
 -X, --xattrs                preserve extended attributes
 -o, --owner                 preserve owner (super-user only)
 -g, --group                 preserve group
     --devices               preserve device files (super-user only)
     --specials              preserve special files
 -D                          same as --devices --specials
 -t, --times                 preserve modification times
 -O, --omit-dir-times        omit directories from --times
 -J, --omit-link-times       omit symlinks from --times
     --super                 receiver attempts super-user activities
     --fake-super            store/recover privileged attrs using xattrs
 -S, --sparse                handle sparse files efficiently
     --preallocate           allocate dest files before writing them
 -n, --dry-run               perform a trial run with no changes made
 -W, --whole-file            copy files whole (without delta-xfer algorithm)
 -x, --one-file-system       don't cross filesystem boundaries
 -B, --block-size=SIZE       force a fixed checksum block-size
 -e, --rsh=COMMAND           specify the remote shell to use
     --rsync-path=PROGRAM    specify the rsync to run on the remote machine
     --existing              skip creating new files on receiver
     --ignore-existing       skip updating files that already exist on receiver
     --remove-source-files   sender removes synchronized files (non-dirs)
     --del                   an alias for --delete-during
     --delete                delete extraneous files from destination dirs
     --delete-before         receiver deletes before transfer, not during
     --delete-during         receiver deletes during the transfer
     --delete-delay          find deletions during, delete after
     --delete-after          receiver deletes after transfer, not during
     --delete-excluded       also delete excluded files from destination dirs
     --ignore-missing-args   ignore missing source args without error
     --delete-missing-args   delete missing source args from destination
     --ignore-errors         delete even if there are I/O errors
     --force                 force deletion of directories even if not empty
     --max-delete=NUM        don't delete more than NUM files
     --max-size=SIZE         don't transfer any file larger than SIZE
     --min-size=SIZE         don't transfer any file smaller than SIZE
     --partial               keep partially transferred files
     --partial-dir=DIR       put a partially transferred file into DIR
     --delay-updates         put all updated files into place at transfer's end
 -m, --prune-empty-dirs      prune empty directory chains from the file-list
     --numeric-ids           don't map uid/gid values by user/group name
     --usermap=STRING        custom username mapping
     --groupmap=STRING       custom groupname mapping
     --chown=USER:GROUP      simple username/groupname mapping
     --timeout=SECONDS       set I/O timeout in seconds
     --contimeout=SECONDS    set daemon connection timeout in seconds
 -I, --ignore-times          don't skip files that match in size and mod-time
 -M, --remote-option=OPTION  send OPTION to the remote side only
     --size-only             skip files that match in size
     --modify-window=NUM     compare mod-times with reduced accuracy
 -T, --temp-dir=DIR          create temporary files in directory DIR
 -y, --fuzzy                 find similar file for basis if no dest file
     --compare-dest=DIR      also compare destination files relative to DIR
     --copy-dest=DIR         ... and include copies of unchanged files
     --link-dest=DIR         hardlink to files in DIR when unchanged
 -z, --compress              compress file data during the transfer
     --compress-level=NUM    explicitly set compression level
     --skip-compress=LIST    skip compressing files with a suffix in LIST
 -C, --cvs-exclude           auto-ignore files the same way CVS does
 -f, --filter=RULE           add a file-filtering RULE
 -F                          same as --filter='dir-merge /.rsync-filter'
                             repeated: --filter='- .rsync-filter'
     --exclude=PATTERN       exclude files matching PATTERN
     --exclude-from=FILE     read exclude patterns from FILE
     --include=PATTERN       don't exclude files matching PATTERN
     --include-from=FILE     read include patterns from FILE
     --files-from=FILE       read list of source-file names from FILE
 -0, --from0                 all *-from/filter files are delimited by 0s
 -s, --protect-args          no space-splitting; only wildcard special-chars
     --address=ADDRESS       bind address for outgoing socket to daemon
     --port=PORT             specify double-colon alternate port number
     --sockopts=OPTIONS      specify custom TCP options
     --blocking-io           use blocking I/O for the remote shell
     --stats                 give some file-transfer stats
 -8, --8-bit-output          leave high-bit chars unescaped in output
 -h, --human-readable        output numbers in a human-readable format
     --progress              show progress during transfer
 -P                          same as --partial --progress
 -i, --itemize-changes       output a change-summary for all updates
     --out-format=FORMAT     output updates using the specified FORMAT
     --log-file=FILE         log what we're doing to the specified FILE
     --log-file-format=FMT   log updates using the specified FMT
     --password-file=FILE    read daemon-access password from FILE
     --list-only             list the files instead of copying them
     --bwlimit=RATE          limit socket I/O bandwidth
     --outbuf=N|L|B          set output buffering to None, Line, or Block
     --write-batch=FILE      write a batched update to FILE
     --only-write-batch=FILE like --write-batch but w/o updating destination
     --read-batch=FILE       read a batched update from FILE
     --protocol=NUM          force an older protocol version to be used
     --iconv=CONVERT_SPEC    request charset conversion of filenames
     --checksum-seed=NUM     set block/file checksum seed (advanced)
 -4, --ipv4                  prefer IPv4
 -6, --ipv6                  prefer IPv6
     --version               print version number
(-h) --help                  show this help (-h is --help only if used alone)

Use "rsync --daemon --help" to see the daemon-mode command-line options.
Please see the rsync(1) and rsyncd.conf(5) man pages for full documentation.
See http://rsync.samba.org/ for updates, bug reports, and answers
\end{lstlisting}

\section{Ejemplo completo de configuración DRBD}
\label{sec:confdrbd}

\begin{lstlisting}
resource example {
	options {
		on-no-data-accessible suspend-io;
	}

	net {
		cram-hmac-alg "sha1";
		shared-secret "secret_string";
	}

	# The disk section is possible on resource level and in each
	# volume section
	disk {
		# If you have a resonable RAID controller
		# with non volatile write cache (BBWC, flash)
		disk-flushes no;
		disk-barrier no;
		md-flushes no;
	}

	# volume sections on resource level, are inherited to all node
	# sections. Place it here if the backing devices have the same
	# device names on all your nodes.
	volume 1 {
		device minor 1;
		disk /dev/sdb1;
		meta-disk internal;

		disk {
			resync-after example/0;
		}
	}

	on wurzel {
		address	192.168.47.1:7780;

		volume 0 {
		       device minor 0;
		       disk /dev/vg_wurzel/lg_example;
		       meta-disk /dev/vg_wurzel/lv_example_md;
		}
	}
	on sepp {
		address	192.168.47.2:7780;

		volume 0 {
		       device minor 0;
		       disk /dev/vg_sepp/lg_example;
		       meta-disk /dev/vg_sepp/lv_example_md;
		}
	}
}

resource "ipv6_example_res" {
	net {
		cram-hmac-alg "sha1";
		shared-secret "ieho4CiiUmaes6Ai";
	}

	volume 2 {
		device	"/dev/drbd_fancy_name" minor 0;
		disk	/dev/vg0/example2;
		meta-disk internal;
	}

	on amd {
		# Here is an example of ipv6.
		# If you want to use ipv4 in ipv6 i.e. something like [::ffff:192.168.22.11]
		# you have to set disable-ip-verification in the global section.
		address	ipv6 [fd0c:39f4:f135:305:230:48ff:fe63:5c9a]:7789;
	}

	on alf {
		address ipv6 [fd0c:39f4:f135:305:230:48ff:fe63:5ebe]:7789;
	}
}


#
# A two volume setup with a node for disaster recovery in an off-site location.
#

resource alpha-bravo {
	net {
		cram-hmac-alg "sha1";
		shared-secret "Gei6mahcui4Ai0Oh";
	}

	on alpha {
		volume 0 {
			device minor 0;
			disk /dev/foo;
			meta-disk /dev/bar;
		}
		volume 1 {
			device minor 1;
			disk /dev/foo1;
			meta-disk /dev/bar1;
		}
		address	192.168.23.21:7780;
	}
	on bravo {
		volume 0 {
			device minor 0;
			disk /dev/foo;
			meta-disk /dev/bar;
		}
		volume 1 {
			device minor 1;
			disk /dev/foo1;
			meta-disk /dev/bar1;
		}
		address	192.168.23.22:7780;
	}
}

resource stacked_multi_volume {
	net {
		protocol A;

		on-congestion pull-ahead;
		congestion-fill 400M;
		congestion-extents 1000;
	}

	disk {
		c-fill-target 10M;
	}

	volume 0 { device minor 10; }
	volume 1 { device minor 11; }

	proxy {
		memlimit 500M;
		plugin {
			lzma contexts 4 level 9;
		}
	}

	stacked-on-top-of alpha-bravo {
		address	192.168.23.23:7780;

		proxy on charly {
			# In the regular production site, there is a dedicated host to run
			# DRBD-proxy
			inside    192.168.23.24:7780; # for connections to DRBD
			outside   172.16.17.18:7780; # for connections over the WAN or VPN
			options {
				memlimit 1G; # Additional proxy options are possible here
			}
		}
	}
	on delta {
		volume 0 {
			device minor 0;
			disk /dev/foo;
			meta-disk /dev/bar;
		}
		volume 1 {
			device minor 1;
			disk /dev/foo1;
			meta-disk /dev/bar1;
		}
		address	127.0.0.2:7780;

		proxy on delta {
			# In the DR-site the proxy runs on the machine that stores the data
			inside 127.0.0.1:7780;
			outside 172.16.17.19:7780;
		}
	}
}

resource drbd_9_two_connection {
	volume 0 {
	       device minor 10;
	       disk /dev/foo/bar;
	       meta-disk internal;
	}

	on alpha {
		node-id 0;
		address 192.168.31.1:7800;
	}
	on bravo {
		node-id 1;
		address 192.168.31.2:7800;
	}
	on charlie {
		node-id 2;
		address 192.168.31.3:7800;
	}

	net {
		ko-count 3;
	}

	connection "optional name" {
		host alpha;
		host bravo;
		net { protocol C; }
	}

	connection {
		host alpha address 127.0.0.1:7800 via proxy on alpha {
			inside 127.0.0.2:7800;
			outside 192.168.31.1:7801;
		}
		host charlie address 127.0.0.1:7800 via proxy on charlie {
			inside 127.0.0.2:7800;
			outside 192.168.31.3:7800;
		}
		net { protocol A; }
	}

	connection {
		host bravo address 127.0.0.1:7800 via proxy on bravo {
			inside 127.0.0.2:7800;
			outside 192.168.31.2:7801;
		}
		host charlie address 127.0.0.1:7800 via proxy on charlie {
			inside 127.0.0.2:7800;
			outside 192.168.31.3:7800;
		}
		net { protocol A; }
	}
}

resource drbd_9_mesh {
	volume 0 {
	       device minor 11;
	       disk /dev/foo/bar2;
	       meta-disk internal;
	}

	on alpha {
		node-id 0;
		address 192.168.31.1:7900;
	}
	on bravo {
		node-id 1;
		address 192.168.31.2:7900;
	}
	on charlie {
		node-id 2;
		address 192.168.31.3:7900;
	}

	connection-mesh {
		hosts alpha bravo charlie;
		net {
			protocol C;
		}
	}
}
\end{lstlisting}

%------------------------------------------------------------

\end{document}
