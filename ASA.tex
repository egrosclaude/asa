
\documentclass[11pt,a4paper]{article}
\usepackage[utf8x]{inputenc}
\usepackage[T1]{fontenc}
\usepackage[spanish]{babel}
\usepackage{amsmath}
\usepackage{amssymb,amsfonts,textcomp}
\usepackage{color}
\usepackage{array}
\usepackage{multirow}
\usepackage{hhline}
\usepackage{hyperref}
\usepackage{float}
\usepackage{xkeyval}
\usepackage[pdftex]{graphicx}
\usepackage[yyyymmdd,hhmmss]{datetime}
\usepackage[usenames,dvipsnames]{xcolor}
\usepackage{appendix}
\usepackage{listings}
\definecolor{dkgreen}{rgb}{0,0.6,0}
\definecolor{gray}{rgb}{0.5,0.5,0.5}
\definecolor{mauve}{rgb}{0.58,0,0.82}
\definecolor{lstbackground}{rgb}{0.90,0.90,0.90}
% \lstset{frame=tb,
% 	backgroundcolor=\color{lstbackground},
%   language=Bash,
%   aboveskip=3mm,
%   belowskip=3mm,
%   showstringspaces=false,
%   columns=flexible,
%   basicstyle={\small\ttfamily},
%   numbers=none,
%   numberstyle=\tiny\color{gray},
%   keywordstyle=\color{blue},
%   %commentstyle=\color{dkgreen},
%   stringstyle=\color{mauve},
%   breaklines=true,
%   breakatwhitespace=true,
%   extendedchars=true,
%   tabsize=3
% }
\lstset{frame=tb,
	backgroundcolor=\color{lstbackground},
%  language=Bash,
  aboveskip=3mm,
  belowskip=3mm,
  showstringspaces=false,
  columns=flexible,
  basicstyle={\small\ttfamily},
  numbers=none,
%  numberstyle=\tiny\color{gray},
%  keywordstyle=\color{blue},
%  commentstyle=\color{dkgreen},
%  stringstyle=\color{mauve},
  breaklines=true,
  breakatwhitespace=true
  tabsize=4
}
\usepackage{caption}

%\DeclareCaptionFont{black}{ \color{black} }
%\DeclareCaptionFormat{listing}{
%  \colorbox[cmyk]{0.93, 0.95, 0.95,0.01 }{
%    \parbox{\textwidth}{\hspace{15pt}#1#2#3}
%  }
%}
%\captionsetup[lstlisting]{ format=listing, labelfont=black, textfont=black, singlelinecheck=false, margin=0pt, font={bf,footnotesize} }
\captionsetup[lstlisting]{format=plain, font={footnotesize}}
% ...


%\renewcommand{\lstlistingname}{Code}
\usepackage{verbatim}
\begin{comment}
	\hypersetup{
		pdftex, 
		colorlinks=true, 
		linkcolor=blue, 
		citecolor=blue, 
		filecolor=blue, 
		urlcolor=blue, 
	pdftitle={Software Libre}, 
	pdfauthor={Eduardo Grosclaude}, 
	pdfsubject={Documento de la materia Software Libre}, 
	pdfkeywords={Software Libre, Tecnicatura en Administración de Sistemas y 		Software Libre, Universidad Nacional del Comahue}
	}
\end{comment}	



%\addto\captionsspanish {%
%	\def\appendixname{Apéndices}
%}
% Outline numbering
\setcounter{secnumdepth}{1}
% Reset section numbering between parts
\makeatletter
\@addtoreset{section}{part}
\makeatother  
% List styles
\newcommand\liststyleLi{%
\renewcommand\labelitemi{\tiny${\blacksquare}$}
\renewcommand\labelitemii{\tiny${\square}$}
\renewcommand\labelitemiii{\tiny${\circ}$}
\renewcommand\labelitemiv{\tiny${\circ}$}
}
\newcommand\liststyleLii{%
\renewcommand\labelitemi{{\textbullet}}
\renewcommand\labelitemii{${\circ}$}
\renewcommand\labelitemiii{${\blacksquare}$}
\renewcommand\labelitemiv{{\textbullet}}
}
\newcommand\liststyleLiii{%
\renewcommand\labelitemi{{\textbullet}}
\renewcommand\labelitemii{${\circ}$}
\renewcommand\labelitemiii{${\blacksquare}$}
\renewcommand\labelitemiv{{\textbullet}}
}

\liststyleLi

% Page layout (geometry)
\setlength\voffset{-1in}
\setlength\hoffset{-1in}
\setlength\topmargin{2cm}
\setlength\oddsidemargin{2cm}
\setlength\textheight{23.246668cm}
\setlength\textwidth{17.006cm}
\setlength\footskip{26.144882pt}
\setlength\headheight{1.016cm}
\setlength\headsep{0.508cm}
% Footnote rule
\setlength{\skip\footins}{0.119cm}
\renewcommand\footnoterule{\vspace*{-0.018cm}\setlength\leftskip{0pt}\setlength\rightskip{0pt plus 1fil}\noindent\textcolor{black}{\rule{0.25\columnwidth}{0.018cm}}\vspace*{0.101cm}}
% Pages styles
\makeatletter
\newcommand\ps@Standard{
  \renewcommand\@oddhead{{\raggedleft Cabecera \ } {\raggedright \thepage{}}}
  \renewcommand\@evenhead{\@oddhead}
  \renewcommand\@oddfoot{}
  \renewcommand\@evenfoot{\@oddfoot}
  \renewcommand\thepage{\arabic{page}}
}

% \pagestyle{Standard}
\usepackage{fancyhdr}
\pagestyle{fancy}


%%--------------------------------------
% F O N T S 
%\usepackage{fancyhdr}
\usepackage{sans}
%\usepackage{libertine}
%\usepackage{lmodern}
%\usepackage{opensans}
%\usepackage{helvet}

%\usepackage{times}

%% LaTeX Preamble - Font choices
%% Each block selects new math, roman (serif), sans serif, and typewriter fonts.
%% Delete or comment out all but one to make your choice.

% Fourier for math | Utopia (scaled) for rm | Helvetica for ss | Latin Modern for tt
%\usepackage{fourier} % math & rm
%\usepackage[scaled=0.875]{helvet} % ss
%\renewcommand{\ttdefault}{lmtt} %tt

% Latin Modern (similar to CM with more characters)
%\usepackage{lmodern} % math, rm, ss, tt
%\usepackage[T1]{fontenc}

% Palatino for rm and math | Helvetica for ss | Courier for tt
%\usepackage{mathpazo} % math & rm
%\linespread{1.05}        % Palatino needs more leading (space between lines)
%\usepackage[scaled]{helvet} % ss
%\usepackage{courier} % tt
%\normalfont
%\usepackage[T1]{fontenc}

% Euler for math | Palatino for rm | Helvetica for ss | Courier for tt
%\renewcommand{\rmdefault}{ppl} % rm
%\linespread{1.05}        % Palatino needs more leading
%\usepackage[scaled]{helvet} % ss
%\usepackage{courier} % tt
%\usepackage{euler} % math
%\usepackage{eulervm} % a better implementation of the euler package (not in gwTeX)

%\normalfont
%\usepackage[T1]{fontenc}

% Times for rm and math | Helvetica for ss | Courier for tt
%\usepackage{mathptmx} % rm & math
%\usepackage[scaled=0.90]{helvet} % ss
%\usepackage{courier} % tt
%\normalfont
%\usepackage[T1]{fontenc}

% !! COMMERICAL FONT !! Lucida Bright (w/expert package)
%\usepackage[T1]{fontenc}
%\usepackage[expert,vargreek,altbullet]{lucidabr}

%% END Font choices
%%---------------------------------------------
% \renewcommand*\familydefault{\sfdefault}
% \pagestyle{Standard}
\usepackage{mdframed}


% footnotes configuration
\makeatletter
\renewcommand\thefootnote{\arabic{footnote}}
\makeatother
\title{Administración de Sistemas Avanzada}
\author{Eduardo Grosclaude}
\date{2014-08-11}
\usepackage{graphicx}

\usepackage{xkeyval}
\usepackage{pifont}
\usepackage{xcolor}
\newcommand{\revisar}[1]{{\color{red}[#1]}}
%\newcommand{\nota}[1]{{\color{red}[#1]}}
%\newcommand{\revisar}[1]{}

\newcommand{\borrador}{
\revisar{\today, \currenttime  -  Material en preparación, se ruega no imprimir mientras aparezca esta nota}
}




\newcommand{\nota}[1]{}

\newcommand{\nonota}[1]{#1}

\newcommand{\quotes}[1]{``#1''}

   
\newcommand{\shade}[1]{\textcolor{black!50}{#1}}

% ancho opcional, por defecto 15cm
% \figura{copyleft}{Símbolo de Copyleft}{copyleft.png}
% \figura[6]{copyleft}{Símbolo de Copyleft}{copyleft.png}
\newcommand{\figura}[4][15]{
 \begin{figure}[htbp] 
 \centering 
 \includegraphics[width=#1cm]{./img/#4} 
 \caption{#3} 
 \label{fig:#2} 
 \end{figure} 
}

% tabla{label}{caption}{columns}{contents}
\newcommand{\tabla}[4]{
 \begin{table} 
 \centering 
 \small
 \begin{tabular}{#3}
 #4
 \end{tabular}
 \caption{#2}
 \label{tab:#1} 
 \end{table} 
}

\newcommand{\recuadro}[1]{
\begin{minipage}[c]{0.84\textwidth}
\begin{mdframed}
#1
\end{mdframed}
\end{minipage}
}


\hypersetup{pdftex, colorlinks=true, linkcolor=blue, citecolor=blue, filecolor=blue, urlcolor=blue, 
	pdftitle={Administración de Sistemas Avanzada}, 
	pdfauthor={Eduardo Grosclaude}, 
	pdfsubject={Documento de la materia Administración de Sistemas Avanzada}, 
	pdfkeywords={Administración de Sistemas, Alta Disponibilidad, Virtualización, Tecnicatura en Administración de Sistemas y Software Libre, Universidad Nacional del Comahue}}

 
% --------------------------------------------------------------------
\begin{document}

\maketitle

\revisar{V0.1 - Material en preparación, se ruega no imprimir mientras aparezca esta nota}

\abstract { En este escrito se presenta la descripción y material inicial de la asignatura \textbf{Administración de Sistemas Avanzada}, para la carrera de Tecnicatura Universitaria en Administración de Sistemas y Software Libre, de la Universidad Nacional del Comahue. 

La materia es cuatrimestral en modalidad presencial y las clases son de carácter teórico-práctico, desarrolladas en forma colaborativa. Está preparada con los objetivos generales de capacitar al estudiante para  \textbf{implementar configuraciones especiales de almacenamiento, aplicar programación avanzada a la automatización de tareas, y diseñar e implementar estrategias de respaldo y de tolerancia a fallos para servicios críticos}. 
 

\newpage
\emph{Página en blanco}
\newpage

\tableofcontents

\newpage
\emph{Página en blanco}

%----------- P R E S E N T A C I O N  ---------
\newpage
\part {La asignatura}


\section{Objetivos}
\subsection{De la carrera}
Según el documento fundamental de la Tecnicatura, el Técnico Superior en Administración de Sistemas y Software Libre estará capacitado para:
\begin{itemize}
	\item Desarrollar actividades de administración de infraestructura. Comprendiendo la administración de sistemas, redes y los distintos componentes que forman la
infraestructura de tecnología de una institución, ya sea pública o privada.
	\item Aportar criterios básicos para la toma de decisiones relativas a la adopción de nuevas tecnologías libres.
	\item Desempeñarse como soporte técnico, solucionando problemas afines por medio de la comunicación con comunidades de Software Libre, empresas y desarrolladores de
software.
	\item Realizar tareas de trabajo en modo colaborativo, intrínseco al uso de tecnologías libres.
	\item Comprender y adoptar el estado del arte local, nacional y regional en lo referente a implementación de tecnologías libres. Tanto en los aspectos técnicos como legales.
\end{itemize}
\subsection{De la asignatura}

\begin{itemize}
	\item Saber implementar configuraciones especiales de almacenamiento
	\item Saber aplicar programación avanzada a la automatización de tareas
	\item Saber diseñar e implementar estrategias de respaldo 
	\item Conocer formas de implementar estrategias de tolerancia a fallos para servicios críticos
\end{itemize}


\section{Cursado}
\begin{itemize}
	\item Cuatrimestral de 16 semanas, 128 horas totales
	\item Clases teórico-prácticas presenciales
	\item Promocionable con trabajos prácticos
\end{itemize}


\section {Contenidos}
\subsection{Contenidos mínimos}
\begin{itemize}
	\item  Instalación sobre configuraciones de almacenamiento especiales. 
	\item  Scripting avanzado. 
	\item  Planificación de tareas. 
	\item  Virtualización. 
	\item  Alta Disponibilidad.
\end{itemize}


\subsection {Programa}
\begin{enumerate}
\item Scripting avanzado
\begin{itemize}
	\item Estructuras de programación
	\item Scripting para tratamiento de archivos
	\item Planificación de tareas
\end{itemize}

\item Configuraciones de almacenamiento
\begin{itemize}
	\item Arquitectura de E/S, Dispositivos de E/S, Filesystems
	\item	Diseños típicos de almacenamiento
	\item	Software RAID, instalación y mantenimiento niveles 0, 1, 10
	\item	LVM, instalación y mantenimiento	 
\end{itemize}
	
\item Estrategias de respaldo
\begin{itemize}
	\item Copiado y sincronización de archivos
	\item Estrategias y herramientas de backup, LVM snapshots
	\item Control de versiones
\end{itemize}
\item Virtualización
\begin{itemize}
	\item Formas de virtualización, herramientas. KVM, Proxmox, otras
	\item Creación, instalación, migración de MV
	\item Cloud. IaaS, PaaS, SaaS, etc.
\end{itemize}
\item Alta Disponibilidad
\begin{itemize}
	\item Clustering de LB, de HA, de HPC. Conceptos de HA.
	\item Balance de Carga
	\item Heartbeat, DRBD, Clustering de aplicaciones
	\item Alta Disponibilidad en Redes. Bonding, STP
\end{itemize}
\end{enumerate}

\section {Bibliografía inicial}
\begin{itemize}
\item Kemp, Juliet. Linux System Administration Recipes: A Problem-Solution Approach. Apress, 2009. 
\item Lakshman, Sarath. Linux Shell Scripting Cookbook Solve Real-World Shell Scripting Problems with over 110 Simple but Incredibly Effective Recipes. Birmingham, U.K.: Packt Pub., 2011. 
\item Parker, Steve. Shell Scripting Expert Recipes for Linux, Bash, and More. Hoboken, N.J.; Chichester: Wiley; John Wiley, 2011.
\item Quigley, Ellie. UNIX Shells by Example. 3rd ed. Upper Saddle River, NJ: Prentice Hall, 2002.
\item W. Soyinka, Linux administration a beginners guide. New York, NY: McGraw-Hill Osborne Media, 2012.
\item C. Wolf and E. M. Halter, Virtualization from the desktop to the enterprise. Berkeley, CA; New York, NY: Apress; Distributed in U.S. by Springer-Verlag New York, 2005.



\end{itemize}



%
\section{Evaluación}
La evaluación de la materia se realizará mediante trabajos grupales de investigación y desarrollo sobre proyectos de Software Libre, de la siguiente manera.
\begin{itemize}
	\item Los estudiantes se dividirán en grupos de 2 a 5 personas. 
	\item Los grupos desarrollarán trabajos prácticos en etapas que se distribuirán a lo largo de la materia. 
	\item Cada grupo abrirá un diario, blog o wiki de acceso público en cualquier sitio disponible y publicará, mediante el Foro de la materia, la forma de acceder al diario para lectura. Los docentes y los demás estudiantes de la materia podrán acceder al diario del grupo para lectura. Todo cambio en la dirección o forma de acceso deberá ser informado mediante el Foro.
	\item El grupo irá aportando los resultados de cada etapa de los trabajos a su diario, y periódicamente comentará además en clase las experiencias surgidas durante la realización de los trabajos.
	\item El material publicado en el diario será reunido en un documento final que será entregado \textbf{en formato electrónico} al finalizar la materia. El documento indicará tema del trabajo, resumen, integrantes del grupo, desarrollo y conclusiones. 
	\item El documento será acompañado por una presentación de no más de treinta minutos que será expuesta según el cronograma adjunto. 
	\item La acreditación final tendrá en cuenta la calidad del material aportado al diario por el grupo, la calidad de los documentos finales de los trabajos, la presentación oral y la participación en clase ofreciendo la experiencia adquirida durante la realización de los trabajos.
\end{itemize}

\subsection {Trabajo I - Colaboración con proyectos libres}
\subsubsection{Etapa 1}  
Descargar e instalar software ofrecido por un proyecto de Software Libre que esté en actividad (puede tratarse de un entorno de escritorio, un programa de sistema, programas de usuario final, una distribución completa, etc.). Familiarizarse con el software utilizándolo. 
\subsubsection{Etapa 2} 
Basándose en el conocimiento adquirido con el uso del software, colaborar de alguna forma con el proyecto que lo origina: 
\begin{itemize}
	\item traduciendo o localizando parte del software,
	\item generando documentación faltante, 
	\item traduciendo parte de la documentación, 
	\item detectando y denunciando errores en el software o en la documentación,
	\item aportando, modificando o corrigiendo código,
	\item aportando conocimiento a los usuarios del proyecto en blogs, salas de chat, bases de conocimiento, etc.
\end{itemize}
Puede abordarse cualquier cantidad manejable de proyectos. La colaboración debe consistir en alguna interacción positiva y completa con cada proyecto. El grupo incorporará al diario los reportes que acrediten esa interacción. Cuando no sea posible realizar o completar la interacción se indicarán las causas, y las acciones realizadas.

El aporte al proyecto debe efectuarse por los canales establecidos por el proyecto. Si se trata de documentación, respetar el formato utilizado; si es el reporte de un error, hacerlo por la vía preferida por el proyecto, etc.

\subsubsection{Etapa 3} 
El grupo entregará un documento conteniendo la historia de las interacciones con cada proyecto, adjuntando las pruebas en anexos y ofrecerá una presentación.

\subsection {Trabajo II - Evaluación de proyectos libres}

\subsubsection{Etapa 1} 
El grupo enunciará un determinado requerimiento concreto de software que puede ser presentado por un empleador. Algunos ejemplos posibles son:
\begin{itemize}
	\item \quotes{un servidor de correo electrónico que maneje listas},
	\item  \quotes{una aplicación de control de asistencia para empleados},
	\item  \quotes{un sistema de edición de textos para traductores},
	\item  \quotes{un sistema de gestión de contenidos web que incluya workflow}, 
	\item \quotes{un motor de juegos 2D para crear juegos que asistan en la enseñanza de matemática},
	\item  \quotes{un programa de simulación de ataques para evaluar postura de seguridad}, 
	\item \quotes{un sistema de control de stock para zapaterías},
	\item \quotes{una distribución de GNU/Linux para escuelas de arte},
	\item \quotes{una distribución para sistemas empotrados}, etc.
\end{itemize}
El grupo debe comprender el propósito del software requerido y debe contar con al menos un integrante con conocimiento razonable de la temática involucrada. El grupo escribirá una entrada en el diario consignando toda la información posible sobre los requerimientos. 

\subsubsection{Etapa 2} 

\begin{itemize}
	\item El grupo $n$ (en adelante \quotes{el proveedor}) tomará a su cargo el requerimiento del grupo $n+1$ (en adelante \quotes{el cliente}), y se atendrá a dicha descripción para el resto del trabajo. 
	\item El grupo proveedor buscará proyectos de SL que apunten a cubrir esos requerimientos y seleccionará al menos dos proyectos, idealmente tres, de entre ellos.
\end{itemize}

\subsubsection{Etapa 3}
Los proyectos serán comparados en función de varios parámetros o dimensiones.
\begin{itemize}
	\item  ajuste a los requerimientos (actual, previsto o potencial),
	\item  licenciamiento, 
	\item  motivación del desarrollo, 
	\item  modelos de negocio del proyecto, 
	\item  tamaño y permanencia de la comunidad,
	\item  dinámica de soporte, 
	\item  dinámica de actualizaciones y mejoras del software.
\end{itemize}

Se podrán agregar a la comparación uno o más desarrollos no libres. 

Las dudas sobre detalles de los requerimientos serán dirigidas al grupo cliente, y contestadas por aquél, mediante el Foro de la página de la materia.  
\subsubsection{Etapa 4} 
El grupo entregará un documento conteniendo la comparación y haciendo una recomendación final, explicando sus fundamentos. Deberán volcar en el trabajo lo que se vaya aprendiendo durante el curso de la materia, en cada uno de los parámetros o dimensiones nombrados. Finalmente ofrecerán una presentación sobre el trabajo.

\label{sub:acreditacion}
%
\subsection {Cronograma de ejecución}
\begin{tabular}{c|l|l|l}
Semana & Unidad & Trabajo I & Trabajo II\\
\hline
\hline
1	& 	1. Introducción, Software Libre & Etapa 1 &  \\
2 	& 								 	& \\
\hline
\hline
3	& 	2. Aspectos técnicos			& Etapa 2 &  \\
4 	& 									&\\
5	& 									&\\
6	& 									&\\
\hline
\hline
7 	& 	3. Aspectos legales				& Etapa 3 \\
8	& 									& Entrega y presentaciones\\ 
9	& 									& & Etapas 1 y 2\\
\hline
\hline
10	& 	4. Uso de SL					&& Etapa 3\\ 
11	& 									& \\
12	& 									&\\
13	& 									&\\
\hline
\hline
14	& 	5. Producción de SL				&& Etapa 4\\
15	& 									&\\
16	& 									&& Entrega y presentaciones\\ 
\hline
\end{tabular}



% \begin{tabular}{|r|c|c|c|c|c|c|c|c|}
% \hline
%\textsf{7} & fbox {algo} & & & & & & &\\ 
%\hline
%\textsf{7} & & & & & & & &\\ 
%\hline
%\end{tabular}

% subsection  (end)


%----------- M A T E R I A L ---------
\newpage
\part {Scripting Avanzado}
\section {Contenidos}

\begin{enumerate}
\item Comandos básicos de archivos ls, cd, mkdir, cp, mv, rm, ln, patrones de nombres
\item Redirección y piping, comandos head, tail, more, less, grep
\item Variables, ambiente, aritmética
\item Sentencias de control if, for, while, case
\item Funciones
\item Arreglos
\item Expresiones regulares, uso de grep
\item Uso de sort, diff, comm, uniq, cut
\item Uso de cron
\item Otros intérpretes: sed, awk, Perl
\end{enumerate}

\section{Ejercitación básica}

\subsection{Redirección y piping}
\begin{enumerate}
	\item Crear un archivo conteniendo la salida del comando ls
	\item Crear un archivo conteniendo la salida del comando ls -lR /tmp
	\item Obtener las cinco primeras líneas del archivo anterior
	\item Crear un archivo conteniendo las cinco primeras líneas y las cinco  últimas del archivo generado en 2
	\item Crear un archivo conteniendo las primeras cinco líneas de la salida del comando ls -lR /tmp
	\item Usando el anterior, crear un archivo conteniendo esas líneas, numeradas
	\item Crear un archivo conteniendo las últimas cinco líneas de la salida del comando ls -lR /tmp
\end{enumerate}


\subsection {Variables, ambiente}
\begin{enumerate}
	\item Asignar e imprimir el contenido de dos variables
	\item Asignar dos variables, imprimir sus valores, intercambiar sus valores, imprimirlos
	\item Crear un script que imprima un valor que será pasado como argumento
	\item Crear un script que imprima dos valores que serán pasados como argumento
	\item Crear un script que imprima todos los valores que le sean pasados como argumento
\end{enumerate}


\subsection{Sentencias de control}
\begin{enumerate}
	\item 
Imprimir cinco veces "Linux"
	\item 
Imprimir cinco veces el contenido de una variable
	\item 
Imprimir los números de 0 a 5
	\item 
Imprimir los dígitos de -1 a 6
	\item 
Imprimir los números de 0 a 99
	\item 
Imprimir junto al nombre de cada archivo en el directorio actual, su tamaño y su fecha de modificación
	\item 
Copiar los archivos terminados en .txt en archivos con igual nombre pero extensión .bak
	\item 
Renombrar los archivos con extensión .tex que comienzan en ASA reemplazando la partícula ASA con RII
	\item 
Para cada archivo modificado hace más de cinco días en un directorio, mostrar su cantidad de líneas
	\item 
Obtener mediante un cliente de HTTP una lista de archivos cuyos nombres están dados por  una expresión variable y controlada por un lazo
	\item 
De un conjunto de archivos tar, encontrar aquellas versiones de un archivo dado, contenido en ellos, que hayan sido modificadas entre dos fechas dadas.
\end{enumerate}


\subsection{Aritmética}
\begin{lstlisting}
$ declare -i num
$ num="hola"
$ echo $num
	0
$ num=5 + 5
	bash: +: command not found
$ num=5+5
$ echo $num
	10
$ num=4*6
$ echo $num
	24
$ num="4 * 6"
$ echo $num
	24
$ num=6.5
	bash: num: 6.5: syntax error in expression (remainder of expression is ".5")
$ i=5; j=$i+1; echo $j
$ i=5; let j=$i+1; echo $j
$ let i=5
$ let i=i+1
$ echo $i
	6
$ let "i = i + 2"
$ echo $i
	8
$ let "i+=1"
$ echo $i
	9
$ i=3
$ (( i+=4 ))
$ echo $i
	7
$ (( i=i-2 ))
$ echo $i
	5
$ let b=2#101; echo $b
$ let h=16#ABCD; echo $h
\end{lstlisting}

\subsection{Arreglos}
\begin{lstlisting}
$ A=(1 2 3 cuatro cinco)
$ echo ${!A[*]}
0 1 2 3 4
$ echo ${A[4]}
cinco
$ echo ${A[*]}
1 2 3 cuatro cinco
$ A[2]='banana'
$ echo ${A[*]}
1 2 banana cuatro cinco
\end{lstlisting}

\subsection{Arreglos asociativos}
\begin{lstlisting}
$ declare -A B
$ B=([francia]='paris' [espana]='madrid' [argentina]='buenos aires')
$ echo ${!B[*]}
espana argentina francia
$ echo ${B[*]}
madrid buenos aires paris
$ echo ${B[francia]}
paris
\end{lstlisting}


\subsection{Here-Documents}
\begin{lstlisting}
$ cat > texto.txt << END
> Hola
> Probando...
> END
$ cat texto.txt
\end{lstlisting}

\subsection{Traps}
\begin{lstlisting}
# man 7 signal
# 1 = SIGHUP (Hangup of controlling terminal or death of parent)
# 2 = SIGINT (Interrupted by the keyboard)
# 3 = SIGQUIT (Quit signal from keyboard)
# 6 = SIGABRT (Aborted by abort(3))
# 9 = SIGKILL (Sent a kill command)

trap limpieza 1 2 3 6 9
function limpieza
{
	echo "Recibimos senal - desmantelando..."
	rm -f ${tempfiles}
	echo Finalizando
}
\end{lstlisting}



\section{Casos de uso}


\subsection{Investigar el sistema}
\begin{enumerate}
	\item 
Modificar la salida del comando blkid para conocer el UUID, el nombre y tipo, y punto de montado, de cada dispositivo de bloques del sistema.
	\item 
Analizar archivos de log buscando conocimiento: duración de sesiones ssh por usuario, mensajes de mail entre usuarios, con histograma por tamaños, etc. (ver iptables.log, \ref{sec:iptables.log})
	\item 
Detectar momentos en que la salida de vmstat muestra picos de I/O, procesos corriendo, procesos en espera, uso de swap, etc.
\end{enumerate}


\subsection{Recuperar espacio de almacenamiento}
\begin{enumerate}
	\item Encontrar los diez archivos más grandes en un directorio y sus hijos, imprimirlos junto con su tamaño de mayor a menor.
	\item Encontrar los diez archivos más grandes en un directorio y sus hijos, moverlos a otro directorio (en otro filesystem).
	\item Encontrar los diez archivos más grandes del sistema, imprimir el nombre de usuario dueño.
	\item Agregar al script anterior el envío de notificación por mail al usuario responsable.
		\item 
Encontrar archivos en directorios de usuario con la cadena \quotes{cache} en su nombre e imprimir el uso de disco de cada uno.
	\item 
Idem, enviando nombres a un archivo y usándolo como lista para borrarlos, comprimirlos o moverlos.

\end{enumerate}



\subsection{Networking}
\begin{enumerate}
	\item 
Disparar un aviso cuando se pierde la conectividad a un conjunto dado de nodos de la red.
	\item 
Analizar la salida del comando netstat para descubrir en qué momento aparece un nuevo port abierto y a qué aplicación corresponde.
	\item 
Obtener un log de tráfico y obtener orígenes máximos y mínimos de tráfico, cantidades totales de bytes traficados por interfaz, etc.
	\item 
Recoger estadísticas de espacio en disco, cantidad de procesos, carga de CPU, en diferentes nodos de la red, y centralizarlos en un nodo monitor que presente los resultados.
\end{enumerate}

\subsection{Seguridad}
\begin{enumerate}
	\item 
Detener el script si la identidad del proceso corresponde a root.
	\item 
Solicitar información confidencial (como claves) con video inhibido.
	\item 
Capturar señales para impedir la interrupción del script por BREAK o fallos de ejecución.
	\item 
Utilizar MD5/SHAx para confirmar integridad de archivos.
\end{enumerate}




\subsection{Tratamiento de datos}
\begin{enumerate}
	\item 
Revisar el uso de los comandos cut, join, sort, uniq, comm.
	\item 
Crear script que administra una base de datos en formato CSV.
	\item 
Dado un archivo con una lista de direcciones IP, adjuntarles la resolución inversa de nombres correspondiente.
	\item 
Crear un histograma de accesos por nombre de dominio, a partir de los paquetes registrados en un archivo de log generado por iptables. 
	\item 
Dada una base de datos CSV implementar búsqueda por expresiones regulares.
	\item 
Dada una base de datos CSV implementar proyección sobre un conjunto de campos dados.
	\item 
Convertir un listado de individuos PDF en archivo CSV.
	\item 
Preparar un conjunto de scripts con un único punto de entrada para el administrador. Estos scripts mantendrán un conjunto de bases de datos en formato CSV:
\begin{lstlisting}
alumnos: UID, Username, Apellido, Nombres, NoLegajo, Activo
materias: MID, Nombre, Carrera, Docente
cursadas: UID, MID, Ano, Cuatrimestre
\end{lstlisting}
El dato Activo es booleano. Con estas bases de datos:
\begin{itemize}
	\item 
Listar todas las materias asignadas a un mismo docente.
	\item 
Listar todas las materias cursadas por un alumno.
	\item 
Listar todos los alumnos activos inscriptos en una materia.
	\item 
Listar todos los alumnos que cursan una misma carrera dada durante un año dado.
	\item 
Listar todos los alumnos, agrupados por materia cursada, dentro de cada año. 
	\item 
Listar todos los alumnos de un mismo docente.
	\item 
Dado un alumno por su legajo, consultar su estado Activo/Inactivo.
	\item 
Para aquellos alumnos que hace más de tres años que no se inscriben en ninguna cursada, pasar su dato Activo a falso (Inactivo).
	\item 
Generar un par de archivos en el formato de /etc/passwd y /etc/shadow para todos los alumnos activos.
	\item 
Generar un directorio /home/usuario para cada alumno activo, con UID correspondiente.
\end{itemize}
\end{enumerate}


\subsection{Accesibilidad para usuarios finales}
\begin{enumerate}
	\item 
Preparar un script con interfaz gráfica para copiar archivos seleccionados a una carpeta preestablecida con el fin de obtener un backup periódico de todos sus contenidos.
	\item 
Preparar un script con interfaz gráfica que presente los cinco directorios con mayor ocupación de almacenamiento dentro del home del usuario.
	\item 
Agregar interfaz gráfica a los scripts de administración de bases de datos de alumnos y materias.
\end{enumerate}


\newpage
\part {Estrategias de Respaldo}
%\input {SL-part2-2}
\newpage
\part {Virtualización}
%\input {SL-part2-3}
\newpage 
\part {Alta Disponibilidad}
%\input {SL-part2-4}



%----------- A N E X O S ---------
\newpage
\part {Anexos}
\appendix
\section{iptables.log}
\label{sec:iptables.log}
\begin{lstlisting}
 Logged 539 packets on interface eth1
   From 0000:0000:1011:1213:0100:0000:0000:0000 - 3 packets to icmpv6(130)
   From 0000:0000:0000:859e:0100:0000:0000:0000 - 1 packet to icmpv6(130)
   From 0000:0023:ff53:4d42:0100:0000:0000:0000 - 1 packet to icmpv6(130)
   From 0000:0000:0000:3433:0100:0000:0000:0000 - 1 packet to icmpv6(130)
   From 0000:0000:0000:3132:0100:0000:0000:0000 - 1 packet to icmpv6(130)
   From 0000:0000:0000:7d3a:0100:0000:0000:0000 - 1 packet to icmpv6(130)
   From 0000:0000:0000:6e63:0100:0000:0000:0000 - 1 packet to icmpv6(130)
   From 0000:0000:0000:937f:0100:0000:0000:0000 - 1 packet to icmpv6(130)
   From 0000:0000:0000:0000:0000:0000:0000:0000 - 2 packets to icmpv6(130)
   From 0000:0000:0000:0000:0100:0000:0000:0000 - 40 packets to icmpv6(130)
   From 0000:0000:0000:6569:0100:0000:0000:0000 - 1 packet to icmpv6(130)
   From 000e:175f:531c:580e:0100:0000:0000:0000 - 1 packet to icmpv6(130)
   From 0011:11db:a2d4:0a00:0100:0000:0000:0000 - 2 packets to icmpv6(130)
   From 002c:799a:0694:8c26:0100:0000:0000:0000 - 1 packet to icmpv6(130)
   From 0100:0000:0600:0000:0100:0000:0000:0000 - 2 packets to icmpv6(130)
   From 0101:080a:00c6:0621:0100:0000:0000:0000 - 1 packet to icmpv6(130)
   From 0101:080a:00c6:f565:0007:994d:0000:0000 - 1 packet to icmpv6(130)
   From 0101:080a:00c7:39ad:0100:0000:0000:0000 - 1 packet to icmpv6(130)
   From 0101:080a:00c7:3e49:0100:0000:0000:0000 - 1 packet to icmpv6(130)
   From 0101:080a:00c7:5bd1:0100:0000:0000:0000 - 1 packet to icmpv6(130)
   From 0101:080a:1551:7183:0100:0000:0000:0000 - 1 packet to icmpv6(130)
   From 0101:080a:1a9a:1543:0100:0000:0000:0000 - 1 packet to icmpv6(130)
   From 0101:080a:4553:94db:0100:0000:0000:0000 - 1 packet to icmpv6(130)
   From 0101:080a:4557:126c:0100:0000:0000:0000 - 1 packet to icmpv6(130)
   From 0101:080a:4559:6ffd:0100:0000:0000:0000 - 1 packet to icmpv6(130)
   From 0101:080a:4559:e9ac:0100:0000:0000:0000 - 1 packet to icmpv6(130)
   From 0101:080a:455a:3fd8:0100:0000:0000:0000 - 1 packet to icmpv6(130)
   From 0101:080a:455a:cc78:0100:0000:0000:0000 - 1 packet to icmpv6(130)
   From 0101:080a:455c:c658:0100:0000:0000:0000 - 1 packet to icmpv6(130)
   From 0101:080a:455d:4147:0100:0000:0000:0000 - 1 packet to icmpv6(130)
   From 0101:080a:455d:bbfc:0100:0000:0000:0000 - 1 packet to icmpv6(130)
   From 0101:080a:455e:3351:0100:0000:0000:0000 - 1 packet to icmpv6(130)
   From 0101:080a:4567:104c:0100:0000:0000:0000 - 1 packet to icmpv6(130)
   From 0101:080a:4575:8fa3:0100:0000:0000:0000 - 1 packet to icmpv6(130)
   From 0101:080a:4575:fcd3:0100:0000:0000:0000 - 1 packet to icmpv6(130)
   From 0101:080a:4584:d388:0100:0000:0000:0000 - 1 packet to icmpv6(130)
   From 0101:080a:458c:f368:0100:0000:0000:0000 - 1 packet to icmpv6(130)
   From 0101:080a:4594:8809:0100:0000:0000:0000 - 1 packet to icmpv6(130)
   From 0102:0417:0000:0000:0100:0000:0000:0000 - 1 packet to icmpv6(130)
   From 0102:a16d:0a00:00c8:0100:0000:0000:0000 - 1 packet to icmpv6(130)
   From 0204:05b4:0402:080a:0100:0000:0000:0000 - 2 packets to icmpv6(130)
   From 0300:0000:0400:0000:0100:0000:0000:0000 - 3 packets to icmpv6(130)
   From 036b:696d:0675:6e63:0100:0000:0000:0000 - 1 packet to icmpv6(130)
   From 10d4:d5cb:f80c:14d5:0100:0000:0000:0000 - 1 packet to icmpv6(130)
   From 1400:0300:9709:0000:0100:0000:0000:0000 - 1 packet to icmpv6(130)
   From 2269:6422:3a20:2232:0100:0000:0000:0000 - 1 packet to icmpv6(130)
   From 3037:3337:3431:3832:0100:0000:0000:0000 - 1 packet to icmpv6(130)
   From 3135:3332:3a38:3439:0100:0000:0000:0000 - 1 packet to icmpv6(130)
   From 3234:3238:323a:3834:0100:0000:0000:0000 - 1 packet to icmpv6(130)
   From 3333:3238:323a:3834:0100:0000:0000:0000 - 1 packet to icmpv6(130)
   From 3335:3839:323a:3834:0100:0000:0000:0000 - 1 packet to icmpv6(130)
   From 3336:3532:323a:3834:0100:0000:0000:0000 - 1 packet to icmpv6(130)
   From 3336:3837:3039:3132:0100:0000:0000:0000 - 8 packets to icmpv6(130)
   From 3430:3734:3330:3532:0100:0000:0000:0000 - 1 packet to icmpv6(130)
   From 3432:3230:3239:3a33:0100:0000:0000:0000 - 1 packet to icmpv6(130)
   From 3432:3635:3239:3a33:0100:0000:0000:0000 - 1 packet to icmpv6(130)
   From 3433:3230:3332:3a38:0100:0000:0000:0000 - 1 packet to icmpv6(130)
   From 3433:3534:3039:3a33:0100:0000:0000:0000 - 1 packet to icmpv6(130)
   From 3734:3237:3339:313a:0100:0000:0000:0000 - 1 packet to icmpv6(130)
   From 3734:3330:3238:313a:0100:0000:0000:0000 - 1 packet to icmpv6(130)
   From 3734:3330:3636:313a:0100:0000:0000:0000 - 1 packet to icmpv6(130)
   From 3734:3330:3930:323a:0100:0000:0000:0000 - 1 packet to icmpv6(130)
   From 3734:3334:3533:313a:0100:0000:0000:0000 - 1 packet to icmpv6(130)
   From 3734:3334:3736:393a:0100:0000:0000:0000 - 1 packet to icmpv6(130)
   From 616a:6f72:223a:2031:0100:0000:0000:0000 - 1 packet to icmpv6(130)
   From 6d65:6e74:7322:3a7b:0100:0000:0000:0000 - 9 packets to icmpv6(130)
   From 6f20:3134:3037:3433:0100:0000:0000:0000 - 2 packets to icmpv6(130)
   From 7473:223a:7b7d:7d00:0100:0000:0000:0000 - 1 packet to icmpv6(130)
   From 7473:223a:7b7d:7d32:0100:0000:0000:0000 - 1 packet to icmpv6(130)
   From 7473:223a:7b7d:7d3a:0100:0000:0000:0000 - 4 packets to icmpv6(130)
   From 7473:223a:7b7d:7d7b:0100:0000:0000:0000 - 3 packets to icmpv6(130)
   From 756e:6e69:6e67:223a:0100:0000:0000:0000 - 1 packet to icmpv6(130)
   From bf1d:f661:984e:fcb0:0100:0000:0000:0000 - 1 packet to icmpv6(130)
   From c011:fdda:43fe:4d59:65fe:e1aa:c7d5:683a - 1 packet to icmpv6(130)
   From c012:aa5c:173c:261c:0100:0000:0000:0000 - 1 packet to icmpv6(130)
   From c012:cfa3:e641:3b5f:0100:0000:0000:0000 - 1 packet to icmpv6(130)
   From c013:4b15:1c15:3311:0100:0000:0000:0000 - 1 packet to icmpv6(130)
   From c013:9389:bcf8:f7d4:0100:0000:0000:0000 - 1 packet to icmpv6(130)
   From c014:ff7d:0000:0000:0100:0000:0000:0000 - 1 packet to icmpv6(130)
   From e063:837f:0000:0000:0100:0000:0000:0000 - 3 packets to icmpv6(130)
   From e063:837f:2077:937f:0100:0000:0000:0000 - 1 packet to icmpv6(130)
   From e063:837f:301f:937f:0100:0000:0000:0000 - 3 packets to icmpv6(130)
   From 0.0.0.0 - 154 packets to igmp(0)
   From 10.0.3.4 - 154 packets to igmp(0)
   From 10.0.3.21 - 75 packets to igmp(0)
   From 10.0.3.209 - 2 packets to igmp(0)

 Listed by source hosts:
 Logged 18 packets on interface virbr0
   From fe80:0000:0000:0000:5054:00ff:feed:8246 - 18 packets to udp(5353)

 Listed by source hosts:
 Logged 50 packets on interface wlan0
   From 10.0.4.1 - 2 packets to udp(68)
   From 64.233.186.188 - 15 packets to tcp(44421,53418)
   From 64.235.151.8 - 5 packets to tcp(56214)
   From 173.194.42.0 - 1 packet to tcp(56677)
   From 173.194.42.21 - 3 packets to tcp(58597)
   From 173.194.42.22 - 7 packets to tcp(35536)
   From 173.194.42.75 - 3 packets to tcp(48585)
   From 173.194.42.85 - 4 packets to tcp(44550)
   From 173.194.42.86 - 1 packet to tcp(51940)
   From 192.168.1.1 - 2 packets to udp(68)
   From 192.168.2.1 - 2 packets to udp(68)
   From 195.135.221.134 - 1 packet to tcp(51756)
   From 195.154.174.66 - 1 packet to tcp(58047)
   From 200.42.136.212 - 3 packets to tcp(59351,59361)
\end{lstlisting}

\section{Opciones de rsync}
\label{sec:rsync}
\begin{lstlisting}
$ rsync --help
rsync  version 3.1.0  protocol version 31
Copyright (C) 1996-2013 by Andrew Tridgell, Wayne Davison, and others.
Web site: http://rsync.samba.org/
Capabilities:
    64-bit files, 64-bit inums, 32-bit timestamps, 64-bit long ints,
    socketpairs, hardlinks, symlinks, IPv6, batchfiles, inplace,
    append, ACLs, xattrs, iconv, symtimes, prealloc

rsync comes with ABSOLUTELY NO WARRANTY.  This is free software, and you
are welcome to redistribute it under certain conditions.  See the GNU
General Public Licence for details.

rsync is a file transfer program capable of efficient remote update
via a fast differencing algorithm.

Usage: rsync [OPTION]... SRC [SRC]... DEST
  or   rsync [OPTION]... SRC [SRC]... [USER@]HOST:DEST
  or   rsync [OPTION]... SRC [SRC]... [USER@]HOST::DEST
  or   rsync [OPTION]... SRC [SRC]... rsync://[USER@]HOST[:PORT]/DEST
  or   rsync [OPTION]... [USER@]HOST:SRC [DEST]
  or   rsync [OPTION]... [USER@]HOST::SRC [DEST]
  or   rsync [OPTION]... rsync://[USER@]HOST[:PORT]/SRC [DEST]
The ':' usages connect via remote shell, while '::' & 'rsync://' usages connect
to an rsync daemon, and require SRC or DEST to start with a module name.

Options
 -v, --verbose               increase verbosity
     --info=FLAGS            fine-grained informational verbosity
     --debug=FLAGS           fine-grained debug verbosity
     --msgs2stderr           special output handling for debugging
 -q, --quiet                 suppress non-error messages
     --no-motd               suppress daemon-mode MOTD (see manpage caveat)
 -c, --checksum              skip based on checksum, not mod-time & size
 -a, --archive               archive mode; equals -rlptgoD (no -H,-A,-X)
     --no-OPTION             turn off an implied OPTION (e.g. --no-D)
 -r, --recursive             recurse into directories
 -R, --relative              use relative path names
     --no-implied-dirs       don't send implied dirs with --relative
 -b, --backup                make backups (see --suffix & --backup-dir)
     --backup-dir=DIR        make backups into hierarchy based in DIR
     --suffix=SUFFIX         set backup suffix (default ~ w/o --backup-dir)
 -u, --update                skip files that are newer on the receiver
     --inplace               update destination files in-place (SEE MAN PAGE)
     --append                append data onto shorter files
     --append-verify         like --append, but with old data in file checksum
 -d, --dirs                  transfer directories without recursing
 -l, --links                 copy symlinks as symlinks
 -L, --copy-links            transform symlink into referent file/dir
     --copy-unsafe-links     only "unsafe" symlinks are transformed
     --safe-links            ignore symlinks that point outside the source tree
     --munge-links           munge symlinks to make them safer (but unusable)
 -k, --copy-dirlinks         transform symlink to a dir into referent dir
 -K, --keep-dirlinks         treat symlinked dir on receiver as dir
 -H, --hard-links            preserve hard links
 -p, --perms                 preserve permissions
 -E, --executability         preserve the file's executability
     --chmod=CHMOD           affect file and/or directory permissions
 -A, --acls                  preserve ACLs (implies --perms)
 -X, --xattrs                preserve extended attributes
 -o, --owner                 preserve owner (super-user only)
 -g, --group                 preserve group
     --devices               preserve device files (super-user only)
     --specials              preserve special files
 -D                          same as --devices --specials
 -t, --times                 preserve modification times
 -O, --omit-dir-times        omit directories from --times
 -J, --omit-link-times       omit symlinks from --times
     --super                 receiver attempts super-user activities
     --fake-super            store/recover privileged attrs using xattrs
 -S, --sparse                handle sparse files efficiently
     --preallocate           allocate dest files before writing them
 -n, --dry-run               perform a trial run with no changes made
 -W, --whole-file            copy files whole (without delta-xfer algorithm)
 -x, --one-file-system       don't cross filesystem boundaries
 -B, --block-size=SIZE       force a fixed checksum block-size
 -e, --rsh=COMMAND           specify the remote shell to use
     --rsync-path=PROGRAM    specify the rsync to run on the remote machine
     --existing              skip creating new files on receiver
     --ignore-existing       skip updating files that already exist on receiver
     --remove-source-files   sender removes synchronized files (non-dirs)
     --del                   an alias for --delete-during
     --delete                delete extraneous files from destination dirs
     --delete-before         receiver deletes before transfer, not during
     --delete-during         receiver deletes during the transfer
     --delete-delay          find deletions during, delete after
     --delete-after          receiver deletes after transfer, not during
     --delete-excluded       also delete excluded files from destination dirs
     --ignore-missing-args   ignore missing source args without error
     --delete-missing-args   delete missing source args from destination
     --ignore-errors         delete even if there are I/O errors
     --force                 force deletion of directories even if not empty
     --max-delete=NUM        don't delete more than NUM files
     --max-size=SIZE         don't transfer any file larger than SIZE
     --min-size=SIZE         don't transfer any file smaller than SIZE
     --partial               keep partially transferred files
     --partial-dir=DIR       put a partially transferred file into DIR
     --delay-updates         put all updated files into place at transfer's end
 -m, --prune-empty-dirs      prune empty directory chains from the file-list
     --numeric-ids           don't map uid/gid values by user/group name
     --usermap=STRING        custom username mapping
     --groupmap=STRING       custom groupname mapping
     --chown=USER:GROUP      simple username/groupname mapping
     --timeout=SECONDS       set I/O timeout in seconds
     --contimeout=SECONDS    set daemon connection timeout in seconds
 -I, --ignore-times          don't skip files that match in size and mod-time
 -M, --remote-option=OPTION  send OPTION to the remote side only
     --size-only             skip files that match in size
     --modify-window=NUM     compare mod-times with reduced accuracy
 -T, --temp-dir=DIR          create temporary files in directory DIR
 -y, --fuzzy                 find similar file for basis if no dest file
     --compare-dest=DIR      also compare destination files relative to DIR
     --copy-dest=DIR         ... and include copies of unchanged files
     --link-dest=DIR         hardlink to files in DIR when unchanged
 -z, --compress              compress file data during the transfer
     --compress-level=NUM    explicitly set compression level
     --skip-compress=LIST    skip compressing files with a suffix in LIST
 -C, --cvs-exclude           auto-ignore files the same way CVS does
 -f, --filter=RULE           add a file-filtering RULE
 -F                          same as --filter='dir-merge /.rsync-filter'
                             repeated: --filter='- .rsync-filter'
     --exclude=PATTERN       exclude files matching PATTERN
     --exclude-from=FILE     read exclude patterns from FILE
     --include=PATTERN       don't exclude files matching PATTERN
     --include-from=FILE     read include patterns from FILE
     --files-from=FILE       read list of source-file names from FILE
 -0, --from0                 all *-from/filter files are delimited by 0s
 -s, --protect-args          no space-splitting; only wildcard special-chars
     --address=ADDRESS       bind address for outgoing socket to daemon
     --port=PORT             specify double-colon alternate port number
     --sockopts=OPTIONS      specify custom TCP options
     --blocking-io           use blocking I/O for the remote shell
     --stats                 give some file-transfer stats
 -8, --8-bit-output          leave high-bit chars unescaped in output
 -h, --human-readable        output numbers in a human-readable format
     --progress              show progress during transfer
 -P                          same as --partial --progress
 -i, --itemize-changes       output a change-summary for all updates
     --out-format=FORMAT     output updates using the specified FORMAT
     --log-file=FILE         log what we're doing to the specified FILE
     --log-file-format=FMT   log updates using the specified FMT
     --password-file=FILE    read daemon-access password from FILE
     --list-only             list the files instead of copying them
     --bwlimit=RATE          limit socket I/O bandwidth
     --outbuf=N|L|B          set output buffering to None, Line, or Block
     --write-batch=FILE      write a batched update to FILE
     --only-write-batch=FILE like --write-batch but w/o updating destination
     --read-batch=FILE       read a batched update from FILE
     --protocol=NUM          force an older protocol version to be used
     --iconv=CONVERT_SPEC    request charset conversion of filenames
     --checksum-seed=NUM     set block/file checksum seed (advanced)
 -4, --ipv4                  prefer IPv4
 -6, --ipv6                  prefer IPv6
     --version               print version number
(-h) --help                  show this help (-h is --help only if used alone)

Use "rsync --daemon --help" to see the daemon-mode command-line options.
Please see the rsync(1) and rsyncd.conf(5) man pages for full documentation.
See http://rsync.samba.org/ for updates, bug reports, and answers
\end{lstlisting}

\section{Ejemplo completo de configuración DRBD}
\label{sec:confdrbd}

\begin{lstlisting}
resource example {
	options {
		on-no-data-accessible suspend-io;
	}

	net {
		cram-hmac-alg "sha1";
		shared-secret "secret_string";
	}

	# The disk section is possible on resource level and in each
	# volume section
	disk {
		# If you have a resonable RAID controller
		# with non volatile write cache (BBWC, flash)
		disk-flushes no;
		disk-barrier no;
		md-flushes no;
	}

	# volume sections on resource level, are inherited to all node
	# sections. Place it here if the backing devices have the same
	# device names on all your nodes.
	volume 1 {
		device minor 1;
		disk /dev/sdb1;
		meta-disk internal;

		disk {
			resync-after example/0;
		}
	}

	on wurzel {
		address	192.168.47.1:7780;

		volume 0 {
		       device minor 0;
		       disk /dev/vg_wurzel/lg_example;
		       meta-disk /dev/vg_wurzel/lv_example_md;
		}
	}
	on sepp {
		address	192.168.47.2:7780;

		volume 0 {
		       device minor 0;
		       disk /dev/vg_sepp/lg_example;
		       meta-disk /dev/vg_sepp/lv_example_md;
		}
	}
}

resource "ipv6_example_res" {
	net {
		cram-hmac-alg "sha1";
		shared-secret "ieho4CiiUmaes6Ai";
	}

	volume 2 {
		device	"/dev/drbd_fancy_name" minor 0;
		disk	/dev/vg0/example2;
		meta-disk internal;
	}

	on amd {
		# Here is an example of ipv6.
		# If you want to use ipv4 in ipv6 i.e. something like [::ffff:192.168.22.11]
		# you have to set disable-ip-verification in the global section.
		address	ipv6 [fd0c:39f4:f135:305:230:48ff:fe63:5c9a]:7789;
	}

	on alf {
		address ipv6 [fd0c:39f4:f135:305:230:48ff:fe63:5ebe]:7789;
	}
}


#
# A two volume setup with a node for disaster recovery in an off-site location.
#

resource alpha-bravo {
	net {
		cram-hmac-alg "sha1";
		shared-secret "Gei6mahcui4Ai0Oh";
	}

	on alpha {
		volume 0 {
			device minor 0;
			disk /dev/foo;
			meta-disk /dev/bar;
		}
		volume 1 {
			device minor 1;
			disk /dev/foo1;
			meta-disk /dev/bar1;
		}
		address	192.168.23.21:7780;
	}
	on bravo {
		volume 0 {
			device minor 0;
			disk /dev/foo;
			meta-disk /dev/bar;
		}
		volume 1 {
			device minor 1;
			disk /dev/foo1;
			meta-disk /dev/bar1;
		}
		address	192.168.23.22:7780;
	}
}

resource stacked_multi_volume {
	net {
		protocol A;

		on-congestion pull-ahead;
		congestion-fill 400M;
		congestion-extents 1000;
	}

	disk {
		c-fill-target 10M;
	}

	volume 0 { device minor 10; }
	volume 1 { device minor 11; }

	proxy {
		memlimit 500M;
		plugin {
			lzma contexts 4 level 9;
		}
	}

	stacked-on-top-of alpha-bravo {
		address	192.168.23.23:7780;

		proxy on charly {
			# In the regular production site, there is a dedicated host to run
			# DRBD-proxy
			inside    192.168.23.24:7780; # for connections to DRBD
			outside   172.16.17.18:7780; # for connections over the WAN or VPN
			options {
				memlimit 1G; # Additional proxy options are possible here
			}
		}
	}
	on delta {
		volume 0 {
			device minor 0;
			disk /dev/foo;
			meta-disk /dev/bar;
		}
		volume 1 {
			device minor 1;
			disk /dev/foo1;
			meta-disk /dev/bar1;
		}
		address	127.0.0.2:7780;

		proxy on delta {
			# In the DR-site the proxy runs on the machine that stores the data
			inside 127.0.0.1:7780;
			outside 172.16.17.19:7780;
		}
	}
}

resource drbd_9_two_connection {
	volume 0 {
	       device minor 10;
	       disk /dev/foo/bar;
	       meta-disk internal;
	}

	on alpha {
		node-id 0;
		address 192.168.31.1:7800;
	}
	on bravo {
		node-id 1;
		address 192.168.31.2:7800;
	}
	on charlie {
		node-id 2;
		address 192.168.31.3:7800;
	}

	net {
		ko-count 3;
	}

	connection "optional name" {
		host alpha;
		host bravo;
		net { protocol C; }
	}

	connection {
		host alpha address 127.0.0.1:7800 via proxy on alpha {
			inside 127.0.0.2:7800;
			outside 192.168.31.1:7801;
		}
		host charlie address 127.0.0.1:7800 via proxy on charlie {
			inside 127.0.0.2:7800;
			outside 192.168.31.3:7800;
		}
		net { protocol A; }
	}

	connection {
		host bravo address 127.0.0.1:7800 via proxy on bravo {
			inside 127.0.0.2:7800;
			outside 192.168.31.2:7801;
		}
		host charlie address 127.0.0.1:7800 via proxy on charlie {
			inside 127.0.0.2:7800;
			outside 192.168.31.3:7800;
		}
		net { protocol A; }
	}
}

resource drbd_9_mesh {
	volume 0 {
	       device minor 11;
	       disk /dev/foo/bar2;
	       meta-disk internal;
	}

	on alpha {
		node-id 0;
		address 192.168.31.1:7900;
	}
	on bravo {
		node-id 1;
		address 192.168.31.2:7900;
	}
	on charlie {
		node-id 2;
		address 192.168.31.3:7900;
	}

	connection-mesh {
		hosts alpha bravo charlie;
		net {
			protocol C;
		}
	}
}
\end{lstlisting}

%------------------------------------------------------------

\end{document}