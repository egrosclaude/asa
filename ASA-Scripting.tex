\section {Contenidos}

\begin{enumerate}
\item Comandos básicos de archivos ls, cd, mkdir, cp, mv, rm, ln, patrones de nombres
\item Redirección y piping, comandos head, tail, more, less, grep
\item Variables, ambiente, aritmética
\item Sentencias de control if, for, while, case
\item Funciones
\item Arreglos
\item Expresiones regulares, uso de grep
\item Uso de sort, diff, comm, uniq, cut
\item Uso de cron
\item Otros intérpretes: sed, awk, Perl
\end{enumerate}

\section{Ejercitación básica}

\subsection{Redirección y piping}
\begin{enumerate}
	\item Crear un archivo conteniendo la salida del comando ls
	\item Crear un archivo conteniendo la salida del comando ls -lR /tmp
	\item Obtener las cinco primeras líneas del archivo anterior
	\item Crear un archivo conteniendo las cinco primeras líneas y las cinco  últimas del archivo generado en 2
	\item Crear un archivo conteniendo las primeras cinco líneas de la salida del comando ls -lR /tmp
	\item Usando el anterior, crear un archivo conteniendo esas líneas, numeradas
	\item Crear un archivo conteniendo las últimas cinco líneas de la salida del comando ls -lR /tmp
\end{enumerate}


\subsection {Variables, ambiente}
\begin{enumerate}
	\item Asignar e imprimir el contenido de dos variables
	\item Asignar dos variables, imprimir sus valores, intercambiar sus valores, imprimirlos
	\item Crear un script que imprima un valor que será pasado como argumento
	\item Crear un script que imprima dos valores que serán pasados como argumento
	\item Crear un script que imprima todos los valores que le sean pasados como argumento
\end{enumerate}


\subsection{Sentencias de control}
\begin{enumerate}
	\item 
Imprimir cinco veces "Linux"
	\item 
Imprimir cinco veces el contenido de una variable
	\item 
Imprimir los números de 0 a 5
	\item 
Imprimir los dígitos de -1 a 6
	\item 
Imprimir los números de 0 a 99
	\item 
Imprimir junto al nombre de cada archivo en el directorio actual, su tamaño y su fecha de modificación
	\item 
Copiar los archivos terminados en .txt en archivos con igual nombre pero extensión .bak
	\item 
Renombrar los archivos con extensión .tex que comienzan en ASA reemplazando la partícula ASA con RII
	\item 
Para cada archivo modificado hace más de cinco días en un directorio, mostrar su cantidad de líneas
	\item 
Obtener mediante un cliente de HTTP una lista de archivos cuyos nombres están dados por  una expresión variable y controlada por un lazo
	\item 
De un conjunto de archivos tar, encontrar aquellas versiones de un archivo dado, contenido en ellos, que hayan sido modificadas entre dos fechas dadas.
\end{enumerate}


\subsection{Aritmética}
\begin{lstlisting}
$ declare -i num
$ num="hola"
$ echo $num
	0
$ num=5 + 5
	bash: +: command not found
$ num=5+5
$ echo $num
	10
$ num=4*6
$ echo $num
	24
$ num="4 * 6"
$ echo $num
	24
$ num=6.5
	bash: num: 6.5: syntax error in expression (remainder of expression is ".5")
$ i=5; j=$i+1; echo $j
$ i=5; let j=$i+1; echo $j
$ let i=5
$ let i=i+1
$ echo $i
	6
$ let "i = i + 2"
$ echo $i
	8
$ let "i+=1"
$ echo $i
	9
$ i=3
$ (( i+=4))
$ echo $i
	7
$ (( i=i-2 ))
$ echo $i
	5
$ let b=2#101; echo $b
$ let h=16#ABCD; echo $h
\end{lstlisting}

\subsection{Arreglos}
\begin{lstlisting}
$ A=(1 2 3 cuatro cinco)
$ echo ${!A[*]}
0 1 2 3 4
$ echo ${A[4]}
cinco
$ echo ${A[*]}
1 2 3 cuatro cinco
$ A[2]='banana'
$ echo ${A[*]}
1 2 banana cuatro cinco
\end{lstlisting}

\subsection{Arreglos asociativos}
\begin{lstlisting}
$ declare -A B
$ B=([francia]='paris' [espana]='madrid' [argentina]='buenos aires')
$ echo ${!B[*]}
espana argentina francia
$ echo ${B[*]}
madrid buenos aires paris
$ echo ${B[francia]}
paris
\end{lstlisting}


\subsection{Here-Documents}
\begin{lstlisting}
$ cat > texto.txt << END
> Hola
> Probando...
> END
$ cat texto.txt
\end{lstlisting}

\subsection{Traps}
\begin{lstlisting}
# man 7 signal
# 1 = SIGHUP (Hangup of controlling terminal or death of parent)
# 2 = SIGINT (Interrupted by the keyboard)
# 3 = SIGQUIT (Quit signal from keyboard)
# 6 = SIGABRT (Aborted by abort(3))
# 9 = SIGKILL (Sent a kill command)

trap limpieza 1 2 3 6 9

function limpieza
{
	echo "Recibimos senal - desmantelando..."
	# Borrar archivos temporarios
	rm -f ${tempfiles}
	echo Finalizando
}
\end{lstlisting}



\section{Casos de uso}


\subsection{Investigar el sistema}
\begin{enumerate}
	\item 
Modificar la salida del comando blkid para conocer el UUID, el nombre y tipo, y punto de montado, de cada dispositivo de bloques del sistema.
	\item 
Analizar archivos de log buscando conocimiento: duración de sesiones ssh por usuario, mensajes de mail entre usuarios, con histograma por tamaños, etc. (ver iptables.log, \ref{subsec:iptables.log})
	\item 
Detectar momentos en que la salida de vmstat muestra picos de I/O, procesos corriendo, procesos en espera, uso de swap, etc.
\end{enumerate}


\subsection{Recuperar espacio de almacenamiento}
\begin{enumerate}
	\item Encontrar los diez archivos más grandes en un directorio y sus hijos, imprimirlos junto con su tamaño de mayor a menor.
	\item Encontrar los diez archivos más grandes en un directorio y sus hijos, moverlos a otro directorio (en otro filesystem).
	\item Encontrar los diez archivos más grandes del sistema, imprimir el nombre de usuario dueño.
	\item Agregar al script anterior el envío de notificación por mail al usuario responsable.
		\item 
Encontrar archivos en directorios de usuario con la cadena \quotes{cache} en su nombre e imprimir el uso de disco de cada uno.
	\item 
Idem, enviando nombres a un archivo y usándolo como lista para borrarlos, comprimirlos o moverlos.

\end{enumerate}



\subsection{Networking}
\begin{enumerate}
	\item 
Disparar un aviso cuando se pierde la conectividad a un conjunto dado de nodos de la red.
	\item 
Analizar la salida del comando netstat para descubrir en qué momento aparece un nuevo port abierto y a qué aplicación corresponde.
	\item 
Obtener un log de tráfico y obtener orígenes máximos y mínimos de tráfico, cantidades totales de bytes traficados por interfaz, etc.
	\item 
Recoger estadísticas de espacio en disco, cantidad de procesos, carga de CPU, en diferentes nodos de la red, y centralizarlos en un nodo monitor que presente los resultados.
\end{enumerate}

\subsection{Seguridad}
\begin{enumerate}
	\item 
Detener el script si la identidad del proceso corresponde a root.
	\item 
Solicitar información confidencial (como claves) con video inhibido.
	\item 
Capturar señales para impedir la interrupción del script por BREAK o fallos de ejecución.
	\item 
Utilizar MD5/SHAx para confirmar integridad de archivos.
\end{enumerate}




\subsection{Tratamiento de datos}
\begin{enumerate}
	\item 
Revisar el uso de los comandos cut, join, sort, uniq, comm.
	\item 
Crear script que administra una base de datos en formato CSV.
	\item 
Dado un archivo con una lista de direcciones IP, adjuntarles la resolución inversa de nombres correspondiente.
	\item 
Crear un histograma de accesos por nombre de dominio, a partir de los paquetes registrados en un archivo de log generado por iptables. 
	\item 
Dada una base de datos CSV implementar búsqueda por expresiones regulares.
	\item 
Dada una base de datos CSV implementar proyección sobre un conjunto de campos dados.
	\item 
Convertir un listado de individuos PDF en archivo CSV.
	\item 
Preparar un conjunto de scripts con un único punto de entrada para el administrador. Estos scripts mantendrán un conjunto de bases de datos en formato CSV:
\begin{lstlisting}
alumnos: UID, Username, Apellido, Nombres, NoLegajo, Activo
materias: MID, Nombre, Carrera, Docente
cursadas: UID, MID, Ano, Cuatrimestre
\end{lstlisting}
El dato Activo es booleano. Con estas bases de datos:
\begin{itemize}
	\item 
Listar todas las materias asignadas a un mismo docente.
	\item 
Listar todas las materias cursadas por un alumno.
	\item 
Listar todos los alumnos activos inscriptos en una materia.
	\item 
Listar todos los alumnos que cursan una misma carrera dada durante un año dado.
	\item 
Listar todos los alumnos, agrupados por materia cursada, dentro de cada año. 
	\item 
Listar todos los alumnos de un mismo docente.
	\item 
Dado un alumno por su legajo, consultar su estado Activo/Inactivo.
	\item 
Para aquellos alumnos que hace más de tres años que no se inscriben en ninguna cursada, pasar su dato Activo a falso (Inactivo).
	\item 
Generar un par de archivos en el formato de /etc/passwd y /etc/shadow para todos los alumnos activos.
	\item 
Generar un directorio /home/usuario para cada alumno activo, con UID correspondiente.
\end{itemize}
\end{enumerate}


\subsection{Accesibilidad para usuarios finales}
\begin{enumerate}
	\item 
Preparar un script con interfaz gráfica para copiar archivos seleccionados a una carpeta preestablecida con el fin de obtener un backup periódico de todos sus contenidos.
	\item 
Preparar un script con interfaz gráfica que presente los cinco directorios con mayor ocupación de almacenamiento dentro del home del usuario.
	\item 
Agregar interfaz gráfica a los scripts de administración de bases de datos de alumnos y materias.
\end{enumerate}


