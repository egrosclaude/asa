Los equipos de computación para usuarios finales serán cada vez más
poderosos. La virtualización puede ser una forma de obtener aún mayor
rendimiento de estos equipos, convirtiéndolos en el equivalente de
varias máquinas a la vez. La charla muestra qué son y para qué sirven
las máquinas virtuales y muestra algunas formas de virtualización, con
sus diferentes alcances, requerimientos y características.
Temario
* Conceptos
	* Sistemas Operativos
		* Kernel, hardware y drivers de dispositivos
		* Virtualización de dispositivos
			* Descubrimiento de dispositivos
				* PCI
			* Control de dispositivos
				* Iniciar operaciones
			* Transferencias de datos 
				* Dispositivo<->memoria
				* DMA
			* Control de Interrupciones
				* Capacidad de notificar al sw de eventos
		* Ejemplos de Virtualización
			* RAM Disk
			* RAID
			* VPN
			* VFS
	* VMM (Virtual Machine Monitor)
		* VMM Tipo 1 (nativo): corre directamente sobre hardware físico 
		* VMM Tipo 2 (hosted): corre como proceso del SO y ofrece un anillo 0 simulado a las MV
		* De procesos o de aplicación (Java VM)
		* De sistema (VMWare)
	* Virtualización de dispositivos
		* CPU
		* E/S
		* Discos
		* Redes
	* Protección y virtualización
	* Máquinas virtuales
		* Guests y Hosts
	* Ejemplos
		* Propietarios
			* :osobox:old:zim:Público:Software Libre:VmWare
			* :osobox:old:zim:Público:Software Libre:VirtualPC
			* Virtuozzo
		* GPL
			* :osobox:old:zim:Público:Software Libre:VirtualBox
			* QEMU
			* Xen
			* KVM
		* Otros
			* WINE
			* Colinux
			* UML
			* :osobox:old:zim:Público:Software Libre:OpenVZ
			* Vserver
* Aplicaciones para el usuario final
	* Revolución del multicore
		* Muchos equipos en uno
	* Sistemas legacy
	* Probar nuevas distribuciones u otro software
	* Probar actualizaciones
* Aplicaciones para el administrador de sistemas
	* Independencia del hardware
	* Provisioning
	* Live Migration para balance de carga
	* Hosting de servicios
	* Consolidación en el server room
		* Menos espacio
		* Menos consumo
		* Menos calor
* Clases de virtualización
	* Emulación 
	* Paravirtualización
	* Virtualización asistida por hardware
* Ejemplos
	* QEMU
		* Emulación
		* Módulo acelerador
	* Xen
		* Hypervisor
		* Para/Full Virt
		* Red default bridge
	* KVM
		* Full Virt
		* Red default, NAT
* Tendencias
	* Hardware cada vez más orientado a virtualización
	* Multicore/manycore
	* VMM monolítico con apoyo del hardware
